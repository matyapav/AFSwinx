\chapter{Analýza}
\section{Funkční specifikace}
V rámci této práce bude zpracován framework ve dvou verzích, pro mobilní platformu Android a mobilní platformu Windows Phone. Musí umožňovat jednoduše vytvářet dva typy komponent, formulář, který umožní uživatelský vstup a list, pro zobrazení většího množství dat uživateli. Kromě vytvoření komponent je nutné poskytkout další funkce, které umožní práci s vytvořenými komponentami, jako je například odeslání dat z komponenty na server. Framework musí samozřejmě disponovat funkcionalitou, která umožní správné vytvoření a nastavení komponenty z hlediska zabezpečení, získávání dat a jejich vložení do komponenty, vzhledu komponenty či její lokalizace. Všechny funkční požadavky jsou uvedeny v následujícím seznamu položek.
\subsection{Funkční požadavky}

Framework by měl splňovat následující požadavky.
\begin{itemize}
\item Framework bude umožňovat automaticky vytvořit formulář nebo list na základě dat získaných ze serveru.
\item Framework bude umožňovat získat ze serveru data, kterými komponentu naplní.
\item Framework bude umožňovat naplnit formulář i list daty.
\item Framework bude umožňovat odeslat data z formuláře zpět na server.
\item Framework bude umožňovat používat lokalizační texty.
\item Framework bude umožňovat validaci vstupních dat na základě definice komponenty, kterou obdržel od serveru.
\item Framework bude umožňovat upravit vzhled komponenty pomocí skinů.
\item Framework bude umožňovat koncovému uživateli specifikovat zdroje definic komponent, dat a cíle pro jejich odeslání ve formátu XML.
\item Framework bude umožňovat vytvářet následující formulářová pole - textové, číselné, pro hesla, pro datum, dropdown pole, checkboxy, option buttony.
\item Framework bude umožňovat resetovat úpravy ve formuláři nebo formulář vyčistit.
\item Framework bude umožňovat získat data z formuláře i listu.
\item Framework bude umožňovat schovat validační chyby.
\item Framework bude umožňovat jednoduše získat komponentu i na jiném místě v programu, než kde ji vytvořil.
\item Framework bude umožňovat generování komponent určených pouze pro čtení. 
\end{itemize}

Pro uživatele, který bude framework využívat, bude proces tvorby komponenty zapouzdřen. Nemusí znát strukturu definice komponenty ani jak se komponenta tvoří či naplňuje daty. Bude potřebovat znát jen kód pro vytvoření komponenty, akce, které lze nad komponentou provádět a jak specifikovat, odkud se bere definice komponenty, data pro její naplnění a kam se případně data odešlou.

\section{Popis architektury a komunikace}
\subsection{Definice komponent}
Frameworky pro mobilní platformy Android a Windows Phone, které je cílem vytvořit, navazují, jak už bylo zmíněno, na projekt AFSwinx \cite{tomasek-thesis}. Tento framework vytváří na straně serveru tzv. definice komponent, které komponentu popisují z hlediska vzhledu, rozložení i obsahu. Jedná se tedy o metadata\cite{https://en.wikipedia.org/wiki/Metadata}, neboli data, která popisují další data. Taková definice vzniká na serveru na základě inspekce modelu. Za tu vděčíme knihovně AspectFaces, která inspekci vytváří ve formátu XML a AFSwinx ji zobecňuje a převádí do formátu JSON. Na serveru zastupuje roli modelu databázová entita a vlastnosti, které má inspekce modelu zachytit a do definice promítnout, jsou určeny datovými typy atributů a pomocí anotací.
Definice komponenty, kterou lze získat ze serveru, obsahuje tyto infomace -
\begin{itemize}
\item název definice,
\item celkové rozložení komponenty,
\item informace o polích v daném pořadí, které se mají v komponentě vyskytnout.
\end{itemize}
V informacích o poli lze nalézt -
\begin{itemize}
\item typ widgetu, kterým má být vytvářené pole reprezentováno,
\item jednoznačný identifikátor v rámci komponenty,
\item popisek neboli label,
\item viditelnost,
\item zda má být pole určeno jen pro čtení,
\item zda se jedná o primitivní či složený datový typ,
\item validační pravidla,
\item v případě některých widgetů i možnosti, ze kterých si má uživatel vybírat.
\end{itemize}

Cílem autora AFSwinx bylo, aby tyto definice komponent byly nezávislé na platformě, což se i jejich využitím na mobilních platformách potvrdí. S tvarem těchto definic bude framework počítat, na základě toho se definice bude na klientovi udržovat a z ní vytvářet uživatelské rozhraní. 

\subsection{Získání definice ze serveru}
Definici komponenty je možné získat pomocí HTTP dotazu na konkrétní zdroj na serveru, který AFSwinx používá a je schopen nám takovou definici poskytnout. Tento konkrétní zdroj, poskytující definici komponenty, je nutné specifikovat. Také lze určit dva další zdroje - zdroj dat a zdroj, na který budeme odesílat uživatelský vstup. Framework by měl, dle požadavků výše, umožňovat uživateli tyto zdroje specifikovat ve formátu XML. Již v AFSwinx byl pro to vytvořen XML soubor a k němu XML parser. V Android frameworku bude žádoucí z hlediska efektivity tento parser a soubor využít. Jelikož AFSwinx je napsaný v Javě a Windows Phone nepodporuje Javu, nýbrž jazyk C\#, a tedy ani import .jar souborů, bude nutné tento parser přepsat. Způsob specifikace všech tří zmíněných zdrojů, konkrétně pro profilový formulář, je zobrazen v následujícím urývku kódu \ref{code:xmlSource} ze zmíněného XML souboru.

\begin{lstlisting}[caption=Ukázka XML specifikace zdrojů,
label={code:xmlSource}, basicstyle=\footnotesize]
<?xml version="1.0" encoding="UTF-8"?>
<connectionRoot xmlns:xsi="http://www.w3.org/2001/XMLSchema-instance">
   <connection id="personProfile">
      <metaModel>
         <endPoint>toms-cz.com</endPoint>
         <endPointParameters>/AFServer/rest/users/profile</endPointParameters>
         <protocol>http</protocol>
         <port></port>
         <header-param>
            <param>content-type</param>
            <value>Application/Json</value>
         </header-param>
      </metaModel>
      <data>
         <endPoint>toms-cz.com</endPoint>
         	<!-- ... obdobne jako metamodel-->
         <security-params>
            <security-method>basic</security-method>
            <userName>#{username}</userName>
            <password>#{password}</password>
         </security-params>
      </data>
      <send>
         <endPoint>toms-cz.com</endPoint>
            	<!--... obdobne jako metamodel -->
         <security-params>
                <!--... obdobne jako data -->
         </security-params>
      </send>
   </connection>
</connectionRoot>
\end{lstlisting}

Jak lze z ukázky vidět, jsou zdroje nadefinovány URL adresou rozdělenou na části a dodatečnými parametry, jako je forma dat, která lze očekávat nebo zabezpečení. Například definice profilového formuláře se nachází na adrese \url{http://toms-cz.com/AFServer/rest/users/profile} a očekáváme ho ve tvaru JSON souboru. Pokud by byl specifikován port, přibude za toms-cz.com ještě dvojtečka a jeho hodnota. Zdroj dat je nadefinován v uzlu <data> a kam se odešle uživatelský vstup určuje uzel <send>. Lze také specifikovat metodu (get, post, put, delete), která se použije, pro data a meta model je v základu použita metoda GET a pro odeslání metoda POST.

Výrazy ve složených závorkách označené vpředu hashtagem jsou určeny k nahrazení. V AFSwinx \cite{tomasek-thesis} se klíč ve složených závorkách hledá v mapě parametrů pro připojení, kterou framework předává jako argument metodě kontaktující zdroj, a nahrazuje se hodnotou v ní pod klíčem uloženou. Umožňuje to tak nadefinovat zdroj v XML souboru pouze jednou, například pro více různých uživatelů. Klíč a hodnotu si může uživatel nastavit sám, jen se musí shodovat klíče ve zmíněné mapě a v souboru. V zájmu znovupoužití XML souboru a parseru se tedy tomuto chování budou muset oba tvořené frameworky přizpůsobit.

\subsection{Reprezentace metadat ve frameworku}
Získaná metadata je potřeba ve frameworku nějak rozumně udržovat. AFSwinx už pro to určitou strukturu definuje \cite{tomasek-thesis} a je tedy žádoucí ji opětovně využít. Tuto část systému, která uchová získané informace o komponentě, zachycuje následující doménový model, vytvořeného za pomocí UML v programu Enterprise Architect. Dle Arlowa je UML, česky unifikovaný modelovací jazyk, univerzální jazyk pro vizuální modelování systémů. UML je velice silný nástroj hlavně proto, že je srozumitelný pro lidi a zároveň je navržen tak, aby byl univerzálně implementovatelný \cite{UmlArlow}. Doménový model definuje jaké části je potřeba v systému mít a jak se vzájemně ovlivňují. Jde tedy o model popisující strukturu i chování systému.

\begin{figure}[h!]
\includegraphics[width=\textwidth]{figures/domainModel}
\caption{Doménový model objektů obsahující metadata o komponentě}
\label{img:metadataModel}
\end{figure}

Následující část práce podrobněji rozebírá diagram z obrázku \ref{img:metadataModel}, neboť je pro vývoj obou frameworků důležitý. Android framework bude tuto část AFSwinx znovu používat a pro Windows Phone bude nutné tuto část přepsat do jazyka C\#.
\subsubsection{AFClassInfo}
AFClassInfo udržuje informace o hlavním objektu metadat. Obsahuje informace o názvu objektu a rozložení komponenty. Dále drží definice 0 až N informací o polích, která se mají v komponentě vyskytnout. Součástí je také 0 až N vnitřích tříd, tedy referencí na objekt stejného typu AFClassInfo. Může se totiž stát, že v modelu, nad kterým je prováděna inspekce a ze kterého se metadata vytváří, obsahuje neprimitivní datový typ. Například v modelu Osoba to může být objekt typu Adresa, který obsahuje další atributy jako třeba název ulice či město. Tento typ je ale nutno v komponentě reprezentovat také, a tak je zevnitř provedena jeho inspekce, která je později v metadatech reprezentována jako vnitřní třída. 

\subsubsection{AFFieldInfo}
Tento objekt popisuje jednu proměnnou, nad kterou byla provedena inspekce a ze které se má vytvořit pole, které se v komponentě vyskytne. Informuje jaký widget má být při vytváření pole použit, určuje jednoznačný identifikátor pole v rámci komponenty, dále pak, zda má být tvořené pole viditelné a upravitelné, repektive jen pro čtení. Definuje jak bude pole rozloženo, hlavně z hlediska pozice labelu, jehož hodnota je ve AFFieldInfo rovněž zaznamenána. V neposlední řadě jsou v tomto objektu uloženy informace o validační pravidlech, oproti kterým se má validovat uživatelský vstup. Navíc ještě v případě, že by uživatel měl mít na výběr pouze z určitých předem definovaných možností, zahrnuje AFFieldInfo i informace o těchto možnostech. 

V tomto objektu je také uloženo, zda se jedná o vnitřní třídu, kterou jsme zmiňovali výše. Tento fakt je velmi důležitý, neboť záleží na pořadí polí v komponentě, ve kterém mají být vykreslovány. Inspekce modelu na serveru s tím počítá, a tak pole umístí na správné místo v metadatech a označí ho jako classType, tedy vnitřní třídu, jejíž popis můžeme nalézt v metadatech v části s vnitřními třídami. V rámci zachování správného pořadí vykreslení polí je tedy nutné, aby klienstký framework fakt, že se jedná o složený datový typ, při vytváření polí komponenty zaznamenal a na pozici, kde tuto skutečnost objeví, vložil pole, o nichž jsou informace uloženy v příslušné vnitřní třídě.

\subsubsection{AFValidationRule}
Tento objekt popisuje pravidlo, které by měl splňovat uživatelský vstup ve vytvářeném poli. Obsahuje typ validace, který určuje o jakou validaci se jedná a případně hodnotu pravidla. Referenční framework AFSwinx obsahuje výčtový typ s názvy validací, které podporuje a které se mohou tedy v metadatech objevit. Například definuje validační pravidlo typu MAX a hodnotou je nějaké číslo. Tedy říká, že hodnota v poli nesmí přesáhnout číslo určené hodnotou pravidla. V obou frameworcích na mobilních platformách tedy bude nutné tyto validace umět zpracovat, přičemž typ zpracování bude na obou platformách trochu jiný. 

\subsubsection{AFOptions}
Pro určité typy widgetů, které mají být použity pro vytvoření polí, je nutné specifikovat možnosti, ze kterých si bude uživatel vybírat. Takovými widgety je například dropdown menu nebo skupina radio buttonů. Tento objekt popisuje tyto možnosti formou klíče a hodnoty. Klíč je hodnota, kterou by měl framework odesílat na server a hodnota by měla být zobrazována klientovi.

\subsubsection{TopLevelLayout}
Objekt by měl být využit k popisu rozložení celé komponenty. Objekt definuje dvě vlastnosti. Za prvé je to orientace, tedy ve směru jaké osy bude komponenta či její část vykreslována. Dále je to pak definice rozložení, která má určovat, jestli bude komponenta či její části vykreslovány v jednom či více sloupcích. 

\subsubsection{Layout}
Popisuje rozložení částí komponenty, tedy vytvářených polí. Jak je vidět na obrázku \ref{metadataModel}, dědí z TopLevelLayoutu orientaci a definici rozložení, které jsou popsány výše. Navíc má vlastnost LabelPosition, která by měla být tvořenými frameworky využita k umístění labelu vzhledem k vytvářenému poli.

\subsection{Tvorba komponent}
Z přijatých a uložených metadat budou frameworky umět vytvořit prozatím dva typy komponent. Jednou komponentou je formulář, který bude hlavně řešit uživatelský vstup, ale může být využit, v případě, že bude předvyplněn daty, i ke zobrazení dat uživateli. Druhou komponentou je list, neboli seznam položek, který bude uživateli umožňovat zobrazení většího množství informací. Ve frameworku AFSwinx tuto možnost zajišťovala tabulka \cite{tomasek-thesis}, která však není pro mobilní zařízení úplně vhodným způsobem, protože vyžaduje pro rozumné zobrazení mnoho místa, které na mobilních zařízeních většinou není k dispozici.
Struktura metadat, získaných ze serveru, by se dala využít zároveň pro formulář i list. Lišit se bude pouze grafická reprezentace metadat. Zatímco ve formuláři lze využít definic proměnných, nad kterými byla provedena inspekce, k tvorbě formulářových polí, v listu je lze použít k tvorbě informací o jedné z jeho položek. Rozložení, které je definováno ve výše popsaném TopLevelLayoutu, zase lze použít v případě formuláře k uspořádání polí, tedy jestli budou pod sebou či vedle sebe nebo v jednom či dvou sloupcích. Stejně to lze udělat v listu s uspořádáním informací o položce. Některé informace sice přijdou v listu nazmar, jako například typ widgetu nebo validace, ale pořád je to výhodnější, než aby obě komponenty měly vlastní strukturu metadat. 

Pro rozlišení, zda se z metadat vytvoří formulář nebo list, bude potřeba vytvořit dva typy builderů, které komponenty postaví. Uživateli, který bude framework používat, by tedy mělo stačit specifikovat builder, který požadovanou komponentu bude tvořit. Aby bylo používání frameworků co nejvíce uživatelsky přívětivé, způsob tvoření builderů a jejich používání by se neměly lišit. Uživateli pak bude stačit naučit se pouze vytvořit jeden typ komponenty a druhý typ vytvoří pouze výměnou builderu. 

Pro zobrazení procesu tvorby formuláře, který je tou složitější komponentou, jelikož uživatelský vstup, který do něho bude zadáván se musí validovat a odesílat na server, narozdíl od listu, který je pouze pro čtení, byl využit diagram aktivit. Diagramy aktivit popsují určitý proces složený z dílčích podprocesů a můžou být použity například právě pro analýzu a popis algoritmu. 

V diagramu \ref{img:createFormActivityDiagram} lze vidět, že pokud bude chtít uživatel frameworku vytvořit formulář, bude muset nejdříve specifikovat, kde framework najde metadata. Ten pak o tato data požádá server. Server musí samozřejmě data dynamicky vytvořit, a tak provede inspekci, o které jsme již psal výše a vytvoří metadata, která pak klientskému frameworku předá zpět. Ten je určitým způsobem zpracuje a postaví na základě nich požadovanou komponentu. Pokud uživatel zároveň nadefinoval i zdroj dat, kterými by se měl formulář naplnit, požádá klient znovu server o tato data. Server je vygeneruje a opět zašle zpět, načež se komponenta těmito daty naplní. Poté se takto vytvořená komponenta předá uživateli, tedy vývojáři, který s ní může dále pracovat, například ji vložit do GUI tam, kam potřebuje.

V metadatech se nachází pro každé pole, které má být součástí komponenty, typ widgetu, kterým má být pole reprezentováno. Například to může být textové pole, checkbox nebo skupina radio buttonů. Ke každému widgetu bude potřeba vytvořit vlastní builder, který bude pole vytvářet a také určovat, jak se z widgetu dají získat data a naopak, jak je do něj vložit. 

\subsection{Lokalizace}
V rámci metadat, která přijdou ze serveru se mohou objevit texty určené pro lokalizaci, tedy překlad. Tyto texty jsou realizované pomocí klíčů, ke kterým lze překlad přiřadit a vyskytují se v labelech, v možnostech, ze kterých může uživatel v daných typech widgetů vybírat nebo ve validačních chybách. Frameworky tedy musí disponovat funkcionalitou, který tuto lokalizaci umožní včetně změny jazyka za běhu aplikace. V Androidu se typicky tyto lokaklizační texty umisťují do souboru strings.xml ve složce values. Tuto vlastnost je požadováno zanechat, a tak i musí být Android verze frameworku navržena. Pro Windows phone je to obdobné. Zde se texty vkládají do souborů Resources.resw, které jsou většinou umístěny ve složce Strings. Aby byl způsob lokalizace textů v aplikaci jednotný, bude možné využít lokalizační část frameworku i nad jeho rámec k překladu například textů tlačítek či položek v menu.

\subsection{Práce s vytvořenou komponentou}
Komponentu nestačí jen vytvořit, ale cílem frameworku je také umožnit s ní další práci. Vývojáři by mělo být umožněno například odeslat formulář, zkontrolovat to, co uživatel zadal nebo upravit její vzhled. Návrh možností, které by měly oba mobilní frameworky podporovat, je vidět v diagramu případu užití, který lze nálezt v příloze. Případy užití určují, jak lze používat systém nebo jeho dílčí části \cite{UmlArlow}. Případ užití je iniciován aktérem, jímž je v tomto případě hlavně vývojář, který bude framework používat. V této práci definuji některé případy užití i pro samotný framework, neboť chciv zobrazit, jaké akce musí framework provést pro splnění úkolu, zadaným uživatelem. Definici posloupností jednotlivých akcí popisuje vždy příslušný scénář případu užití. Pro práci s formulářem byl ještě vytvořen diagram aktivit, který popisuje tři hlavní možnosti práce s formulářem a to odesílání dat, resetování a vyčištění formuláře. Tento diagram lze nalézt v příloze \ref{img:formWorkActivityDiagram}.

Nyní ale zpět k případům užití. Jedním z hlavních případů užití je odeslání formuláře. Tuto část use case digramu systému, kterou lze vidět na následujícím obrázku, podrobněji popíšu. Pro stručnost předpokládám hladký průběh scénáře, samozřejmě všude tam, kde se vyskytne IF, tedy podmínka, by měl být uveden i alternativní scénář, který řeší to, co se stane při nesplnění podmínky.
\begin{figure}[h!]
\includegraphics[width=\textwidth]{figures/useCaseFormSend}
\caption{Diagram případů užití pro odeslání formuláře}
\label{img:useCaseModelFormSend}
\end{figure}

\subsubsection{Případ užití: Validace formuláře}
Na obrázku \ref{img:useCaseModelFormSend} se tento případ jmenuje Validovat formulář a má zachycovat validaci uživatelského vstupu ve formuláři. V diagramu jde také vidět, že by uživatel měl být schopen užívat validaci formuláře i explcitně, tedy ne jen v rámci odeslání dat. Následující scénář tohoto případu užití popisuje postup, jak by měla validace formuláře probíhat.\\\\
Vstupní podmínky:
\begin{itemize}
\item Framework zná formulář, který chce validovat. 
\item Tento formulář musí být vytovřený za pomoci metadat.
\end{itemize}
Scénář případu užití:
\begin{enumerate}
\item Uživatel požádá systém o validaci daného formuláře.
\item Systém získá z formuláře všechna pole
\item WHILE existuje nezvalidované pole DO
\begin{enumerate}
\item Systém získá všechna pravidla, kterým pole podléhá
\item WHILE existuje nenavštívené pravidlo DO
\begin{enumerate}
\item Systém získá pro dané pole a pravidlo příslušný validátor.
\item Systém provede validaci.
\item IF validace selhala THEN
\begin{enumerate}
\item Systém přidá k chybovým hláškám určených pro zobrazení příslušnou hlášku o chybě validace.
\end{enumerate}
\end{enumerate}
\item IF alespoň jedna z validací na poli selhala THEN
\begin{enumerate}
\item Systém zobrazí k danému poli validační chyby.
\end{enumerate}
\end{enumerate} 
\end{enumerate}

\subsubsection{Případ užití: Vygenerování odesílaných dat}
Tento případ užití popisuje akci frameworku, která musí být provedena, aby šlo úspěšně na server odeslat data. Framework nebude znát přímo objekt, který chce na serveru odesláním dat vytvořit nebo upravit. Co však zná, je jeho struktura, na základě které sice objekt nevytvoří, ale je schopen poskládat pro server přijatelná data, ze kterých si server, který objekt již zná, dokáže tento objekt vytvořit. Důležitý je formát dat, ve kterém mají být serveru tyto data zaslány. Referenční AFSwinx podporuje prozatím jen JSON formát, ale počítá se s přidáním dalších formátů \cite{tomasek-thesis}. Bylo by tedy dobré, navrhnout tento případ užití pro více možných formátů. V diagramu \ref{img:useCaseModelFormSend} je tento use case nazván Vytvoření dat z formuláře pro odeslání a popisuje ho níže zmíněný scénář.\\\\
Vstupní podmínky:
\begin{itemize}
\item Framework zná formulář, ze kterého chce sestavovat data pro odeslání. 
\item Tento formulář musí být vytovřený za pomoci metadat. 
\item Framework musí znát serverem akceptovaný formát dat. 
\end{itemize}
Scénář případu užití:
\begin{enumerate}
\item Uživatel chce poskládat z formuláře data k odeslání.
\item <<include>> Validovat formulář
\item IF validace proběhla úspěšně THEN
\begin{enumerate}
\item Systém získá z formuláře všechna pole
\item WHILE existuje pole, které nebylo zahrnuto v odesílaných datech DO
\begin{enumerate}
\item Systém získá builder, který pole postavil.
\item Systém požádá builder o data, která se v poli nachází.
\item  Systém určí název proměnné a třídu, do které patří, a nastaví jí data.
\item  Systém podle formátu dat, které server očekává, rozhodne, v jakém formátu data zaslat a na tento formát data převede. 
\end{enumerate}
\end{enumerate}
\end{enumerate}

\subsubsection{Případ užití: Odeslání dat na server}
Posledním případem užití z obrázku \ref{img:useCaseModelFormSend} je use case Odeslat data na server. Ten zahrnuje předchozí případ, což jde ostatně vidět v níže uvedeném scénáři.\\\\
Vstupní podmínky:
\begin{itemize}
\item Framework zná formulář, ze kterého chce sestavovat data pro odeslání.
\item Tento formulář musí být vytovřený za pomoci metadat.
\item Framework musí znát zdroj, na který mají být data odeslána a všechny potřebné informace, které zdroj vyžaduje. 
\end{itemize}
Scénář případu užití:
\begin{enumerate}
\item Uživatel chce odeslat data z formuláře na server.
\item <<include>> Vytvoření dat z formuláře pro odeslání
\item IF bylo vytvoření dat z formuláře úspěšné THEN
\begin{enumerate}
\item Systém odešle data na specifikovaný zdroj
\item Systém informuje uživatele o výsledku akce
\end{enumerate}
\end{enumerate}

\subsubsection{Úprava vzhledu komponenty}
Vzhled je nedílnou součástí každé aplikace s grafickým uživatelským rozhraním. V Android aplikacích se vzhled definuje buď v XML šablonách pro daný view, to v případě, že je GUI vytvořeno staticky \cite{android-themes}, nebo pomocí Javy v případě, že jej vytváříme dynamicky. Na Windows Phone platformě jsou obdobou xml šablon xaml soubory a případě dynamického vytváření, slouží k úpravě metody napsané v C\#. Ve Windows Phone aplikacích se však preferuje neměnit barvy komponent. Ke změně barev slouží barevná témata, která mají výhodu v tom, že jsou konzistentní na všech zařízeních \cite{wp-themes}.

Navíc Windows phone umožňuje změnit barevné téma zařízení v nastavení. Toto téma neovlivní jen GUI operačního systému, ale také aplikace. Může se tedy stát, že pokud vývojář nastaví například barvu textu na bílou a aktuální téma, které zařízení používá, má bílé pozadí, nepůjdou texty jednoduše vidět. Windows Phone verze frameworku bude tedy muset zohlednit i tyto situace.  

Jelikož se komponenty tvoří dynamicky v rámci frameworku a uživatel k procesu tvorby nemá explicitně přístup, je nutné poskytnout mu možnost nadefinovat vzhled komponenty předem. K tomuto účelu by měly frameworky disponovat skiny, které bude mít uživatel možnost nastavit builderu komponenty. V příloze \ref{img:useCaseModel} se tento případ užití nazývá Nastavit skin. V případě, že by uživatel frameworku skin nenadefinoval, měl by existovat základní vzhled. Od tohoto základního skinu může uživatel ve svých skinech dědit a překrýt pouze části, které mu nevyhovují a chtěl by je jinak. Po sestavení komponenty už není jednoduše možné za pomocí frameworku tento vzhled změnit. Framework sice bude poskytovat možnost získat reprezentaci komponent i jejich částí, ale použití již bude komplikovanější a bude vyžadovat znalost dynamické tvorby UI na jednotlivých platformách.

\subsection{Práce na existujícím řešení}
Cílem práce je také přidat nějakou hodnotu do stávající verze frameworku AFSwinx. Pro tento účel bylo rozhodnuto přidat novou anotaci, kterou zohlední dříve zmiňovaná inspekce na serveru při tvorbě metadat. Cílem anotace bude sdělit klientovi, že se má anotací označený atribut v modelu na serveru porovnat druhým atributem, který bude určen v parametrech anotace, ve smyslu menší nebo rovno než. Anotace se bude týkat hlavně datumů a bude se tedy řešit, zda anotací označený atribut typu Datum obsahuje datum dřívejší nebo stejné než druhý atribut určený v parametrech anotace. Tato informace se v metadatech objeví prozatím pouze u informací o poli s typem widgetu CALENDAR, tedy u těch, jež mají být reprezentovány jako DatePicker. Kontrétně bude informace přidána do sekce rules, bude se tedy jednat o validační pravidlo. AspectFaces \cite{aspect-faces} ve své dokumentaci popisuje, že pokud chce uživatel přidat novou anotaci, musí vytvořit tzv. anotační deskriptor, který pak musí zaregistrovat v konfiguračním souboru aspectfaces-config.xml uloženém ve složce WEB-INF. Tím se zajistí, že si inspekce anotace všimne a promítne ji do metadat ve formě XML. AFSwinx tyto XML soubory ještě převádí do jiné platformově nezávislé podoby \cite{tomasek-thesis} a v tomto procesu tedy bude nutné taktéž provést změny. Po vytvoření anotace je nutné přidat ji na daná místa na serveru a je žádoucí vytvořit validátory nejen v obou vytvářených frameworcích pro mobilní platformy, ale také v již existujícím řešení AFSwinx pro platformu Java SE.



