\chapter{Závěr}
\section{Budoucí vývoj}
Jelikož uživatelský test měl přívětivé výsledky a všichni testovaní projevili o frameworky zájem, je v plánu oba frameworky nadále vyvíjet. Dalším krokem bude rozšířit řešení i na iOS platformu. Windows Phone verze frameworku běží na platformě C\#, na které běží i Windows desktopové aplikace a proto bude možné pouze s drobnými úpravami rozšířit framework i tímto směrem. Frameworky nyní obsahují množinu validačních pravidel, kterou bychom rádi také rozšířili o další pravidla. Také plánujeme přidat další typy vstupních polí jako například widget pro nahrání fotografie nebo posuvníky. V Android frameworku existuje navíc návrh další komponenty - tabulky, která nyní není plně funkční, neboť bylo rozhodnuto nahradit ji listem. Tabulky bychom ale rádi dotvořili, neboť může být vhodnou komponentou pro zobrazování dat například na tabletech. Také bychom rádi přidali více možností pro úpravu vzhledu komponent, neboť je vzhled velmi důležitou součástí UI. Mobilní zařízení o sobě také ví spoustu užitečných informací jako je například aktuální orientace displeje nebo stav baterie, kterých bychom rádi také při generování a interpretaci UI využili.

Budoucí vývoj bude průběžně testován s uživateli, protože jedním z hlavních cílů frameworku je i jeho použitelnost a hodnocení uživatelů nám výrazně pomůže upravit způsob fungování a využití frameworku.

Ve finále by měly všechny frameworky pro mobilní platformy být plně nasaditelné a připravené k vydání v rámci AspectFaces \cite{aspect-faces}. Mimo mobilních platforem se pod AspectFaces těchto klientských frameworků plánuje více, již existuje interpret pro Swing aplikace na Java SE platformě a je také ve vývoji verze využívající framework AngularJS. Všechny klientské frameworky využívají stejných definic, které přijímají ze serveru, kde lze nadefinovat UI pouze jednou a na jednom místě a klienstké frameworky budou případné změny automaticky reflektovat. To umožní výrazně snížit náklady na vývoj a údržbu klientských aplikací, neboť bude potřeba méně kódu a v případě změny méně úprav.

\section{Zhodnocení práce}
Cílem této práce bylo navrhnout a naimplementovat klienstké frameworky pro dvě mobilní platformy, které získávají a interpretují definice UI ze serveru. Při vytváření frameworků bylo nutné se adaptovat na strukturu těchto definic, které vytváří framework AFRest. Na základě tohoto popisu bylo potřeba vytvořit konkrétní komponenty, případně je naplnit daty a umožnit uživateli s nimi dále pracovat. Při vytváření komponent frameworky zohledňují rozložení komponent, přidružená validační pravidla, bezpečnost, různé typy aktivních prvků, kterými mají být části komponenty reprezentovány a umožňují uživateli nastavit vzhled komponenty. Důležitým požadavkem také byla lokalizace textů, které ze serveru přicázely ve formě klíčů určených k přeložení a v zájmu jednotného použití, umožnit uživateli tuto část framewoku použít i pro překlad svých textů. 

Všechny tyto požadavky byly splněny. Frameworky umí na základě definic sestavit formulář či list, kterým umí správně nastavit rozložení, reprezentovat jejich části správným grafickým prvkem a umí je korektně naplnit daty. Frameworky podporují rozložení částí komponenty do jednoho či dvou sloupců a prvky mohou být vykreslovány ve vertikálním i horizontálním směru. Co se widgetů týče, jsou frameworky schopné vytvořit nejčastěji používané aktivní prvky jako například pole pro text, heslo, číslo nebo třeba datepicker a při vytváření jsou zohledňovány i vlastnosti jako viditelnost nebo zda má být prvek jen pro čtení. Při odeslání formuláře na server je framework schopen ze znalosti popisu modelu, ze kterého je komponenta vytvořena, sestavit data k odeslání ve správné formě, kterou server bez problému přijme. Formulář dále disponuje dalšími funkcemi jako je resetování dat, vyčištění formulářea a jeho validace. Každé pole ve formuláři má přidružená validační pravidla, která jsou kontrolována za pomocí příslušných validátorů. List, který je určen k zobrazování většího množství dat, disponuje možností získat data o jednom jeho prvku a vložit je do formuláře k případné úpravě. Z hlediska bezpečnosti podporují frameworky basic autentizaci. Lokalizační část umožňuje překlad textů přicházejících ze serveru i textů vyskytujících se mimo proces tvoření komponent, tedy například texty tlačítek, které uživatel vytvořil. Lokalizační část dále umožňuje změnit jazyk za běhu aplikace. Frameworky také disponují v rámci skinů značným množstvím funkcí pro úpravu vzhledu komponent.

Z hlediska použitelnosti tvorba a práce s komponentami vyžadují menší množství kódu a jsou relativně jednoduché na naučení, což potvrdil i uživatelský test. Frameworky díky neustálému stahování aktuálních definic ze serveru umí pružně reagovat na změnu v datém modelu na serveru, což je nesporně výhodou, neboť není nutné vydávat při změně modelu novou verzi aplikace. Nevýhodou je, že musí být klientské aplikace připojeny k internetu, což ale v dnešní době není tak velkým problémem. 

Oba frameworky byly vytvořeny a otestovány unit testy, vytvořením ukázkových aplikací a také uživateli, kteří měli na oba frameworky z hlediska použitelnosti velmi pozitivní názor.