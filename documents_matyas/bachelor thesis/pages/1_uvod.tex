\chapter{Úvod}
Tato bakalářská práce se zabývá analýzou, návrhem a otestováním frameworků pro mobilní platformy Android a Windows Phone interpretující uživatelské rozhraní z dat generovaných serverem z modelu, která poskytuje pomocí webových služeb.

První část práce popisuje aktuální situaci v tvorbě UI, specifikuje požadavky a cíle práce a zkoumá již existující řešení. V druhé části se pak analyzují požadavky, které by měla práce splňovat a návrh řešení, které stanovené požadavky a cíle splňuje. Ve třetí části je popsána struktura a vlastní implementace řešení. Poslední část pak obsahuje otestování řešení a ukazuje vzorovou aplikaci na dvou různých mobilních prostředích.

Práce obsahuje seznam použitých zkratek, které lze najít v příloze A, instalační příručka v příloze B, použité UML diagramy a obrázky viz. příloha C a obsah CD s prací a zdrojovými kódy je v příloze D.
 

\section{Motivace}
Nedílnou součástí většíny dnešních aplikací je uživatelské rozhraní. Uživatelské rozhraní by mělo uživateli co nejvíce usnadňovat manipulaci se softwarem a tudíž být intuitivní a použitelné, nemluvě o tom, že by mělo pěkně vypadat. Vývoj takového rozhraní je však časově velmi náročný proces, který zahrnuje nejenom samotný vývoj, ale také rozsáhlé testování, hlavně z hlediska funkčnosti a použitelnosti. Celý problém navíc umocňuje fakt, že se vývojáři snaží zajistit podporu software na více platformách, neboť chtějí uživatelům nabídnout možnost operovat s jejich vytvořeným systémem nejen z počítače či laptopu, ale také z tabletu nebo mobilního zařízení. Mobilní verze grafického uživatelského rozhraní nebývá často moc rozdílná od rozhraní ostatních platforem v tom smyslu, že se v ní vyskytují vesměs stejné grafické prvky, a tak pro mobilní verzi vzniká téměř indentická kopie tohoto rozhraní, čímž vzniká duplicita. Problémem je, že se často uživatelská rozhraní mění, ať už se změna týká rozložení komponent, přidání nebo odebrání komponenty nebo třeba validace uživatelského vstupu, protože takováto změna se pak musí provést na všech platformách a někdy i na více místech v rámci aplikace, což může být v případě rozsáhlých systémů nejednoduchý úkol, který stojí vývojáře spoustu zbytečného času. Pokud bychom byli schopni nadefinovat uživatelské rozhraní jen jednou pro všechny platformy na jednom místě, tento problém bychom odstranili. Definice by tedy byla obecná, ale každá platforma je jiná a něčím specifická, proto je třeba vytvořit pro různé platformy frameworky, které obecnou definici pro danou platformu interpretují. 

Bylo mi nabítnuto vytvořit takovýto framework pro dvě mobilní platformy, kontrétně pro Android a Windows Phone, což mi přislo velmi užitečné a zajímavé a proto jsem se rozhodl zpracovat toto téma jako bakalářskou práci.
