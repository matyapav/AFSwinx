\chapter{Úvod}
Tato bakalářská práce se zabývá analýzou, návrhem a otestováním frameworků pro mobilní platformy Android a Windows Phone interpretující uživatelské rozhraní z dat generovaných serverem z modelu, která jsou poskytována serverem pomocí webových služeb.

První část práce popisuje aktuální situaci v tvorbě UI, specifikuje požadavky a cíle práce a zkoumá již existující řešení. V druhé části se analyzují požadavky, které by měla práce splňovat, a návrh řešení, které stanovené požadavky a cíle splňuje. Ve třetí části je popsána struktura a vlastní implementace řešení. Poslední část obsahuje otestování řešení a ukazuje vzorové aplikace na dvou různých mobilních prostředích.

Práce obsahuje seznam použitých zkratek, které lze nalézt v příloze A, instalační příručku v příloze B, použité UML diagramy a obrázky viz příloha C, ukázky zdrojových kódů v příloze D a obsah CD s prací a zdrojovými kódy je v příloze E.
 
\section{Motivace}
Nedílnou součástí většiny dnešních aplikací je uživatelské rozhraní. Uživatelské rozhraní by mělo uživateli co nejvíce usnadňovat manipulaci se softwarem a tudíž být intuitivní a použitelné, mělo by mít adekvátní design, nejlépe korespondující s aktuálními trendy. Vývoj takového rozhraní je však časově velmi náročný proces, který zahrnuje nejenom samotný vývoj, ale také rozsáhlé testování, hlavně z hlediska funkčnosti a použitelnosti. Celý problém navíc umocňuje fakt, že se vývojáři snaží zajistit podporu softwaru na více platformách, neboť chtějí uživatelům nabídnout možnost operovat s jejich vytvořeným systémem nejen z počítače či laptopu, ale také z tabletu nebo mobilního zařízení. Mobilní verze grafického uživatelského rozhraní nebývá často moc rozdílná od rozhraní ostatních platforem v tom smyslu, že se v ní vyskytují převážně stejné grafické prvky, a tak pro mobilní verzi vzniká téměř indentická kopie tohoto rozhraní, což vede i k duplicitě v kódu aplikace. Problémem je, že se často uživatelská rozhraní mění, ať už se změna týká rozložení komponent, přidání nebo odebrání komponenty nebo třeba validace uživatelského vstupu, protože takováto změna se musí provést na všech platformách a někdy i na více místech v rámci aplikace, což může být v případě rozsáhlých systémů nejednoduchý úkol, který stojí vývojáře spoustu zbytečného času a může vést i ke vzniku nových typů chyb \cite{towards-smart-design}. Pokud bychom byli schopni nadefinovat uživatelské rozhraní jen jednou pro všechny platformy na jednom místě, tento problém bychom odstranili. Definice by tedy byla obecná, ale každá platforma je jiná a něčím specifická, proto je třeba vytvořit pro různé platformy frameworky, které obecnou definici pro danou platformu interpretují. 

Bylo mi nabítnuto vytvořit takovýto framework pro dvě mobilní platformy, kontrétně pro Android a Windows Phone, což mi přislo velmi užitečné a zajímavé, a proto jsem se rozhodl zpracovat toto téma jako bakalářskou práci.
