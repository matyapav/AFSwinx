\documentclass[11pt,twoside,a4paper]{book}  
% definice dokumentu
\usepackage[czech, english]{babel}
\usepackage[T1]{fontenc} 				% pouzije EC fonty 
\usepackage[utf8]{inputenc} 			% utf8 kódování vstupu 
\usepackage[square, numbers]{natbib}	% sazba pouzite literatury
\usepackage{indentfirst} 				% 1. odstavec jako v cestine, pro práci v aj možno zakomentovat
\usepackage{fancyhdr}					% tisk hlaviček a patiček stránek
\usepackage{nomencl} 					% umožňuje snadno definovat zkratky a jejich seznam


%%%%%%%%%%%%%%%%%%%%%%%%%%%%%%%%%%%%%%%%%%%%%%%%%%%%%%%%%%%%%%%
% informace o práci
\newcommand\WorkTitle{Servisně orientovaný aspektový vývoj uživatelských rozhraní pro mobilní aplikace}		% název
\newcommand\FirstandFamilyName{Pavel Matyáš}															% autor
\newcommand\Supervisor{Ing. Martin Tomášek}															% vedoucí

\newcommand\TypeOfWork{Bakalářská práce}	% typ práce [Diplomová práce | Bakalářská práce | Bachelor's Project | Master's Thesis ]	

% Nastavte následují podle vašeho oboru a programu (pomoc hledejte na http://www.fel.cvut.cz/cz/education/bk/prehled.html)								
\newcommand\StudProgram{Otevřená informatika, Bakalářský}	% program
\newcommand\StudBranch{Softwarové inženýrství}           					% obor

%%%%%%%%%%%%%%%%%%%%%%%%%%%%%%%%%%%%%%%%%%%%%%%%%%%%%%%%%%%%%%%
% minimální importy
\usepackage{graphicx}					% pro vkládání obrázků
\usepackage{k336_thesis_macros} 		% specialni makra pro formatovani DP a BP
\usepackage[
pdftitle={\WorkTitle},				% nastaví v informacích o pdf název
pdfauthor={\FirstandFamilyName},	% nastaví v informacích o pdf autora
colorlinks=true,					% před tiskem doporučujeme nastavit na false, aby odkazy a url nebyly šedé při ČB tisku
breaklinks=true,
urlcolor=red,
citecolor=blue,
linkcolor=blue,
unicode=true,
]
{hyperref}								% pro zobrazování "prokliknutelných" linků 

% rozšiřující importy
\usepackage{listings} 			%slouží pro tisk zdrojových kódů se syntax higlighting
\usepackage{algorithmicx} 		%slouží pro zápis algoritmů
\usepackage{algpseudocode} 		%slouží pro výpis pseudokódu

%%%%%%%%%%%%%%%%%%%%%%%%%%%%%%%%%%%%%%%%%%%%%%%%%%%%%%%%%%%%%%%
% příkazy šablony
\makenomenclature								% při překladu zajistí vytvoření pracovního souboru se seznamem zkratek

\let\oldUrl\url									% url adresy budou zobrazeny: <url> 
\renewcommand\url[1]{<\texttt{\oldUrl{#1}}>}

%%%%%%%%%%%%%%%%%%%%%%%%%%%%%%%%%%%%%%%%%%%%%%%%%%%%%%%%%%%%%%%
% vaše vlastní příkazy
\newcommand*{\nomExpl}[2]{#2 (#1)\nomenclature{#1}{#2}} 	% usnadňuje zápis zkratek : Slova ke Zkrácení (SZ)
\newcommand*{\nom}[2]{#1\nomenclature{#1}{#2}} 			% usnadňuje zápis zkratek : SZ



%%%%%%%%%%%%%%%%%%%%%%%%%%%%%%%%%%%%%%%%%%%%%%%%%%%%%%%%%%%%%%%
% vlastní dokument
%%%%%%%%%%%%%%%%%%%%%%%%%%%%%%%%%%%%%%%%%%%%%%%%%%%%%%%%%%%%%%%
\begin{document}
	
	%%%%%%%%%%%%%%%%%%%%%%%%%% 
	% nastavení jazyka, kterým je práce psána
	\selectlanguage{czech}	% podle jazyka práce nastavte na [czech | english]
	\translate				% nastaví české nebo anglické popisy (např. katedra -> department); viz k336_thesis_macros

	%%%%%%%%%%%%%%%%%%%%%%%%%%    
	% Poznamky ke kompletaci prace
	% Nasledujici pasaz uzavrenou v {} ve sve praci samozrejme 
	% zakomentujte nebo odstrante. 
	% Ve vysledne svazane praci bude nahrazena skutecnym 
	% oficialnim zadanim vasi prace.
	{
	\pagenumbering{roman} \cleardoublepage \thispagestyle{empty}
	\chapter*{Na tomto místě bude oficiální zadání vaší práce}
	\begin{itemize}
		\item Toto zadání je podepsané děkanem a vedoucím katedry,
		\item musíte si ho vyzvednout na studijním oddělení Katedry počítačů na Karlově náměstí,
		\item v jedné odevzdané práci bude originál tohoto zadání (originál zůstává po obhajobě na katedře),
		\item ve druhé bude na stejném místě neověřená kopie tohoto dokumentu (tato se vám vrátí po obhajobě).
	\end{itemize}
	\newpage
	}

	%%%%%%%%%%%%%%%%%%%%%%%%%%    
	% Titulni stranka / Title page 
	\coverpagestarts

	%%%%%%%%%%%%%%%%%%%%%%%%%%%    
	% Poděkovani / Acknowledgements 

	\acknowledgements
	\noindent
	Zde můžete napsat své poděkování, pokud chcete a máte komu děkovat.


	%%%%%%%%%%%%%%%%%%%%%%%%%%%   
	% Prohlášení / Declaration 

	\declaration{V~Kořenovicích nad Bečvárkou dne 15.\,5.\,2008}
	%\declaration{In Kořenovice nad Bečvárkou on May 15, 2008}


	%%%%%%%%%%%%%%%%%%%%%%%%%%%%    
	% Abstrakt / Abstract 
 
	\abstractpage

	Translation of Czech abstract into English.

	% Prace v cestine musi krome abstraktu v anglictine obsahovat i
	% abstrakt v cestine.
	\vglue60mm

	\noindent{\Huge \textbf{Abstrakt}}
	\vskip 2.75\baselineskip

	\noindent
	Abstrakt práce by měl velmi stručně vystihovat její obsah. Tedy čím se práce zabývá a co je jejím výsledkem/přínosem.

	\noindent
	Očekávají se cca 1 -- 2 odstavce, maximálně půl stránky.

	%%%%%%%%%%%%%%%%%%%%%%%%%%    
	% obsahy a seznamy
	\tableofcontents		% Obsah / Table of Contents 

	% pokud v práci nejsou obrázky nebo tabulky - odstraňte jejich seznam
	\listoffigures			% Obsah / Table of Contents 
	\listoftables			% Seznam tabulek / List of Tables

	%%%%%%%%%%%%%%%%%%%%%%%%%% 
	% začátek textu  
	\mainbodystarts
%**************************************************************

% Pro snadnejsi praci s vetsimi texty je rozumne tyto rozdelit
% do samostatnych souboru nejlepe dle kapitol a tyto potom vkladat
% pomoci prikazu \include{jmeno_souboru.tex} nebo \include{jmeno_souboru}.
% Napr.:
% \chapter{Úvod}
Tato bakalářská práce se zabývá analýzou, návrhem a otestováním frameworků pro mobilní platformy Android a Windows Phone interpretující uživatelské rozhraní z dat generovaných serverem z modelu, která jsou poskytována serverem pomocí webových služeb.

První část práce popisuje aktuální situaci v tvorbě UI, specifikuje požadavky a cíle práce a zkoumá již existující řešení. V druhé části se analyzují požadavky, které by měla práce splňovat, a návrh řešení, které stanovené požadavky a cíle splňuje. Ve třetí části je popsána struktura a vlastní implementace řešení. Poslední část obsahuje otestování řešení a ukazuje vzorové aplikace na dvou různých mobilních prostředích.

Práce obsahuje seznam použitých zkratek, které lze nalézt v příloze A, instalační příručku v příloze B, použité UML diagramy a obrázky viz příloha C, ukázky zdrojových kódů v příloze D a obsah CD s prací a zdrojovými kódy je v příloze E.
 
\section{Motivace}
Nedílnou součástí většiny dnešních aplikací je uživatelské rozhraní. Uživatelské rozhraní by mělo uživateli co nejvíce usnadňovat manipulaci se softwarem a tudíž být intuitivní a použitelné, mělo by mít adekvátní design, nejlépe korespondující s aktuálními trendy. Vývoj takového rozhraní je však časově velmi náročný proces, který zahrnuje nejenom samotný vývoj, ale také rozsáhlé testování, hlavně z hlediska funkčnosti a použitelnosti. Celý problém navíc umocňuje fakt, že se vývojáři snaží zajistit podporu softwaru na více platformách, neboť chtějí uživatelům nabídnout možnost operovat s jejich vytvořeným systémem nejen z počítače či laptopu, ale také z tabletu nebo mobilního zařízení. Mobilní verze grafického uživatelského rozhraní nebývá často moc rozdílná od rozhraní ostatních platforem v tom smyslu, že se v ní vyskytují převážně stejné grafické prvky, a tak pro mobilní verzi vzniká téměř indentická kopie tohoto rozhraní, což vede i k duplicitě v kódu aplikace. Problémem je, že se často uživatelská rozhraní mění, ať už se změna týká rozložení komponent, přidání nebo odebrání komponenty nebo třeba validace uživatelského vstupu, protože takováto změna se musí provést na všech platformách a někdy i na více místech v rámci aplikace, což může být v případě rozsáhlých systémů nejednoduchý úkol, který stojí vývojáře spoustu zbytečného času a může vést i ke vzniku nových typů chyb \cite{towards-smart-design}. Pokud bychom byli schopni nadefinovat uživatelské rozhraní jen jednou pro všechny platformy na jednom místě, tento problém bychom odstranili. Definice by tedy byla obecná, ale každá platforma je jiná a něčím specifická, proto je třeba vytvořit pro různé platformy frameworky, které obecnou definici pro danou platformu interpretují. 

Bylo mi nabítnuto vytvořit takovýto framework pro dvě mobilní platformy, kontrétně pro Android a Windows Phone, což mi přislo velmi užitečné a zajímavé, a proto jsem se rozhodl zpracovat toto téma jako bakalářskou práci.

% \include{2_teorie}
% atd...

%*****************************************************************************
\chapter{Úvod}
Tato bakalářská práce se zabývá analýzou, návrhem a otestováním frameworků pro mobilní platformy Android a Windows Phone interpretující uživatelské rozhraní z dat generovaných serverem z modelu, která jsou poskytována serverem pomocí webových služeb.

První část práce popisuje aktuální situaci v tvorbě UI, specifikuje požadavky a cíle práce a zkoumá již existující řešení. V druhé části se analyzují požadavky, které by měla práce splňovat, a návrh řešení, které stanovené požadavky a cíle splňuje. Ve třetí části je popsána struktura a vlastní implementace řešení. Poslední část obsahuje otestování řešení a ukazuje vzorové aplikace na dvou různých mobilních prostředích.

Práce obsahuje seznam použitých zkratek, které lze nalézt v příloze A, instalační příručku v příloze B, použité UML diagramy a obrázky viz příloha C, ukázky zdrojových kódů v příloze D a obsah CD s prací a zdrojovými kódy je v příloze E.
 
\section{Motivace}
Nedílnou součástí většiny dnešních aplikací je uživatelské rozhraní. Uživatelské rozhraní by mělo uživateli co nejvíce usnadňovat manipulaci se softwarem a tudíž být intuitivní a použitelné, mělo by mít adekvátní design, nejlépe korespondující s aktuálními trendy. Vývoj takového rozhraní je však časově velmi náročný proces, který zahrnuje nejenom samotný vývoj, ale také rozsáhlé testování, hlavně z hlediska funkčnosti a použitelnosti. Celý problém navíc umocňuje fakt, že se vývojáři snaží zajistit podporu softwaru na více platformách, neboť chtějí uživatelům nabídnout možnost operovat s jejich vytvořeným systémem nejen z počítače či laptopu, ale také z tabletu nebo mobilního zařízení. Mobilní verze grafického uživatelského rozhraní nebývá často moc rozdílná od rozhraní ostatních platforem v tom smyslu, že se v ní vyskytují převážně stejné grafické prvky, a tak pro mobilní verzi vzniká téměř indentická kopie tohoto rozhraní, což vede i k duplicitě v kódu aplikace. Problémem je, že se často uživatelská rozhraní mění, ať už se změna týká rozložení komponent, přidání nebo odebrání komponenty nebo třeba validace uživatelského vstupu, protože takováto změna se musí provést na všech platformách a někdy i na více místech v rámci aplikace, což může být v případě rozsáhlých systémů nejednoduchý úkol, který stojí vývojáře spoustu zbytečného času a může vést i ke vzniku nových typů chyb \cite{towards-smart-design}. Pokud bychom byli schopni nadefinovat uživatelské rozhraní jen jednou pro všechny platformy na jednom místě, tento problém bychom odstranili. Definice by tedy byla obecná, ale každá platforma je jiná a něčím specifická, proto je třeba vytvořit pro různé platformy frameworky, které obecnou definici pro danou platformu interpretují. 

Bylo mi nabítnuto vytvořit takovýto framework pro dvě mobilní platformy, kontrétně pro Android a Windows Phone, což mi přislo velmi užitečné a zajímavé, a proto jsem se rozhodl zpracovat toto téma jako bakalářskou práci.

\chapter{Popis problému a specifikace cíle}
\section{Popis problematiky}

Softwarový systém má sloužit člověku k řešení nějakého problému. Uživatel přitom často problém blíže specifikuje a systém musí mít způsob, jak uživateli sdělit jeho řešení. K tomu slouží uživatelská rozhraní, která umožňují vzájemnou komunikaci systému s uživatelem. Uživatelské rozhraní přitom není jednoduchá záležitost. V první řadě by mělo být navrženo tak, aby sloužilo uživateli. Mělo by mu umožnit jednoduchou interakci se systémem, být intuitivní, funkční a hlavně použitelné. 


\subsection{Různá uživatelská rozhraní}
Aby bylo uživatelské rozhraní použitelné a uživatelsky přívětivé je třeba prozkoumat, jakým způsobem člověk s aplikacemi spolupracuje. Nejen tímto se zabývá disciplína zvaná Human Computer Interaction, která zkoumá potřeby uživatelů z různorodých hledisek. Díky této disciplíně vzniklo množství uživatelských rozhraní, pomocí kterých může člověk s počítačem komunikovat. Jedním z takových rozhraní je textové uživatelské rozhraní, značené CUI, jehož typickým zástupcem je příkazová řádka. Dalším typem je Hlasové uživatelské rozhraní, které dokáže interpretovat povely zadané lidskou řečí. Nejrozšířenějším je však grafické uživatelské rozhraní, zkráceně GUI, které využívá grafické prvky. GUI se stalo velmi oblíbeným právě proto, že je jednoduché a grafické prvky v člověku vyvolávají podobnost s vnějším světem. Také není nutné znát žádně specifické příkazy, jako v případě příkazové řádky, nebo hlasové povely jako v případě hlasového rozhraní. Zmínil bych ještě multimodální rozhraní, která používají k interakci více lidských smyslů a tak jsou vhodná například i pro lidi s postižením \cite{uiTypes}. 

V softwarových systémech je nejběžnějsím způsobem interakce uživatele se systémem právě grafické uživatelské rozhraní a tím se taky v této práci zabýváme. Návrh GUI je potřeba důkladně zvážit, neboť závisí na mnoha aspektech. Důležité je, pro jaké zařízení GUI tvoříme, jaký účel má aplikace, která bude rozhraním disponovat a také stav uživatele a prostředí, ve kterém se nachází. Typ zařízení je důležitý hlavně proto, protože každé zařízení používá jiné ovládací prvky. Zatímco u mobilního zařízení můžeme očekávat použití dotykového displeje, u počítače zase použití myši, klávesnice nebo dokonce jiných externích vstupních zařízení, jako může být například grafický tablet. Také je třeba vzít v potaz jinou velikost displeje a jiná rozlišení jednotlivých zařízení. Účel aplikace ovlivňuje GUI hlavně z hlediska obsahu, tedy jaké komponenty je nutné mít, aby byla aplikace využívána k daném účelu. Těžko by asi někdo chtěl emailového klienta bez možnosti odeslat zprávu. Stav uživatele a prostředí může zase ovlivnit způsob ovládání aplikace. Příkladem může být palubní počítač v automobilu, na kterém by měl uživatel být schopen přepnout rádiovou stanici, aniž by se přestal věnovat řízení. 

Co ale takové grafické rozhraní nejčastěji obsahuje? Běžně GUI disponuje ovládacími a vizuálními prvky pomocí kterých lze aplikaci ovládat. Osobně bych GUI prvky rozdělil na vstupní a výstupní. Vstupní prvky zachycují uživatelský vstup a akce, které systém ovládají a výstupní zobrazují uživateli výsledky těchto akcí, data a aktuální stav systému. Vstupními prvky jsou nejčastěji vstupní pole, do kterých může uživatel napsat text, něco zaškrtnout, vybrat mezi možnostmi atp. Takovéto skupině vstupních polí se říká formulář. Systém samozřejmě může pole ve formuláři předvyplnit a z polí tak udělat výstupní prvky, které rovněž budou uživatele informovat o stavu systému. Dalšími vstupními prvky jsou také tlačítka, které provádějí dané akce jako například odeslání formuláře. Výstupními prvky jsou například statické texty, tabulky nebo seznamy položek. 

\subsection{Tvorba uživatelského rozhraní}

Tvorba uživatelského rozhraní není vůbec jednoduchá záležitost. Vývojáři vkládají do tvorby uživatelského rozhraní velké úsilí a značné množství času. Bylo zjištěno, že uživatelské rozhraní zabírá přibližně 48\% kódu aplikace a zhruba 50\% času, který vývoji aplikace věnujeme \cite{towards-smart-design}. Další čas a úsilí také zabere testování rozhraní hlavně z hlediska použitelnosti, které opět stojí spoustu času a nákladů. Například vývojář mnohdy nedokáže odhadnout chování cílové skupiny, která systém bude používat, a tak se často dělají testy s koncovým uživatelem, u kterých se zkoumá, jak daný uživatel software ovládá. Z těchto testů často odhalíme, že uživatelské rozhraní je nedostačující a neposkytuje uživateli komfort při ovládání systému, který by poskystkout mělo. Z vlastní zkušenosti s tímto testem také vím, že velkým problémem je uživatelský vstup, který musí být validován, aby uživatel nevložil data, která jsou v rozporu s modelem, na který je rozhraní namapováno \cite{cernyTEA}. Také je žádoucí zobrazovat uživateli pouze to, co by vidět měl, například na základě jeho uživatelské role v systému. V neposlední řadě je také důležité, jak rozhraní vypadá. Důležitým aspektem rozhraní je, jakým způsobem jsou v něm reprezentována data , a také jak jsou uspořádány jeho jednotlivé části. Z výše uvedeného lze vidět, že je tvorba uživatelského rozhraní opravdu náročný a rozsáhlý proces. Právě proto je poskytovat pro systém více verzí uživatelských rozhraní, například pro různé platformy nebo pro různé uživatelské role, obtížný úkol \cite{cernyTEA}.

Softwarový systém se po nasazení musí také dále udržovat. Ne nadarmo je jedním z hlavních a kritických aspektů dobrého softwaru jeho udržovatelnost, anglicky Maintainability. Udržovatelnost můžeme definovat jako schopnost systému se dále měnit a vyjívet na základě požadavků zákazníka. Změny by přitom měly být lehce proveditelné a něměly by nějak výrazně ovlivnit stav systému. Požadavky na změnu lze očekávat vždy, neboť nevyhnutelně vznikají jako reakce na změny v podnikatelském prostředí (http://faculty.mu.edu.sa/public/uploads/1429431793.203Software%20Engineering%20by%20Somerville.pdf strana 8). Neboli musíme provést změny, jinak nás konkurence předčí. Bohužel však uživatelské rozhraní tuto vlastnost moc nesplňuje. 
Mějme například desktopovou a mobilní aplikaci, které obě obsahují formulář namapovaný na určitou entitu v databázovém modelu. Tento model se nějak změní, například v dané entitě rozdělíme jeden sloupec na dva. Naneštěstí neexistuje žádný machanismus, který by automaticky zaručil, že je UI v souladu s modelem \cite{cernyTEA}. Z pohledu vývojáře to pak znamená, že pokud změní databázový model, musí také změnit uživatelské rozhraní v obou klientských aplikacích, aby korespondovalo s novým databázovým modelem, což je jistá forma typové kontroly. Zde nejenom, že musí vývojář udělat dvakrát stejnou věc, ale také může udělat chybu, což vyústí v nefunkčnost systému. Také pokud se takový formulář vyskytuje třeba na pěti místech v aplikaci, změna je už časově náročnější, hůře proveditelná a ještě více náchylná na chybu vývojáře, který může na nějaký výskyt formuláře zapomenout.
Takovým zásahem do systému nemusí být jen změna databázového modelu, ale také změna validací uživatelského vstupu, které se týkají i bussiness modelu, nebo třeba změna rozložení či pořadí jednotlivých polí ve formuláři.

\subsection{Využití webových služeb pro zisk a odeslání dat}
Jak už bylo řečeno v grafickém uživatelském rozhraní máme výstupní grafické prvky, jako například tabulky či seznamy položek. Tyto komponenty jsou určeny k tomu, aby zobrazovaly uživateli určitá data. Otázkou je odkud se tato data berou. Je hned několik způsobů, kde mohou být data uložena. Jednou z možností je, že má aplikace vlastní databázi. Takováto aplikace není určena k tomu, aby komunikovala nebo sdílela data s dalšími instancemi této aplikace na jiných zařízeních. Pokud komunikaci chceme, je vhodná architektura klient-server. Server může mít vlastní databázi, ze které poskytuje klientům informace například prostřednictvím webových služeb. Webová služba umožňuje jednomu zařízení interakci s jiným zařízením prostřednictvím sítě\cite{wiki-ws}. V tomto případě je jedním zařízením server ,druhým klienstká aplikace a interakcí je myšlen vzájemný přenos dat. V mobilních aplikacích jsou velmi populární interpretací webových služeb RESTful Web Services využívající Representional State Transfer (REST), který byl navržen tak, aby získával data ze zdrojů pomocí jednotných identifikátorů zdrojů (URI), což jsou typicky odkazy na webu. Využívá se právě v aplikacích s klient-server architekturou a ke komunikaci používá HTTP protokol. Výhodou využití HTTP protokolu je, že jeho metody poskytují jednotné rozhraní pro manipulaci se zdroji dat. Http metoda PUT se využívá k vytvoření nového zdroje, DELETE zdroj maže, GET se používá pro získání aktuálního stavu zdroje v nějaké dané reprezentaci a POST stav zdroje upravuje \cite{oracle-ws}. 

Víme tedy, že klient je schopen získat ze serveru data pomocí HTTP dotazu. Aby přijatá data mohl reprezentovat v UI, musí znát jejich strukturu., což by mohl být problém. Naštěstí jsou data ve spojitosti s RESTful službami nejčastěji přenášena ve formě XML nebo alternativně ve formě JSON\cite{ws-formats}. Zmíněné formy dat vzikají serialializací objektů, jejichž definici můžeme většinou získat z dokumentace poskytovatele webové služby, stejně tak jako formát, který pro bude pro serializaci použitý. Je také nutné znát metodu, kterou lze pro využití zdroje použít, popřípadě dodatečné parametry, kterými lze webovou službu nastavit. Tato data na klientovi můžeme zpracovat více způsoby. Jednou z možností je napsat si vlastní parser. Dalším způsobem je využít nějaké knihovny, která umí data sama deserializovat do objektu. Obdobně to funguje i v případě odesílání dat. Zdroj webové služby definuje v jakém formátu data přijme a v dokumentaci opět nalezneme definici objektu, do kterého se bude snažit data deserializovat. Z toho plyne, že klientská aplikace se i při zisku i při odesílání dat musí adaptovat na určitou, předem danou strukturu. Tedy pokud se změní struktura objektu, ze kterého serializací data vznikají a deserializuje se do něj vstup z klienta, je nutné upravit i příslušná místa v klienstké aplikaci, která zpracování a odesílání uživatelského vstupu mají nastarosti.  

Představme si nyní následující problém. Mějme server a na něm model naříklad Tým, který obsahuje dva sloupce - název týmu a počet členů. Vytvoříme si klienstkou aplikaci, která tato data získá a zobrazí, například v tabulce. Nyní se rozhodneme, že by měl přibýt sloupec, obsahující zkratku týmu. Nejdříve se na to podíváme z pohledu zisku dat. Upravíme tedy model na serveru a v datech, která jsou poskytována webovou službou, tedy přibude další hodnota. Proto musíme upravit klientskou aplikaci, aby s těmito dodatečnými daty počítala a rovněž je zobrazila v tabulce. Pokud se však rozhodneme, že se nějaký sloupec odstraní, je situace o trochu složitější. Po získání dat nám na klientovi hodnota bude chybět. Pokud nad hodnotou provádíme nějaké operace a nemáme klienta správně ošetřeného, může to vyústit i v pád aplikace.

Nyní budeme data posílat. Předpokládejme, že jsme ještě neprovedli změny výše a klient je tedy v souladu s modelem na serveru. Je důležité poznamenat, že server může určovat, které hodnoty vyžaduje. Pokud tedy přidáme novou hodnotu, kterou server označí jako povinnou, bude pokus neupraveného klienta zaslat data neúspěšný, neboť je server odmítne. Musíme tedy klienta upravit tak, aby bylo možné novou hodnotu zadat, to znamená přidat nové vstupní pole a upravit parser, či objekt, ze kterého se data připravují serializací na odeslání. Nastane-li odstranění nějakého sloupce z modelu na serveru, bude to pro klienta opět problém, protože bude zasílat data obsahující hodnotu, kterou server nezná a ten data opět odmítne. Znovu je nutné klienta upravit. 

Poznamenejme ještě, že pokaždé, kdy je nutné upravit klienta, se musí vydat nová verze aplikace. Bohužel v dnešní době je možnost aktualizaci neprovést, a to hlavně na mobilních zařízeních, příkladem může být Google Play na Androidu \cite{android-auto-update}. Když si novou verzi člověk nenainstaluje, hrozí tvůrcům buď uživatel s nefunkční aplikací nebo chyba na serveru, záleží na provedené změně. Spousta vývojářů tohle řeší třeba podmínkami na verzi aplikace. Znamená to, že na serveru je stále stará verze modelu, která podporuje starou strukturu dat? Nebo označili na serveru nová pole za nepovinná? Druhým používaným řešením je vynutit aktualizaci aplikace, což se mi zdá jako docela dobré řešení, ale i zde se vyskytují otázky. Co když člověk třeba nemá dostatek mobilních dat na stažení nové verze aplikace? Co když na aktualizaci právě nemá čas? Tento problém bychom eliminovali, pokud by server klienta informoval o tom, co vyžaduje a klient by se dynamicky těmto potřebám přizpůsobil.

\subsection{Existující řešení}
Snažil jsem se najít existující řešení pro mobilní aplikace, které by vytvářelo definici komponenty, například formuláře, na základě modelu a které by pro zisk těchto definic využívalo webových služeb. Bohužel jsem nenašel žádné řešení, které by přesně odpovídalo těmto specifikacím, uvádím však řešení, která řeší alespoň jejich část. Dále pak uvádím projekt AFSwinx \cite{citation-needed}, který sice požadavky splňuje, ale není určen pro mobilní aplikace, nýbrž pro Java SE platformu a AspectFaces \cite{aspect-faces} , z něhož AFSwinx vychází a který je v základu určen pro Java EE aplikace.

\subsubsection{Řešení z IBM developerWorks}
Článek Build dynamic user interfaces with Android and XML \cite{dynamic-android-xml} popisuje možnost dynamického vytvoření formuláře z XML souboru pro Android aplikace. Podle návodu aplikace stáhne z URL adresy určitý XML soubor, ve kterém je nadefinována struktura formuláře. Návod dále ukazuje, jak stažené XML parsovat a dynamicky vytvořit na jeho základě v aplikaci formulář. Tento způsob tedy formulář centralizuje. Pokud se tedy formulář vyskytuje na více místech v aplikaci a je třeba ho změnit, stačí upravit daný XML soubor. Bohužel není XML dokument generován automaticky z modelu a není využito k jeho získání webových služeb, jinak by tento způsob byl pro naše účely řešením. Také je škoda, že na základě návodu nebyla vytvořena žádná knihovna, kterou by Android aplikace mohly používat.

\subsubsection{PHP Database Form}
http://phpdatabaseform.com/
PHP Database Form je rozšíření pro PHP. Toto rozšíření dokáže automaticky z modelu v databázi vytvořit HTML kód formuláře, včetně validací jednotlivých polí. Umožňuje vybrat pro vytvoření pouze některou část tabulky a to pomocí SQL dotazu. Dále pak umožňuje dodatečná nastavení. Lze nastavit názvy polí, jejich viditelnost, dodat validace tam, kde nejsou, nastavit, jak se bude pole zobrazovat atd. Hlavními výhodami tohoto rozšíření jsou: menší množství kódu, jednoduché validování dat a možnost upravit si formulář dle libosti pomocí CSS. Využití vyžaduje PHP verzi 5.3 a Apache, Tomcat nebo Microsoft IIS web server. PHP Database Form podporuje všechny majoritně využívané databáze a webové prohlížeče. Dnes už by se i toto rozšíření dalo použít pro mobilní aplikace, neboť existují možnosti vytvářet multiplatformní mobilní aplikace pomocí HTML, CSS a JavaScriptu, které spouští aplikaci na mobilním zařízení v režimu webového prohlížeče. Takovou možností je například Apache Cordova \cite{apache-cordova}.

\subsubsection{AspectFaces}
AspectFaces je framework, který se snaží o to, aby bylo UI generováno na základě modelu \cite{aspectdriven}, k čemuž využívá inspekci tříd. To umožní nadefinovat UI pouze jednou a veškeré změny v modelu jsou automaticky do uživatelského rozhraní reflektovány. UI lze nadefinovat v modelu pomocí velkého množství anotací z JPA, Hibernate nebo si lze nadefinovat i anotace vlastní. Lze určit například pravidla pro dané pole, pořadí v UI nebo třeba label. Framework zatím poskytuje dynamickou integraci pouze s JavaServer Faces 2.0, ale pracuje se na integraci i s jinými technologiemi. Poslední stabilní verze frameworku je 1.4.0 a je dostupný pod licencí LGPL v3.

\subsubsection{AFSwinx}
TODO citovat bakalářku od Martina
Tento framework byl vytvořen jako koncept a slouží pro generování uživatelského rozhraní v Java SE aplikacích využívajících pro tvorbu UI knihovnu Swing. Tento framework používá RESTful webové služby pro zisk definic komponent, díky kterým je schopen dynamicky postavit formulář či tabulku. Takové definice komponent vznikají za pomocí části frameworku AFRest, která ke generování dat využívá inspekce příslušného modelu na serveru, na který by měla být komponenta namapována. Jelikož se tvoří komponenta na základě tohoto modelu, nenastane tak, že by s ním nebyla v souladu. Inspekci tříd zprostředkovává knihovna AspectFaces, které věnuji samostatný odstavec. Definice komponenty je přenášena ve formátu JSON a obsahuje informace o komponentě, například její rozložení, pole, které má obsahovat nebo pravidla, která pro jednotlivíá pole platí. Pole z definice se v případě formuláře interpretuje jako vstupní políčko, v případě tabulky jako sloupec. 

\subsection{Cíle práce}
TODO citovat Martinovu bakalářku
Vzorem pro tento projekt je výše zmíněný framework AFSwinx. Framework se snaží o zjednodušení tvorby uživatelských rozhraní hlavně z hlediska množství kódu a udržovatelsnosti. Framework na straně serveru využívá inspekce tříd k vytvoření definice modelu, které poskytuje klientovi pomocí webových služeb, stejně tak jako data, kterými se má budoucí komponenta naplnit. Klient tyto informace pouze získává a interpretuje je. Klient také nemá informaci o celém procesu tvorby komponenty, zná pouze nutné informace jako je formát dat, například JSON, XML a připojení. Na vytvoření komponenty stačí klientovi pouze pár řádků kódu. Cílem této práce je vytvořit obdobný framework pro mobilní platformy Android a Windows Phone. Žádoucí je také některé prvky z AFSwinx znovupoužít. Cílem práce je také přinést do stávajícího frameworku něco navíc.

\chapter{Analýza}
\section{Funkční specifikace}
V rámci této práce bude zpracován framework ve dvou verzích, pro mobilní platformu Android a mobilní platformu Windows Phone. Musí umožňovat jednoduše vytvářet dva typy komponent, formulář, který umožní uživatelský vstup a list, pro zobrazení většího množství dat uživateli. Kromě vytvoření komponent je nutné poskytkout další funkce, které umožní práci s vytvořenými komponentami, jako je například odeslání dat z komponenty na server. Framework musí samozřejmě disponovat funkcionalitou, která umožní správné vytvoření a nastavení komponenty z hlediska zabezpečení, získávání dat a jejich vložení do komponenty, vzhledu komponenty či její lokalizace. Všechny funkční požadavky jsou uvedeny v následujícím seznamu položek.
\subsection{Funkční požadavky}

Framework by měl splňovat následující požadavky.
\begin{itemize}
\item Framework bude umožňovat automaticky vytvořit formulář nebo list na základě dat získaných ze serveru.
\item Framework bude umožňovat získat ze serveru data, kterými komponentu naplní.
\item Framework bude umožňovat naplnit formulář i list daty.
\item Framework bude umožňovat odeslat data z formuláře zpět na server.
\item Framework bude umožňovat používat lokalizační texty.
\item Framework bude umožňovat validaci vstupních dat na základě definice komponenty, kterou obdržel od serveru.
\item Framework bude umožňovat upravit vzhled komponenty pomocí skinů.
\item Framework bude umožňovat koncovému uživateli specifikovat zdroje definic komponent, dat a cíle pro jejich odeslání ve formátu XML.
\item Framework bude umožňovat vytvářet následující formulářová pole - textové, číselné, pro hesla, pro datum, dropdown pole, checkboxy, option buttony.
\item Framework bude umožňovat resetovat úpravy ve formuláři nebo formulář vyčistit.
\item Framework bude umožňovat získat data z formuláře i listu.
\item Framework bude umožňovat schovat validační chyby.
\item Framework bude umožňovat jednoduše získat komponentu i na jiném místě v programu, než kde ji vytvořil.
\item Framework bude umožňovat generování komponent určených pouze pro čtení. 
\end{itemize}

Pro uživatele, který bude framework využívat, bude proces tvorby komponenty zapouzdřen. Nemusí vědět, jak definice dat vypadá ani jak se komponenta tvoří či naplňuje daty. Bude potřebovat znát jen kód pro vytvoření komponenty, akce, které lze nad komponentou provádět a jak specifikovat, odkud se bere definice komponenty, data pro její naplnění a kam se případně data odešlou.

\section{Popis architektury a komunikace}
\subsection{Definice komponent}
TODO citovat bakalářku
Frameworky pro mobilní platformy Android a Windows Phone, které je cílem vytvořit, navazují, jak už bylo zmíněno, na projekt AFSwinx. Tento framework vytváří na straně serveru tzv. definice komponent, které komponentu popisují z hlediska vzhledu, rozložení i obsahu. Jedná se tedy o metadata \cite{https://en.wikipedia.org/wiki/Metadata}, neboli data, která popisují další data. Tyto definice framework poskytuje prozatím ve formátu JSON, plánuje se i XML, ale zatím není podporováno. Proto budeme s JSON formátem počítat i v tomto projektu. Cílem autora AFSwinx bylo, aby tyto definice komponent byly nezávislé na platformě, což i jejich využitím potvrdíme.
Definice typicky obsahuje tyto informace:
\begin{itemize}
\item název definice,
\item celkové rozložení komponenty,
\item seznam polí, které se v komponentě vyskytují.
\end{itemize}
Jednotlivá pole mají velké množství dalších vlastností, jako např. identifikátor, popisek, viditelnost, validační pravidla atp. S tvarem těchto definic bude framework počítat, na základě toho se definice bude na klientovi udržovat a z ní vytvářet uživatelské rozhraní. 

Taková definice vzniká na serveru na základě inspekce modelu. Za tu vděčíme knihovně AspectFaces, která inspekci vytváří ve formátu XML a AFSwinx ji zobecňuje a převádí do formátu JSON. Na serveru zastupuje roli modelu databázová entita a vlastnosti, které má inspekce modelu zachytit a do definice promítnout, jsou určeny datovými typy atributů a pomocí anotací. Inspekce dokáže do definice na základě datového typu nebo anotace promítnout například pořadí v UI, jakým widgetem bude daný atribut reprezentován či jaký label bude budoucí widget mít.
Definici komponenty je možné získat pomocí HTTP dotazu na konkrétní zdroj na serveru, který AFSwinx používá a je schopen nám takovou definici poskytnout. Tento konkrétní zdroj, poskytující definici komponenty, samozřejmě musíme specifikovat. Také lze určit dva další zdroje, jeden dat a zdroj, na který budeme odesílat uživatelský vstup. V požadavcích jsme definovali, aby framework umožňoval uživateli tyto zdroje specifikovat ve formátu XML. Již v AFSwinx byl pro to vytvořen XML soubor a k němu XML parser. V Android frameworku bude žádoucí z hlediska efektivity tento parser a soubor využít. Jelikož AFSwinx je napsaný v Javě a Windows Phone nepodporuje Javu, nýbrž jazyk C\#, a tedy ani import .jar souborů, bude nutné tento parser přepsat. Pro představu jak vypadá struktura zmíněného XML souboru, uvedeme definici všech tří zdrojů pro profilový formulář.
\begin{lstlisting}[caption=Ukázka XML specifikace zdrojů,
label={code:xmlSource}, basicstyle=\footnotesize]
<?xml version="1.0" encoding="UTF-8"?>
<connectionRoot xmlns:xsi="http://www.w3.org/2001/XMLSchema-instance">
   <connection id="personProfile">
      <metaModel>
         <endPoint>toms-cz.com</endPoint>
         <endPointParameters>/AFServer/rest/users/profile</endPointParameters>
         <protocol>http</protocol>
         <port></port>
         <header-param>
            <param>content-type</param>
            <value>Application/Json</value>
         </header-param>
      </metaModel>
      <data>
         <endPoint>toms-cz.com</endPoint>
         	<!-- ... obdobne jako metamodel-->
         <security-params>
            <security-method>basic</security-method>
            <userName>#{username}</userName>
            <password>#{password}</password>
         </security-params>
      </data>
      <send>
         <endPoint>toms-cz.com</endPoint>
            	<!--... obdobne jako metamodel -->
         <security-params>
                <!--... obdobne jako data -->
         </security-params>
      </send>
   </connection>
</connectionRoot>
\end{lstlisting}

Jak lze z ukázky vidět, jsou zdroje nadefinovány vlastně URL adresou rozdělenou na části a dodatečnými parametry, jako je forma dat, která lze očekávat nebo zabezpečení. Například definice profilového formuláře se nachází na adrese \url{http://toms-cz.com/AFServer/rest/users/profile} a očekáváme ho ve tvaru JSON souboru. Pokud by byl specifikován port, přibude za toms-cz.com ještě dvojtečka a jeho hodnota. Zdroj dat je nadefinován v uzlu <data> a kam se odešle uživatelský vstup určuje uzel <send>.
Za zmínku stojí výrazy ve složených závorkách označené vpředu hashtagem. Tyto výrazy jsou určeny k nahrazení. V AFSwinx se pak klíč ve složených závorkách hledá v mapě parametrů, kterou framework předává jako parametr metodě kontaktující zdroj, a nahrazuje se hodnotou v ní pod klíčem uloženou. Umožňuje to tak nadefinovat zdroj v XML souboru pouze jednou, například pro více různých uživatelů. Klíč a hodnotu si může uživatel nastavit sám, jen se musí shodovat klíče ve zmíněné mapě a v souboru. V zájmu znovupoužití XML souboru a parseru se tedy tomuto chování budou muset oba tvořené frameworky přizpůsobit.

\subsection{Reprezentace metadat ve frameworku}
Získaná metadata je potřeba ve frameworku rozparsovat a nějak rozumně udržovat. K tomuto účelu byl vytvořen návrh této části systému v podobě doménového modelu, vytvořeného za pomocí UML v programu Enterprise Architect. Dle Arlowa je UML, neboli česky unifikovaný modelovací jazyk, univerzální jazyk pro vizuální modelování systémů. UML je velice silný nástroj hlavně proto, že je srozumitelný pro lidi a zároveň je navržen tak, aby byl univerzálně implementovatelný \cite{viz-plocha-citace-arlow}. Doménový model definuje jaké části je potřeba v systému mít a jak se vzájemně ovlivňují. Jde tedy o model popisující strukturu i chování. Model části systému, který zachycuje návrh uložení metadat, je na následujícím obrázku.

\begin{figure}[h!]
\includegraphics[width=\textwidth]{figures/domainModel}
\caption{Doménový model objektů obsahující metadata o komponentě}
\label{img:metadataModel}
\end{figure}

Tento diagram si nyní popíšeme.
\subsubsection{ClassDefinition}
ClassDefinition udržuje informace o hlavním objektu metadat. Obsahuje informace o názvu objektu a rozložení komponenty. Dále drží definice 0 až N informací o polích, která se mají v komponentě vyskytnout. Součástí je také 0 až N vnitřích tříd, tedy referencí na objekt stejného typu ClassDefinition. Může se totiž stát, že v modelu, nad kterým je prováděna inspekce a ze kterého se metadata vytváří, obsahuje neprimitivní datový typ. Například v modelu Osoba to může být objekt typu Adresa, který obsahuje další atributy jako třeba název ulice či město. Tento typ je ale nutno v komponentě reprezentovat také, a tak je zevnitř provedena jeho inspekce, která je později v metadatech reprezentována jako vnitřní třída. 

\subsubsection{FieldInfo}
Tento objekt popisuje jednu proměnnou, nad kterou byla provedena inspekce a ze které se má vytvořit pole, které se v komponentě vyskytne. Informuje jaký widget má být při vytváření pole použit, určuje jednoznačný identifikátor pole v rámci komponenty, dále pak, zda má být tvořené pole viditelné a upravitelné, repektive jen pro čtení. Definuje jak bude pole rozloženo, hlavně z hlediska pozice labelu, jehož hodnota je ve FieldInfo rovněž zaznamenána. V neposlední řadě jsou v tomto objektu uloženy informace o validační pravidlech, oproti kterým se má validovat uživatelský vstup. Navíc ještě v případě, že by uživatel měl mít na výběr pouze z určitých předem definovaných možností, zahrnuje FieldInfo i informace o těchto možnostech. 

V tomto objektu je také uloženo, zda se jedná o vnitřní třídu, kterou jsme zmiňovali výše. Tento fakt je velmi důležitý, neboť záleží na pořadí polí v komponentě, ve kterém mají být vykreslovány. Inspekce modelu na serveru s tím počítá, a tak pole umístí na správné místo v metadatech a označí ho jako classType, tedy vnitřní třídu, jejíž popis můžeme nalézt v metadatech v části s vnitřními třídami. V rámci zachování správného pořadí vykreslení polí je tedy nutné, aby klienstký framework fakt, že se jedná o složený datový typ, při vytváření polí komponenty zaznamenal a na pozici, kde tuto skutečnost objeví, vložil pole, o nichž jsou informace uloženy v příslušné vnitřní třídě.

\subsubsection{ValidationRule}
Tento objekt popisuje pravidlo, které by měl splňovat uživatelský vstup ve vytvářeném poli. Obsahuje typ validace, který určuje o jakou validaci se jedná a případně hodnotu pravidla. Referenční framework AFSwinx obsahuje výčtový typ s názvy validací, které podporuje a které se mohou tedy v metadatech objevit. Například definuje validační pravidlo typu MAX a hodnotou je nějaké číslo. Tedy říká, že hodnota v poli nesmí přesáhnout číslo určené hodnotou pravidla. V obou frameworcích na mobilních platformách tedy bude nutné tyto validace umět zpracovat, přičemž typ zpracování bude na obou platformách trochu jiný. 

\subsubsection{FieldOption}
Pro určité typy widgetů, které mají být použity pro vytvoření polí, je nutné specifikovat možnosti, ze kterých si bude uživatel vybírat. Takovými widgety je například dropdown menu nebo skupina radio buttonů. Tento objekt popisuje tyto možnosti formou klíče a hodnoty. Klíč je hodnota, kterou by měl framework odesílat na server a hodnota by měla být zobrazována klientovi.

\subsubsection{LayoutProperties}
Objekt by měl být využit k popisu rozložení komponenty i jejich jednotlivých částí. Objekt definuje tři vlastnosti. Za prvé je to orientace, tedy ve směru jaké osy bude komponenta či její část vykreslována. Dále je to pak definice rozložení, která má určovat, jestli bude komponenta či její části vykreslovány v jednom či více sloupcích. Nakonec je to pozice labelu, který by měl být před nebo za vytvářeným polem nebo by něměl být vůbec zobrazen.




%*****************************************************************************
% Seznam literatury je v samostatnem souboru reference.bib. Ten
% upravte dle vlastnich potreb, potom zpracujte (a do textu
% zapracujte) pomoci prikazu bibtex a nasledne pdflatex (nebo
% latex). Druhy z nich alespon 2x, aby se poresily odkazy.

% originally following specification for bibliography formating was used
%\bibliographystyle{abbrv}

% Here is an improvment by Petr Dlouhy (April 2010).
% It is mainly for supervisors who expect Czech fomrating rules for references
% Additional feature is live url addresses to sources from your pdf file
% It requires the file csplainnat.bst (included in this sample zipfile).
\bibliographystyle{csplainnat}

%bibliographystyle{plain}
%\bibliographystyle{psc}
{
%JZ: 11.12.2008 Kdo chce mit v techto ukazkovych odkazech take odkaz na CSTeX:
\def\CS{$\cal C\kern-0.1667em\lower.5ex\hbox{$\cal S$}\kern-0.075em $}
\bibliography{reference}
}
%%%%%%%%%%%%%%%%%%%%%%%%%% 
% vše co následuje bude uvedeno v přílohách
\appendix	

\printnomenclature
\label{apx:zkratky}
\chapter{Seznam použitých zkratek}

\begin{description}
\item[CUI] Character User Interface
\item[GUI] Graphical User Interface
\item[UI] User Interface
\item[REST] Representational State Transfer
\item[URI] Uniform Resource Identifier
\item[HTTP] Hypertext Transfer Protocol
\item[JSON] JavaScript Object Notation
\item[XML] Extensible Markup Language
\item[XAML] Extensible Application Markup Language 
\item[Java SE] Java Platform Standart Edition
\item[Java EE] Java Platform Enterprise Edition
\item[URL] Uniform Resource Locator
\item[PHP] Personal Home Page
\item[HTML] HyperText Markup Language
\item[SQL] Structured Query Language
\item[CSS] Cascading Style Sheets
\item[JPA] Java Persistence API
\item[LGPL] GNU Lesser General Public License 
\item[UML] Unified Modeling Language
\item[C\#] C Sharp
\item[WP] Windows Phone


\end{description}
\chapter{UML diagramy a obrázky}
V této sekci naleznete použité UML diagramy a velké obrázky, na které bylo v textu odkazováno.

\begin{figure}
\begin{center}
\includegraphics[angle=270]{figures/createFormActivityDiagram}
\caption{Diagram aktivit popisující proces tvorby formuláře}
\label{img:createFormActivityDiagram}
\end{center}
\end{figure}

\begin{figure}
\begin{center}
\includegraphics{figures/formWorkActivityDiagram}
\caption{Diagram aktivit popisující práci s formulářem}
\label{img:formWorkActivityDiagram}
\end{center}
\end{figure}

\begin{figure}
\begin{center}
\includegraphics{figures/useCaseModel}
\caption{Model případů užití frameworku}
\label{img:useCaseModel}
\end{center}
\end{figure}




\chapter{Obsah přiloženého CD}
\textbf{\large Tato příloha je povinná pro každou práci. Každá práce musí totiž obsahovat přiložené CD. Viz dále.}

Může vypadat například takto. Váš seznam samozřejmě bude odpovídat typu vaší práce. (viz \cite{infodp}):

\begin{figure}[h]
\begin{center}
\includegraphics[width=14cm]{figures/seznamcd}
\caption{Seznam přiloženého CD --- příklad}
\label{fig:seznamcd}
\end{center}
\end{figure}

Na GNU/Linuxu si strukturu přiloženého CD můžete snadno vyrobit příkazem:\\ 
\verb|$ tree . >tree.txt|\\
Ve vzniklém souboru pak stačí pouze doplnit komentáře.

Z \textbf{README.TXT} (případne index.html apod.)  musí být rovněž zřejmé, jak programy instalovat, spouštět a jaké požadavky mají tyto programy na hardware.

Adresář \textbf{text}  musí obsahovat soubor s vlastním textem práce v PDF nebo PS formátu, který bude později použit pro prezentaci diplomové práce na WWW.

\end{document}
