\documentclass[11pt,twoside,a4paper]{book}  
% definice dokumentu
\usepackage[czech, english]{babel}
\usepackage[T1]{fontenc} 				% pouzije EC fonty 
\usepackage[utf8]{inputenc} 			% utf8 kódování vstupu 
\usepackage[square, numbers]{natbib}	% sazba pouzite literatury
\usepackage{indentfirst} 				% 1. odstavec jako v cestine, pro práci v aj možno zakomentovat
\usepackage{fancyhdr}					% tisk hlaviček a patiček stránek
\usepackage{nomencl} 					% umožňuje snadno definovat zkratky a jejich seznam


%%%%%%%%%%%%%%%%%%%%%%%%%%%%%%%%%%%%%%%%%%%%%%%%%%%%%%%%%%%%%%%
% informace o práci
\newcommand\WorkTitle{Servisně orientovaný aspektový vývoj uživatelských rozhraní pro mobilní aplikace}		% název
\newcommand\FirstandFamilyName{Pavel Matyáš}															% autor
\newcommand\Supervisor{Ing. Martin Tomášek}															% vedoucí

\newcommand\TypeOfWork{Bakalářská práce}	% typ práce [Diplomová práce | Bakalářská práce | Bachelor's Project | Master's Thesis ]	

% Nastavte následují podle vašeho oboru a programu (pomoc hledejte na http://www.fel.cvut.cz/cz/education/bk/prehled.html)								
\newcommand\StudProgram{Otevřená informatika, Bakalářský}	% program
\newcommand\StudBranch{Softwarové inženýrství}           					% obor

%%%%%%%%%%%%%%%%%%%%%%%%%%%%%%%%%%%%%%%%%%%%%%%%%%%%%%%%%%%%%%%
% minimální importy
\usepackage{graphicx}					% pro vkládání obrázků
\usepackage{k336_thesis_macros} 		% specialni makra pro formatovani DP a BP
\usepackage[
pdftitle={\WorkTitle},				% nastaví v informacích o pdf název
pdfauthor={\FirstandFamilyName},	% nastaví v informacích o pdf autora
colorlinks=true,					% před tiskem doporučujeme nastavit na false, aby odkazy a url nebyly šedé při ČB tisku
breaklinks=true,
urlcolor=red,
citecolor=blue,
linkcolor=blue,
unicode=true,
]
{hyperref}								% pro zobrazování "prokliknutelných" linků 

% rozšiřující importy
\usepackage{listings} 			%slouží pro tisk zdrojových kódů se syntax higlighting
\usepackage{algorithmicx} 		%slouží pro zápis algoritmů
\usepackage{algpseudocode} 		%slouží pro výpis pseudokódu

%%%%%%%%%%%%%%%%%%%%%%%%%%%%%%%%%%%%%%%%%%%%%%%%%%%%%%%%%%%%%%%
% příkazy šablony
\makenomenclature								% při překladu zajistí vytvoření pracovního souboru se seznamem zkratek

\let\oldUrl\url									% url adresy budou zobrazeny: <url> 
\renewcommand\url[1]{<\texttt{\oldUrl{#1}}>}

%%%%%%%%%%%%%%%%%%%%%%%%%%%%%%%%%%%%%%%%%%%%%%%%%%%%%%%%%%%%%%%
% vaše vlastní příkazy
\newcommand*{\nomExpl}[2]{#2 (#1)\nomenclature{#1}{#2}} 	% usnadňuje zápis zkratek : Slova ke Zkrácení (SZ)
\newcommand*{\nom}[2]{#1\nomenclature{#1}{#2}} 			% usnadňuje zápis zkratek : SZ



%%%%%%%%%%%%%%%%%%%%%%%%%%%%%%%%%%%%%%%%%%%%%%%%%%%%%%%%%%%%%%%
% vlastní dokument
%%%%%%%%%%%%%%%%%%%%%%%%%%%%%%%%%%%%%%%%%%%%%%%%%%%%%%%%%%%%%%%
\begin{document}
	
	%%%%%%%%%%%%%%%%%%%%%%%%%% 
	% nastavení jazyka, kterým je práce psána
	\selectlanguage{czech}	% podle jazyka práce nastavte na [czech | english]
	\translate				% nastaví české nebo anglické popisy (např. katedra -> department); viz k336_thesis_macros

	%%%%%%%%%%%%%%%%%%%%%%%%%%    
	% Poznamky ke kompletaci prace
	% Nasledujici pasaz uzavrenou v {} ve sve praci samozrejme 
	% zakomentujte nebo odstrante. 
	% Ve vysledne svazane praci bude nahrazena skutecnym 
	% oficialnim zadanim vasi prace.
	{
	\pagenumbering{roman} \cleardoublepage \thispagestyle{empty}
	\begin{figure}
	\begin{center}
	\includegraphics[width=\textwidth, height=\textheight]{figures/zadani}
	\end{center}
	\end{figure}
	\newpage
	}

	%%%%%%%%%%%%%%%%%%%%%%%%%%    
	% Titulni stranka / Title page 
	\coverpagestarts

	%%%%%%%%%%%%%%%%%%%%%%%%%%%    
	% Poděkovani / Acknowledgements 

	\acknowledgements
	\noindent
	Tímto bych rád poděkoval celé svojí rodině za podporu během studia. Také bych rád poděkoval vedoucímu mé bakalářské práce Ing. Martinu Tomáškovi za jeho čas, zkušenosti a dobré rady, které mi při psaní práce věnoval. Poděkování patří také všem, kteří se podíleli na testování nebo zapůjčili zařízení potřebné pro vývoj.


	%%%%%%%%%%%%%%%%%%%%%%%%%%%   
	% Prohlášení / Declaration 

	\declaration{V Praze dne 10.\,5.\,2016}
	%\declaration{In Kořenovice nad Bečvárkou on May 15, 2008}


	%%%%%%%%%%%%%%%%%%%%%%%%%%%%    
	% Abstrakt / Abstract 
 
	\abstractpage

	The goal of this work is creation of Android and Windows Phone frameworks, which can interpret user interface generated from model. This work is focused mainly on client side of applications, which are using client-server architecture, where client is considered a mobile application on corresponding platform. Frameworks are focused on creating graphical reprezentation of components from definition, which is generated from model on server, filling these components with data and eventually sending data from components back to server. Components are form, which is used mainly for capturing user input and list, which is space-efficient alternative of table, used for showing bigger amount of information. Generation of user interface from model allows client to quickly adapt to potential changes in server data model and provides centralization of data on server, which means that changes in data can be done in one place for all client applications. Nowadays there is only small group of interpreters of such generated user interface so main contribution of this work is that this group is extended by two new interpreters for mobile platforms.

	% Prace v cestine musi krome abstraktu v anglictine obsahovat i
	% abstrakt v cestine.
	\vglue28mm

	\noindent{\Huge \textbf{Abstrakt}}
	\vskip 2.75\baselineskip

	\noindent
	Cílem této práce je vytvořit frameworky pro Android a Windows Phone interpretující uživatelské rozhraní generované z modelu. Převážně se práce zaměřuje na klienstkou část aplikací užívající architekturu klient-server, klientem je myšlena mobilní aplikace na příslušné platformě. Frameworky se zaměřují na vytvoření grafické reprezentace komponent z definice, která vzniká generováním z modelu na serveru, dále pak na naplnění komponent daty a případně odeslání dat z komponent na server. Komponentami jsou formulář, který slouží hlavně pro zachycení uživatelského vstupu a list, který je prostorově úspornější alternativou tabulky, zobrazující větší množství informací. Způsob generování uživatelského rozhraní z modelu umožňuje klientovi se ihned přizpůsobit případným změnám v datovém modelu na serveru a zajišťuje, že jsou data na serveru centralizována, což znamená, že změny lze provést pro všechny klientské aplikace na jednom místě. Nyní však existuje jen malá skupina interpretů takto vygenerovaného uživatelského rozraní, tudíž hlavním přínosem této práce je rozšíření této skupiny o dva interprety pro mobilní platformy.
	\noindent

	%%%%%%%%%%%%%%%%%%%%%%%%%%    7
	% obsahy a seznamy
	\tableofcontents		% Obsah / Table of Contents 

	% pokud v práci nejsou obrázky nebo tabulky - odstraňte jejich seznam
	\listoffigures			% Obsah / Table of Contents 
	\listoftables			% Seznam tabulek / List of Tables

	%%%%%%%%%%%%%%%%%%%%%%%%%% 
	% začátek textu  
	\mainbodystarts
%**************************************************************

% Pro snadnejsi praci s vetsimi texty je rozumne tyto rozdelit
% do samostatnych souboru nejlepe dle kapitol a tyto potom vkladat
% pomoci prikazu \include{jmeno_souboru.tex} nebo \include{jmeno_souboru}.
% Napr.:
% \chapter{Úvod}
Tato bakalářská práce se zabývá analýzou, návrhem a otestováním frameworků pro mobilní platformy Android a Windows Phone interpretující uživatelské rozhraní z dat generovaných serverem z modelu, která jsou poskytována serverem pomocí webových služeb.

První část práce popisuje aktuální situaci v tvorbě UI, specifikuje požadavky a cíle práce a zkoumá již existující řešení. V druhé části se analyzují požadavky, které by měla práce splňovat, a návrh řešení, které stanovené požadavky a cíle splňuje. Ve třetí části je popsána struktura a vlastní implementace řešení. Poslední část obsahuje otestování řešení a ukazuje vzorové aplikace na dvou různých mobilních prostředích.

Práce obsahuje seznam použitých zkratek, které lze nalézt v příloze A, instalační příručku v příloze B, použité UML diagramy a obrázky viz příloha C, ukázky zdrojových kódů v příloze D a obsah CD s prací a zdrojovými kódy je v příloze E.
 
\section{Motivace}
Nedílnou součástí většiny dnešních aplikací je uživatelské rozhraní. Uživatelské rozhraní by mělo uživateli co nejvíce usnadňovat manipulaci se softwarem a tudíž být intuitivní a použitelné, mělo by mít adekvátní design, nejlépe korespondující s aktuálními trendy. Vývoj takového rozhraní je však časově velmi náročný proces, který zahrnuje nejenom samotný vývoj, ale také rozsáhlé testování, hlavně z hlediska funkčnosti a použitelnosti. Celý problém navíc umocňuje fakt, že se vývojáři snaží zajistit podporu softwaru na více platformách, neboť chtějí uživatelům nabídnout možnost operovat s jejich vytvořeným systémem nejen z počítače či laptopu, ale také z tabletu nebo mobilního zařízení. Mobilní verze grafického uživatelského rozhraní nebývá často moc rozdílná od rozhraní ostatních platforem v tom smyslu, že se v ní vyskytují převážně stejné grafické prvky, a tak pro mobilní verzi vzniká téměř indentická kopie tohoto rozhraní, což vede i k duplicitě v kódu aplikace. Problémem je, že se často uživatelská rozhraní mění, ať už se změna týká rozložení komponent, přidání nebo odebrání komponenty nebo třeba validace uživatelského vstupu, protože takováto změna se musí provést na všech platformách a někdy i na více místech v rámci aplikace, což může být v případě rozsáhlých systémů nejednoduchý úkol, který stojí vývojáře spoustu zbytečného času a může vést i ke vzniku nových typů chyb \cite{towards-smart-design}. Pokud bychom byli schopni nadefinovat uživatelské rozhraní jen jednou pro všechny platformy na jednom místě, tento problém bychom odstranili. Definice by tedy byla obecná, ale každá platforma je jiná a něčím specifická, proto je třeba vytvořit pro různé platformy frameworky, které obecnou definici pro danou platformu interpretují. 

Bylo mi nabítnuto vytvořit takovýto framework pro dvě mobilní platformy, kontrétně pro Android a Windows Phone, což mi přislo velmi užitečné a zajímavé, a proto jsem se rozhodl zpracovat toto téma jako bakalářskou práci.

% \include{2_teorie}
% atd...

%*****************************************************************************
\chapter{Úvod}
Tato bakalářská práce se zabývá analýzou, návrhem a otestováním frameworků pro mobilní platformy Android a Windows Phone interpretující uživatelské rozhraní z dat generovaných serverem z modelu, která jsou poskytována serverem pomocí webových služeb.

První část práce popisuje aktuální situaci v tvorbě UI, specifikuje požadavky a cíle práce a zkoumá již existující řešení. V druhé části se analyzují požadavky, které by měla práce splňovat, a návrh řešení, které stanovené požadavky a cíle splňuje. Ve třetí části je popsána struktura a vlastní implementace řešení. Poslední část obsahuje otestování řešení a ukazuje vzorové aplikace na dvou různých mobilních prostředích.

Práce obsahuje seznam použitých zkratek, které lze nalézt v příloze A, instalační příručku v příloze B, použité UML diagramy a obrázky viz příloha C, ukázky zdrojových kódů v příloze D a obsah CD s prací a zdrojovými kódy je v příloze E.
 
\section{Motivace}
Nedílnou součástí většiny dnešních aplikací je uživatelské rozhraní. Uživatelské rozhraní by mělo uživateli co nejvíce usnadňovat manipulaci se softwarem a tudíž být intuitivní a použitelné, mělo by mít adekvátní design, nejlépe korespondující s aktuálními trendy. Vývoj takového rozhraní je však časově velmi náročný proces, který zahrnuje nejenom samotný vývoj, ale také rozsáhlé testování, hlavně z hlediska funkčnosti a použitelnosti. Celý problém navíc umocňuje fakt, že se vývojáři snaží zajistit podporu softwaru na více platformách, neboť chtějí uživatelům nabídnout možnost operovat s jejich vytvořeným systémem nejen z počítače či laptopu, ale také z tabletu nebo mobilního zařízení. Mobilní verze grafického uživatelského rozhraní nebývá často moc rozdílná od rozhraní ostatních platforem v tom smyslu, že se v ní vyskytují převážně stejné grafické prvky, a tak pro mobilní verzi vzniká téměř indentická kopie tohoto rozhraní, což vede i k duplicitě v kódu aplikace. Problémem je, že se často uživatelská rozhraní mění, ať už se změna týká rozložení komponent, přidání nebo odebrání komponenty nebo třeba validace uživatelského vstupu, protože takováto změna se musí provést na všech platformách a někdy i na více místech v rámci aplikace, což může být v případě rozsáhlých systémů nejednoduchý úkol, který stojí vývojáře spoustu zbytečného času a může vést i ke vzniku nových typů chyb \cite{towards-smart-design}. Pokud bychom byli schopni nadefinovat uživatelské rozhraní jen jednou pro všechny platformy na jednom místě, tento problém bychom odstranili. Definice by tedy byla obecná, ale každá platforma je jiná a něčím specifická, proto je třeba vytvořit pro různé platformy frameworky, které obecnou definici pro danou platformu interpretují. 

Bylo mi nabítnuto vytvořit takovýto framework pro dvě mobilní platformy, kontrétně pro Android a Windows Phone, což mi přislo velmi užitečné a zajímavé, a proto jsem se rozhodl zpracovat toto téma jako bakalářskou práci.

\chapter{Popis problému a specifikace cíle}
\section{Popis problematiky}

Softwarový systém má sloužit člověku k řešení nějakého problému. Uživatel přitom často problém blíže specifikuje a systém musí mít způsob, jak uživateli sdělit jeho řešení. K tomu slouží uživatelská rozhraní, která umožňují vzájemnou komunikaci systému s uživatelem. Uživatelské rozhraní přitom není jednoduchá záležitost. V první řadě by mělo být navrženo tak, aby sloužilo uživateli. Mělo by mu umožnit jednoduchou interakci se systémem, být intuitivní, funkční a hlavně použitelné. 


\subsection{Různá uživatelská rozhraní}
Aby bylo uživatelské rozhraní použitelné a uživatelsky přívětivé je třeba prozkoumat, jakým způsobem člověk s aplikacemi spolupracuje. Nejen tímto se zabývá disciplína zvaná Human Computer Interaction, která zkoumá potřeby uživatelů z různorodých hledisek. Díky této disciplíně vzniklo množství uživatelských rozhraní, pomocí kterých může člověk s počítačem komunikovat. Jedním z takových rozhraní je textové uživatelské rozhraní, značené CUI, jehož typickým zástupcem je příkazová řádka. Dalším typem je Hlasové uživatelské rozhraní, které dokáže interpretovat povely zadané lidskou řečí. Nejrozšířenějším je však grafické uživatelské rozhraní, zkráceně GUI, které využívá grafické prvky. GUI se stalo velmi oblíbeným právě proto, že je jednoduché a grafické prvky v člověku vyvolávají podobnost s vnějším světem. Také není nutné znát žádně specifické příkazy, jako v případě příkazové řádky, nebo hlasové povely jako v případě hlasového rozhraní. Zmínil bych ještě multimodální rozhraní, která používají k interakci více lidských smyslů a tak jsou vhodná například i pro lidi s postižením \cite{uiTypes}. 

V softwarových systémech je nejběžnějsím způsobem interakce uživatele se systémem právě grafické uživatelské rozhraní a tím se taky v této práci zabýváme. Návrh GUI je potřeba důkladně zvážit, neboť závisí na mnoha aspektech. Důležité je, pro jaké zařízení GUI tvoříme, jaký účel má aplikace, která bude rozhraním disponovat a také stav uživatele a prostředí, ve kterém se nachází. Typ zařízení je důležitý hlavně proto, protože každé zařízení používá jiné ovládací prvky. Zatímco u mobilního zařízení můžeme očekávat použití dotykového displeje, u počítače zase použití myši, klávesnice nebo dokonce jiných externích vstupních zařízení, jako může být například grafický tablet. Také je třeba vzít v potaz jinou velikost displeje a jiná rozlišení jednotlivých zařízení. Účel aplikace ovlivňuje GUI hlavně z hlediska obsahu, tedy jaké komponenty je nutné mít, aby byla aplikace využívána k daném účelu. Těžko by asi někdo chtěl emailového klienta bez možnosti odeslat zprávu. Stav uživatele a prostředí může zase ovlivnit způsob ovládání aplikace. Příkladem může být palubní počítač v automobilu, na kterém by měl uživatel být schopen přepnout rádiovou stanici, aniž by se přestal věnovat řízení. 

Co ale takové grafické rozhraní nejčastěji obsahuje? Běžně GUI disponuje ovládacími a vizuálními prvky pomocí kterých lze aplikaci ovládat. Osobně bych GUI prvky rozdělil na vstupní a výstupní. Vstupní prvky zachycují uživatelský vstup a akce, které systém ovládají a výstupní zobrazují uživateli výsledky těchto akcí, data a aktuální stav systému. Vstupními prvky jsou nejčastěji vstupní pole, do kterých může uživatel napsat text, něco zaškrtnout, vybrat mezi možnostmi atp. Takovéto skupině vstupních polí se říká formulář. Systém samozřejmě může pole ve formuláři předvyplnit a z polí tak udělat výstupní prvky, které rovněž budou uživatele informovat o stavu systému. Dalšími vstupními prvky jsou také tlačítka, které provádějí dané akce jako například odeslání formuláře. Výstupními prvky jsou například statické texty, tabulky nebo seznamy položek. 

\subsection{Tvorba uživatelského rozhraní}

Tvorba uživatelského rozhraní není vůbec jednoduchá záležitost. Vývojáři vkládají do tvorby uživatelského rozhraní velké úsilí a značné množství času. Bylo zjištěno, že uživatelské rozhraní zabírá přibližně 48\% kódu aplikace a zhruba 50\% času, který vývoji aplikace věnujeme \cite{towards-smart-design}. Další čas a úsilí také zabere testování rozhraní hlavně z hlediska použitelnosti, které opět stojí spoustu času a nákladů. Například vývojář mnohdy nedokáže odhadnout chování cílové skupiny, která systém bude používat, a tak se často dělají testy s koncovým uživatelem, u kterých se zkoumá, jak daný uživatel software ovládá. Z těchto testů často odhalíme, že uživatelské rozhraní je nedostačující a neposkytuje uživateli komfort při ovládání systému, který by poskystkout mělo. Z vlastní zkušenosti s tímto testem také vím, že velkým problémem je uživatelský vstup, který musí být validován, aby uživatel nevložil data, která jsou v rozporu s modelem, na který je rozhraní namapováno \cite{cernyTEA}. Také je žádoucí zobrazovat uživateli pouze to, co by vidět měl, například na základě jeho uživatelské role v systému. V neposlední řadě je také důležité, jak rozhraní vypadá. Důležitým aspektem rozhraní je, jakým způsobem jsou v něm reprezentována data , a také jak jsou uspořádány jeho jednotlivé části. Z výše uvedeného lze vidět, že je tvorba uživatelského rozhraní opravdu náročný a rozsáhlý proces. Právě proto je poskytovat pro systém více verzí uživatelských rozhraní, například pro různé platformy nebo pro různé uživatelské role, obtížný úkol \cite{cernyTEA}.

Softwarový systém se po nasazení musí také dále udržovat. Ne nadarmo je jedním z hlavních a kritických aspektů dobrého softwaru jeho udržovatelnost, anglicky Maintainability. Udržovatelnost můžeme definovat jako schopnost systému se dále měnit a vyjívet na základě požadavků zákazníka. Změny by přitom měly být lehce proveditelné a něměly by nějak výrazně ovlivnit stav systému. Požadavky na změnu lze očekávat vždy, neboť nevyhnutelně vznikají jako reakce na změny v podnikatelském prostředí (http://faculty.mu.edu.sa/public/uploads/1429431793.203Software%20Engineering%20by%20Somerville.pdf strana 8). Neboli musíme provést změny, jinak nás konkurence předčí. Bohužel však uživatelské rozhraní tuto vlastnost moc nesplňuje. 
Mějme například desktopovou a mobilní aplikaci, které obě obsahují formulář namapovaný na určitou entitu v databázovém modelu. Tento model se nějak změní, například v dané entitě rozdělíme jeden sloupec na dva. Naneštěstí neexistuje žádný machanismus, který by automaticky zaručil, že je UI v souladu s modelem \cite{cernyTEA}. Z pohledu vývojáře to pak znamená, že pokud změní databázový model, musí také změnit uživatelské rozhraní v obou klientských aplikacích, aby korespondovalo s novým databázovým modelem, což je jistá forma typové kontroly. Zde nejenom, že musí vývojář udělat dvakrát stejnou věc, ale také může udělat chybu, což vyústí v nefunkčnost systému. Také pokud se takový formulář vyskytuje třeba na pěti místech v aplikaci, změna je už časově náročnější, hůře proveditelná a ještě více náchylná na chybu vývojáře, který může na nějaký výskyt formuláře zapomenout.
Takovým zásahem do systému nemusí být jen změna databázového modelu, ale také změna validací uživatelského vstupu, které se týkají i bussiness modelu, nebo třeba změna rozložení či pořadí jednotlivých polí ve formuláři.

\subsection{Využití webových služeb pro zisk a odeslání dat}
Jak už bylo řečeno v grafickém uživatelském rozhraní máme výstupní grafické prvky, jako například tabulky či seznamy položek. Tyto komponenty jsou určeny k tomu, aby zobrazovaly uživateli určitá data. Otázkou je odkud se tato data berou. Je hned několik způsobů, kde mohou být data uložena. Jednou z možností je, že má aplikace vlastní databázi. Takováto aplikace není určena k tomu, aby komunikovala nebo sdílela data s dalšími instancemi této aplikace na jiných zařízeních. Pokud komunikaci chceme, je vhodná architektura klient-server. Server může mít vlastní databázi, ze které poskytuje klientům informace například prostřednictvím webových služeb. Webová služba umožňuje jednomu zařízení interakci s jiným zařízením prostřednictvím sítě\cite{wiki-ws}. V tomto případě je jedním zařízením server ,druhým klienstká aplikace a interakcí je myšlen vzájemný přenos dat. V mobilních aplikacích jsou velmi populární interpretací webových služeb RESTful Web Services využívající Representional State Transfer (REST), který byl navržen tak, aby získával data ze zdrojů pomocí jednotných identifikátorů zdrojů (URI), což jsou typicky odkazy na webu. Využívá se právě v aplikacích s klient-server architekturou a ke komunikaci používá HTTP protokol. Výhodou využití HTTP protokolu je, že jeho metody poskytují jednotné rozhraní pro manipulaci se zdroji dat. Http metoda PUT se využívá k vytvoření nového zdroje, DELETE zdroj maže, GET se používá pro získání aktuálního stavu zdroje v nějaké dané reprezentaci a POST stav zdroje upravuje \cite{oracle-ws}. 

Víme tedy, že klient je schopen získat ze serveru data pomocí HTTP dotazu. Aby přijatá data mohl reprezentovat v UI, musí znát jejich strukturu., což by mohl být problém. Naštěstí jsou data ve spojitosti s RESTful službami nejčastěji přenášena ve formě XML nebo alternativně ve formě JSON\cite{ws-formats}. Zmíněné formy dat vzikají serialializací objektů, jejichž definici můžeme většinou získat z dokumentace poskytovatele webové služby, stejně tak jako formát, který pro bude pro serializaci použitý. Je také nutné znát metodu, kterou lze pro využití zdroje použít, popřípadě dodatečné parametry, kterými lze webovou službu nastavit. Tato data na klientovi můžeme zpracovat více způsoby. Jednou z možností je napsat si vlastní parser. Dalším způsobem je využít nějaké knihovny, která umí data sama deserializovat do objektu. Obdobně to funguje i v případě odesílání dat. Zdroj webové služby definuje v jakém formátu data přijme a v dokumentaci opět nalezneme definici objektu, do kterého se bude snažit data deserializovat. Z toho plyne, že klientská aplikace se i při zisku i při odesílání dat musí adaptovat na určitou, předem danou strukturu. Tedy pokud se změní struktura objektu, ze kterého serializací data vznikají a deserializuje se do něj vstup z klienta, je nutné upravit i příslušná místa v klienstké aplikaci, která zpracování a odesílání uživatelského vstupu mají nastarosti.  

Představme si nyní následující problém. Mějme server a na něm model naříklad Tým, který obsahuje dva sloupce - název týmu a počet členů. Vytvoříme si klienstkou aplikaci, která tato data získá a zobrazí, například v tabulce. Nyní se rozhodneme, že by měl přibýt sloupec, obsahující zkratku týmu. Nejdříve se na to podíváme z pohledu zisku dat. Upravíme tedy model na serveru a v datech, která jsou poskytována webovou službou, tedy přibude další hodnota. Proto musíme upravit klientskou aplikaci, aby s těmito dodatečnými daty počítala a rovněž je zobrazila v tabulce. Pokud se však rozhodneme, že se nějaký sloupec odstraní, je situace o trochu složitější. Po získání dat nám na klientovi hodnota bude chybět. Pokud nad hodnotou provádíme nějaké operace a nemáme klienta správně ošetřeného, může to vyústit i v pád aplikace.

Nyní budeme data posílat. Předpokládejme, že jsme ještě neprovedli změny výše a klient je tedy v souladu s modelem na serveru. Je důležité poznamenat, že server může určovat, které hodnoty vyžaduje. Pokud tedy přidáme novou hodnotu, kterou server označí jako povinnou, bude pokus neupraveného klienta zaslat data neúspěšný, neboť je server odmítne. Musíme tedy klienta upravit tak, aby bylo možné novou hodnotu zadat, to znamená přidat nové vstupní pole a upravit parser, či objekt, ze kterého se data připravují serializací na odeslání. Nastane-li odstranění nějakého sloupce z modelu na serveru, bude to pro klienta opět problém, protože bude zasílat data obsahující hodnotu, kterou server nezná a ten data opět odmítne. Znovu je nutné klienta upravit. 

Poznamenejme ještě, že pokaždé, kdy je nutné upravit klienta, se musí vydat nová verze aplikace. Bohužel v dnešní době je možnost aktualizaci neprovést, a to hlavně na mobilních zařízeních, příkladem může být Google Play na Androidu \cite{android-auto-update}. Když si novou verzi člověk nenainstaluje, hrozí tvůrcům buď uživatel s nefunkční aplikací nebo chyba na serveru, záleží na provedené změně. Spousta vývojářů tohle řeší třeba podmínkami na verzi aplikace. Znamená to, že na serveru je stále stará verze modelu, která podporuje starou strukturu dat? Nebo označili na serveru nová pole za nepovinná? Druhým používaným řešením je vynutit aktualizaci aplikace, což se mi zdá jako docela dobré řešení, ale i zde se vyskytují otázky. Co když člověk třeba nemá dostatek mobilních dat na stažení nové verze aplikace? Co když na aktualizaci právě nemá čas? Tento problém bychom eliminovali, pokud by server klienta informoval o tom, co vyžaduje a klient by se dynamicky těmto potřebám přizpůsobil.

\subsection{Existující řešení}
Snažil jsem se najít existující řešení pro mobilní aplikace, které by vytvářelo definici komponenty, například formuláře, na základě modelu a které by pro zisk těchto definic využívalo webových služeb. Bohužel jsem nenašel žádné řešení, které by přesně odpovídalo těmto specifikacím, uvádím však řešení, která řeší alespoň jejich část. Dále pak uvádím projekt AFSwinx \cite{citation-needed}, který sice požadavky splňuje, ale není určen pro mobilní aplikace, nýbrž pro Java SE platformu a AspectFaces \cite{aspect-faces} , z něhož AFSwinx vychází a který je v základu určen pro Java EE aplikace.

\subsubsection{Řešení z IBM developerWorks}
Článek Build dynamic user interfaces with Android and XML \cite{dynamic-android-xml} popisuje možnost dynamického vytvoření formuláře z XML souboru pro Android aplikace. Podle návodu aplikace stáhne z URL adresy určitý XML soubor, ve kterém je nadefinována struktura formuláře. Návod dále ukazuje, jak stažené XML parsovat a dynamicky vytvořit na jeho základě v aplikaci formulář. Tento způsob tedy formulář centralizuje. Pokud se tedy formulář vyskytuje na více místech v aplikaci a je třeba ho změnit, stačí upravit daný XML soubor. Bohužel není XML dokument generován automaticky z modelu a není využito k jeho získání webových služeb, jinak by tento způsob byl pro naše účely řešením. Také je škoda, že na základě návodu nebyla vytvořena žádná knihovna, kterou by Android aplikace mohly používat.

\subsubsection{PHP Database Form}
http://phpdatabaseform.com/
PHP Database Form je rozšíření pro PHP. Toto rozšíření dokáže automaticky z modelu v databázi vytvořit HTML kód formuláře, včetně validací jednotlivých polí. Umožňuje vybrat pro vytvoření pouze některou část tabulky a to pomocí SQL dotazu. Dále pak umožňuje dodatečná nastavení. Lze nastavit názvy polí, jejich viditelnost, dodat validace tam, kde nejsou, nastavit, jak se bude pole zobrazovat atd. Hlavními výhodami tohoto rozšíření jsou: menší množství kódu, jednoduché validování dat a možnost upravit si formulář dle libosti pomocí CSS. Využití vyžaduje PHP verzi 5.3 a Apache, Tomcat nebo Microsoft IIS web server. PHP Database Form podporuje všechny majoritně využívané databáze a webové prohlížeče. Dnes už by se i toto rozšíření dalo použít pro mobilní aplikace, neboť existují možnosti vytvářet multiplatformní mobilní aplikace pomocí HTML, CSS a JavaScriptu, které spouští aplikaci na mobilním zařízení v režimu webového prohlížeče. Takovou možností je například Apache Cordova \cite{apache-cordova}.

\subsubsection{AspectFaces}
AspectFaces je framework, který se snaží o to, aby bylo UI generováno na základě modelu \cite{aspectdriven}, k čemuž využívá inspekci tříd. To umožní nadefinovat UI pouze jednou a veškeré změny v modelu jsou automaticky do uživatelského rozhraní reflektovány. UI lze nadefinovat v modelu pomocí velkého množství anotací z JPA, Hibernate nebo si lze nadefinovat i anotace vlastní. Lze určit například pravidla pro dané pole, pořadí v UI nebo třeba label. Framework zatím poskytuje dynamickou integraci pouze s JavaServer Faces 2.0, ale pracuje se na integraci i s jinými technologiemi. Poslední stabilní verze frameworku je 1.4.0 a je dostupný pod licencí LGPL v3.

\subsubsection{AFSwinx}
TODO citovat bakalářku od Martina
Tento framework byl vytvořen jako koncept a slouží pro generování uživatelského rozhraní v Java SE aplikacích využívajících pro tvorbu UI knihovnu Swing. Tento framework používá RESTful webové služby pro zisk definic komponent, díky kterým je schopen dynamicky postavit formulář či tabulku. Takové definice komponent vznikají za pomocí části frameworku AFRest, která ke generování dat využívá inspekce příslušného modelu na serveru, na který by měla být komponenta namapována. Jelikož se tvoří komponenta na základě tohoto modelu, nenastane tak, že by s ním nebyla v souladu. Inspekci tříd zprostředkovává knihovna AspectFaces, které věnuji samostatný odstavec. Definice komponenty je přenášena ve formátu JSON a obsahuje informace o komponentě, například její rozložení, pole, které má obsahovat nebo pravidla, která pro jednotlivíá pole platí. Pole z definice se v případě formuláře interpretuje jako vstupní políčko, v případě tabulky jako sloupec. 

\subsection{Cíle práce}
TODO citovat Martinovu bakalářku
Vzorem pro tento projekt je výše zmíněný framework AFSwinx. Framework se snaží o zjednodušení tvorby uživatelských rozhraní hlavně z hlediska množství kódu a udržovatelsnosti. Framework na straně serveru využívá inspekce tříd k vytvoření definice modelu, které poskytuje klientovi pomocí webových služeb, stejně tak jako data, kterými se má budoucí komponenta naplnit. Klient tyto informace pouze získává a interpretuje je. Klient také nemá informaci o celém procesu tvorby komponenty, zná pouze nutné informace jako je formát dat, například JSON, XML a připojení. Na vytvoření komponenty stačí klientovi pouze pár řádků kódu. Cílem této práce je vytvořit obdobný framework pro mobilní platformy Android a Windows Phone. Žádoucí je také některé prvky z AFSwinx znovupoužít. Cílem práce je také přinést do stávajícího frameworku něco navíc.

\chapter{Analýza}
\section{Funkční specifikace}
V rámci této práce bude zpracován framework ve dvou verzích, pro mobilní platformu Android a mobilní platformu Windows Phone. Musí umožňovat jednoduše vytvářet dva typy komponent, formulář, který umožní uživatelský vstup a list, pro zobrazení většího množství dat uživateli. Kromě vytvoření komponent je nutné poskytkout další funkce, které umožní práci s vytvořenými komponentami, jako je například odeslání dat z komponenty na server. Framework musí samozřejmě disponovat funkcionalitou, která umožní správné vytvoření a nastavení komponenty z hlediska zabezpečení, získávání dat a jejich vložení do komponenty, vzhledu komponenty či její lokalizace. Všechny funkční požadavky jsou uvedeny v následujícím seznamu položek.
\subsection{Funkční požadavky}

Framework by měl splňovat následující požadavky.
\begin{itemize}
\item Framework bude umožňovat automaticky vytvořit formulář nebo list na základě dat získaných ze serveru.
\item Framework bude umožňovat získat ze serveru data, kterými komponentu naplní.
\item Framework bude umožňovat naplnit formulář i list daty.
\item Framework bude umožňovat odeslat data z formuláře zpět na server.
\item Framework bude umožňovat používat lokalizační texty.
\item Framework bude umožňovat validaci vstupních dat na základě definice komponenty, kterou obdržel od serveru.
\item Framework bude umožňovat upravit vzhled komponenty pomocí skinů.
\item Framework bude umožňovat koncovému uživateli specifikovat zdroje definic komponent, dat a cíle pro jejich odeslání ve formátu XML.
\item Framework bude umožňovat vytvářet následující formulářová pole - textové, číselné, pro hesla, pro datum, dropdown pole, checkboxy, option buttony.
\item Framework bude umožňovat resetovat úpravy ve formuláři nebo formulář vyčistit.
\item Framework bude umožňovat získat data z formuláře i listu.
\item Framework bude umožňovat schovat validační chyby.
\item Framework bude umožňovat jednoduše získat komponentu i na jiném místě v programu, než kde ji vytvořil.
\item Framework bude umožňovat generování komponent určených pouze pro čtení. 
\end{itemize}

Pro uživatele, který bude framework využívat, bude proces tvorby komponenty zapouzdřen. Nemusí vědět, jak definice dat vypadá ani jak se komponenta tvoří či naplňuje daty. Bude potřebovat znát jen kód pro vytvoření komponenty, akce, které lze nad komponentou provádět a jak specifikovat, odkud se bere definice komponenty, data pro její naplnění a kam se případně data odešlou.

\section{Popis architektury a komunikace}
\subsection{Definice komponent}
TODO citovat bakalářku
Frameworky pro mobilní platformy Android a Windows Phone, které je cílem vytvořit, navazují, jak už bylo zmíněno, na projekt AFSwinx. Tento framework vytváří na straně serveru tzv. definice komponent, které komponentu popisují z hlediska vzhledu, rozložení i obsahu. Jedná se tedy o metadata \cite{https://en.wikipedia.org/wiki/Metadata}, neboli data, která popisují další data. Tyto definice framework poskytuje prozatím ve formátu JSON, plánuje se i XML, ale zatím není podporováno. Proto budeme s JSON formátem počítat i v tomto projektu. Cílem autora AFSwinx bylo, aby tyto definice komponent byly nezávislé na platformě, což i jejich využitím potvrdíme.
Definice typicky obsahuje tyto informace:
\begin{itemize}
\item název definice,
\item celkové rozložení komponenty,
\item seznam polí, které se v komponentě vyskytují.
\end{itemize}
Jednotlivá pole mají velké množství dalších vlastností, jako např. identifikátor, popisek, viditelnost, validační pravidla atp. S tvarem těchto definic bude framework počítat, na základě toho se definice bude na klientovi udržovat a z ní vytvářet uživatelské rozhraní. 

Taková definice vzniká na serveru na základě inspekce modelu. Za tu vděčíme knihovně AspectFaces, která inspekci vytváří ve formátu XML a AFSwinx ji zobecňuje a převádí do formátu JSON. Na serveru zastupuje roli modelu databázová entita a vlastnosti, které má inspekce modelu zachytit a do definice promítnout, jsou určeny datovými typy atributů a pomocí anotací. Inspekce dokáže do definice na základě datového typu nebo anotace promítnout například pořadí v UI, jakým widgetem bude daný atribut reprezentován či jaký label bude budoucí widget mít.
Definici komponenty je možné získat pomocí HTTP dotazu na konkrétní zdroj na serveru, který AFSwinx používá a je schopen nám takovou definici poskytnout. Tento konkrétní zdroj, poskytující definici komponenty, samozřejmě musíme specifikovat. Také lze určit dva další zdroje, jeden dat a zdroj, na který budeme odesílat uživatelský vstup. V požadavcích jsme definovali, aby framework umožňoval uživateli tyto zdroje specifikovat ve formátu XML. Již v AFSwinx byl pro to vytvořen XML soubor a k němu XML parser. V Android frameworku bude žádoucí z hlediska efektivity tento parser a soubor využít. Jelikož AFSwinx je napsaný v Javě a Windows Phone nepodporuje Javu, nýbrž jazyk C\#, a tedy ani import .jar souborů, bude nutné tento parser přepsat. Pro představu jak vypadá struktura zmíněného XML souboru, uvedeme definici všech tří zdrojů pro profilový formulář.
\begin{lstlisting}[caption=Ukázka XML specifikace zdrojů,
label={code:xmlSource}, basicstyle=\footnotesize]
<?xml version="1.0" encoding="UTF-8"?>
<connectionRoot xmlns:xsi="http://www.w3.org/2001/XMLSchema-instance">
   <connection id="personProfile">
      <metaModel>
         <endPoint>toms-cz.com</endPoint>
         <endPointParameters>/AFServer/rest/users/profile</endPointParameters>
         <protocol>http</protocol>
         <port></port>
         <header-param>
            <param>content-type</param>
            <value>Application/Json</value>
         </header-param>
      </metaModel>
      <data>
         <endPoint>toms-cz.com</endPoint>
         	<!-- ... obdobne jako metamodel-->
         <security-params>
            <security-method>basic</security-method>
            <userName>#{username}</userName>
            <password>#{password}</password>
         </security-params>
      </data>
      <send>
         <endPoint>toms-cz.com</endPoint>
            	<!--... obdobne jako metamodel -->
         <security-params>
                <!--... obdobne jako data -->
         </security-params>
      </send>
   </connection>
</connectionRoot>
\end{lstlisting}

Jak lze z ukázky vidět, jsou zdroje nadefinovány vlastně URL adresou rozdělenou na části a dodatečnými parametry, jako je forma dat, která lze očekávat nebo zabezpečení. Například definice profilového formuláře se nachází na adrese \url{http://toms-cz.com/AFServer/rest/users/profile} a očekáváme ho ve tvaru JSON souboru. Pokud by byl specifikován port, přibude za toms-cz.com ještě dvojtečka a jeho hodnota. Zdroj dat je nadefinován v uzlu <data> a kam se odešle uživatelský vstup určuje uzel <send>.
Za zmínku stojí výrazy ve složených závorkách označené vpředu hashtagem. Tyto výrazy jsou určeny k nahrazení. V AFSwinx se pak klíč ve složených závorkách hledá v mapě parametrů, kterou framework předává jako parametr metodě kontaktující zdroj, a nahrazuje se hodnotou v ní pod klíčem uloženou. Umožňuje to tak nadefinovat zdroj v XML souboru pouze jednou, například pro více různých uživatelů. Klíč a hodnotu si může uživatel nastavit sám, jen se musí shodovat klíče ve zmíněné mapě a v souboru. V zájmu znovupoužití XML souboru a parseru se tedy tomuto chování budou muset oba tvořené frameworky přizpůsobit.

\subsection{Reprezentace metadat ve frameworku}
Získaná metadata je potřeba ve frameworku rozparsovat a nějak rozumně udržovat. K tomuto účelu byl vytvořen návrh této části systému v podobě doménového modelu, vytvořeného za pomocí UML v programu Enterprise Architect. Dle Arlowa je UML, neboli česky unifikovaný modelovací jazyk, univerzální jazyk pro vizuální modelování systémů. UML je velice silný nástroj hlavně proto, že je srozumitelný pro lidi a zároveň je navržen tak, aby byl univerzálně implementovatelný \cite{viz-plocha-citace-arlow}. Doménový model definuje jaké části je potřeba v systému mít a jak se vzájemně ovlivňují. Jde tedy o model popisující strukturu i chování. Model části systému, který zachycuje návrh uložení metadat, je na následujícím obrázku.

\begin{figure}[h!]
\includegraphics[width=\textwidth]{figures/domainModel}
\caption{Doménový model objektů obsahující metadata o komponentě}
\label{img:metadataModel}
\end{figure}

Tento diagram si nyní popíšeme.
\subsubsection{ClassDefinition}
ClassDefinition udržuje informace o hlavním objektu metadat. Obsahuje informace o názvu objektu a rozložení komponenty. Dále drží definice 0 až N informací o polích, která se mají v komponentě vyskytnout. Součástí je také 0 až N vnitřích tříd, tedy referencí na objekt stejného typu ClassDefinition. Může se totiž stát, že v modelu, nad kterým je prováděna inspekce a ze kterého se metadata vytváří, obsahuje neprimitivní datový typ. Například v modelu Osoba to může být objekt typu Adresa, který obsahuje další atributy jako třeba název ulice či město. Tento typ je ale nutno v komponentě reprezentovat také, a tak je zevnitř provedena jeho inspekce, která je později v metadatech reprezentována jako vnitřní třída. 

\subsubsection{FieldInfo}
Tento objekt popisuje jednu proměnnou, nad kterou byla provedena inspekce a ze které se má vytvořit pole, které se v komponentě vyskytne. Informuje jaký widget má být při vytváření pole použit, určuje jednoznačný identifikátor pole v rámci komponenty, dále pak, zda má být tvořené pole viditelné a upravitelné, repektive jen pro čtení. Definuje jak bude pole rozloženo, hlavně z hlediska pozice labelu, jehož hodnota je ve FieldInfo rovněž zaznamenána. V neposlední řadě jsou v tomto objektu uloženy informace o validační pravidlech, oproti kterým se má validovat uživatelský vstup. Navíc ještě v případě, že by uživatel měl mít na výběr pouze z určitých předem definovaných možností, zahrnuje FieldInfo i informace o těchto možnostech. 

V tomto objektu je také uloženo, zda se jedná o vnitřní třídu, kterou jsme zmiňovali výše. Tento fakt je velmi důležitý, neboť záleží na pořadí polí v komponentě, ve kterém mají být vykreslovány. Inspekce modelu na serveru s tím počítá, a tak pole umístí na správné místo v metadatech a označí ho jako classType, tedy vnitřní třídu, jejíž popis můžeme nalézt v metadatech v části s vnitřními třídami. V rámci zachování správného pořadí vykreslení polí je tedy nutné, aby klienstký framework fakt, že se jedná o složený datový typ, při vytváření polí komponenty zaznamenal a na pozici, kde tuto skutečnost objeví, vložil pole, o nichž jsou informace uloženy v příslušné vnitřní třídě.

\subsubsection{ValidationRule}
Tento objekt popisuje pravidlo, které by měl splňovat uživatelský vstup ve vytvářeném poli. Obsahuje typ validace, který určuje o jakou validaci se jedná a případně hodnotu pravidla. Referenční framework AFSwinx obsahuje výčtový typ s názvy validací, které podporuje a které se mohou tedy v metadatech objevit. Například definuje validační pravidlo typu MAX a hodnotou je nějaké číslo. Tedy říká, že hodnota v poli nesmí přesáhnout číslo určené hodnotou pravidla. V obou frameworcích na mobilních platformách tedy bude nutné tyto validace umět zpracovat, přičemž typ zpracování bude na obou platformách trochu jiný. 

\subsubsection{FieldOption}
Pro určité typy widgetů, které mají být použity pro vytvoření polí, je nutné specifikovat možnosti, ze kterých si bude uživatel vybírat. Takovými widgety je například dropdown menu nebo skupina radio buttonů. Tento objekt popisuje tyto možnosti formou klíče a hodnoty. Klíč je hodnota, kterou by měl framework odesílat na server a hodnota by měla být zobrazována klientovi.

\subsubsection{LayoutProperties}
Objekt by měl být využit k popisu rozložení komponenty i jejich jednotlivých částí. Objekt definuje tři vlastnosti. Za prvé je to orientace, tedy ve směru jaké osy bude komponenta či její část vykreslována. Dále je to pak definice rozložení, která má určovat, jestli bude komponenta či její části vykreslovány v jednom či více sloupcích. Nakonec je to pozice labelu, který by měl být před nebo za vytvářeným polem nebo by něměl být vůbec zobrazen.



\chapter{Implementace}
\section{Architektura}
Jak už bylo popsáno, tato práce se zabývá tvorbou dvou klienstkých frameworků pro dvě různé mobilní platformy. Ke správnému fungování frameworků je předpokládána serverová část, která generuje data v určité struktuře a formátu, z nichž oba frameworky umí vytvořit a zobrazit koncovému uživateli prvky grafického uživatelského rozhraní. Použití frameworků na jednotlivých zařízeních zachycuje diagram nasazení na obrázku \ref{img:deploymentDiagram}. Diagram nasazení je určen k tomu, aby zobrazil, jak je architektura softwaru namapovaná na architekturu hardwaru. Jedná se o diagram, který patří do implementační fáze, avšak často již vzniká určitá první verze ve fázi návrhové a poté se doplňuje \cite{UmlArlow}.

Obrázek \ref{img:deploymentDiagram} zobrazuje tři zařízení - server, Android klienta a Windows Phone klienta. Pro účely vývoje těchto frameworků byla na serveru nasazena Java EE aplikace AFServer, která za pomoci AFRest \cite{tomasek-thesis} využívající AspectFaces \cite{aspect-faces} generuje definice komponent z modelu, které upraví a poskytuje klientovi v požadovaném formátu, který je může získat pomocí http dotazu. Android klient interpretuje tuto specifikaci komponenty za pomoci frameworku AFAndroid, který využívá stejných částí jako serverová strana, což zajišťuje kompatibilitu objektů na serverové a klienstké straně. Windows Phone klient interpretuje definici za pomoci AFWinPhone, který bohužel neumožňuje využívat stejných částí jako server, neboť běží na rozdílných platformách. Aby bylo zajištěno stejné chování jako u Android klienta, byly tyto části znovu vytvořeny.

Aby klienti mohli framework využívat, musí ho nejdříve vložit. V případě Androidu se framework kompiluje do AAR souboru, který lze do projektů přidat jako Gradle závislost. Gradle je systém pro build projektů a správu závislostí nejen v Androidu. V současné době ho lze využít i v C, C++, Java aplikacích \cite{gradle}. Závislost může být lokální nebo se může stahovat z online repozitáře. Jelikož není AFAndroid nyní k dispozici v žádném z online repozitářů, je nutné ho přidat lokálně. To lze udělat přidáním AAR souboru do složky lib nebo vytvořit z knihovny nový modul. Windows Phone framework je zkompilován do DLL souboru, který je v současné chvíli nutné také přidat lokálně jako knihovnu mezi reference.

\subsection{Komponenty}
Oba frameworky jsou schopny generovat dva typy komponent - formulář a list. Obě tyto komponenty dědí od třídy AFComponent, která představuje společnou část obou komponent. AFComponent implementuje rozhraní třídy AbstractComponent, která definuje metody pro získání definice komponenty, naplnění daty, generování dat pro odeslání a jejich odeslání na server včetně validace. Tyto metody musí AFComponent nebo některá z tříd, která od něj dědí, nuceně implementovat. Společná část komponent konkrétně implementuje metody pro získání definice ze serveru a získání dat, kterými se má komponenta naplnit, protože tyto části jsou jak pro formulář, tak pro list identické. Způsob reprezentace komponenty a vložení získaných dat řeší jednotlivé komponenty každá jiným způsobem. List nedisponuje stejnou funkcionalitou jako formulář, je jen pro čtení a proto nepodporuje metody pro generování a následné odesílání dat či jejich validaci. Struktura části Android frameworku zachycující komponenty je na obrázku \ref{img:classDiagramComponents} popsána diagramem tříd. Diagram pro tuto část ve Windows Phone frameworku je až na syntaktické odlišnosti shodný.  

Obě komponenty obsahují metodu, která získá jejich grafickou reprezentaci, kterou může vývojář vložit do libovolné části uživatelského rozhraní, neboť tato metoda má návratový typ u Android verze View a u WP verze FrameworkElement, což jsou jedny ze základních prvků GUI na těchto platformách a lze je vložit do jakéhokoliv jiného elementu. V AFSwinx \cite{tomasek-thesis} komponenty rovnou dědily od třídy JPanel a tudíž metodu pro zisk grafické reprezentace nepotřebovaly. Tento přístup však vyústil v to, že se metody poskytované komponentou mísily s nepřeberným množstvím metod příslušících třídě JPanel, což způsobovalo dle uživatelského testu nepřehlednost a zhoršenou orientaci v poskytovaných metodách, proto byl zvolen výše zmíněný přístup. 

Vytvářené komponenty se ve frameworku určitým způsobem ukládají, aby bylo možné je získat a pracovat s nimi v celém programu, ne jen v místě, kde byly vytvořeny. Konkrétně se komponenty skladují ve třídách AFAndroid pro Android a AfWindowsPhone pro WP, zobrazených na obrázku \ref{img:facades}. Tyto třídy jsou implementovány jako singleton, což je návrhový vzor, který se využívá, když je potřeba mít pouze jednu instanci této třídy, ke které lze přistupovat z více míst \cite{gamma}. Singleton často bývá součástí jiných návrhových vzorů, jako je například Facade neboli fasáda. Fasáda se používá v případě, že programátor chce poskytnout jednoduché rozhraní pro ovládání složitějšího systému a tím klienty odstínit od vnitřní implementace systému schovaného pod fasádou \cite{gamma}. AFAndroid a AFWinPhone tedy neslouží pouze jako sklad vytvořených komponent, ale také jako fasády pro ovládání frameworků. Nabízí získání vytvořené komponenty, jejich smazání, zisk builderů pro tvorbu komponent a nastavení základního skinu, který se použije při sestavování komponent, není-li při inicializaci builderu specifikováno jinak. Proces sestavení komponenty je složitější proces několika funkcí, které na sebe musí navazovat ve správném pořadí, proto, kdyby tyto třídy neexistovaly, nebylo by použití frameworků vůbec snadné.

\begin{figure}[h!]
\centering
\includegraphics[width=0.8\linewidth, trim=4 4 4 4, clip]{figures/facades}
\caption{Třídy AFAndroid a AfWindowsPhone sloužící jako fasády pro ovládání frameworků}
\label{img:facades}
\end{figure}

Přiklad vytvoření komponenty v Android frameworku, konkrétně formuláře, za použití fasády AFAndroid je v ukázce kódu \ref{code:createForm}. Kód ukazuje, že je třeba za použití fasády získat builder, který je nutné nainicializovat. Inicializace zahrnuje určení zdroje, ze kterého se má definice komponenty získat, definuje InputStream s načteným souborem s definicemi zdrojů a klíč, pod kterým se získá konkrétní připojení. Soubor byl popsán v rámci analýzy a jeho struktura je ukázána v části kódu \ref{code:xmlSource}. Vývojář si komponentu také při inicializaci builderu pojmenuje, tedy přiřadí jí jednoznačný textový řetězec, pod kterým se komponenta uloží do vytvořených komponent, a na základě tohoto řetězce ji lze odtud skrz fasádu získat. Existuje také přetížená metoda pro inicializaci builderu, která umožnuje přidat dodatečná nastavení pro připojení. Po inicializaci builderu lze nadefinovat také skin, který se při vytváření komponenty použije. Jelikož je komponenta vytvářena dynamicky včetně jejích vlastností a vzhledu v rámci metody, která je pro vývojáře zapouzdřená, jediným způsobem, jak pohodlně upravit její vzhled, je nastavit skin před zahájením její tvorby. Jelikož má vývojář k některým částem komponenty přístup i po jejím vytvoření, lze vzhled komponenty upravit i později, vyžaduje to však znalosti tvorby GUI pro danou platformu, což je složitější. Po dokončení výše zmíněného nastavení lze komponentu vytvořit a provádět nad ní akce, které poskytuje. Pro Windows Phone je kód identický, pouze se používá WP fasáda AfWindowsPhone a místo InputStreamu stačí nadefinovat jméno souboru se zdroji, respektive cesta k němu. Poté je nutné komponentu přetypovat na formulář či list, protože metoda createComponent vrací ve WP frameworku AFComponent místo konkrétního AFForm nebo AFList.

V případě selhání systému, například se nelze připojit ke zdroji na serveru, je vyhazována výjimka. Vývojář může výjimku libovolně zpracovat, například zobrazit dialog s chybou.

\begin{lstlisting}[float=h, caption=Ukázka tvorby formuláře,
label={code:createForm}, basicstyle=\footnotesize]
InputStream connectionResource = getResources().openRawResource(R.raw.connection);
 try {
	AFForm form = 
		AFAndroid.getInstance().getFormBuilder()
		.initBuilder(getActivity(), LOGIN_FORM_NAME, connectionResource, 
		LOGIN_FORM_CONNECTION_KEY)
		.setSkin(new LoginSkin(getContext()))
		.createComponent();
} catch (Exception e) {
	//zpracovani vyjimky, ktera mohla nastat pri vytvareni komponenty
}
\end{lstlisting} 

\section{Komunikace serveru a klienta}
Jak již bylo zmíněno v analýze, server poskytuje zdroje pro získání definice komponent a dat, kterými má být komponenta naplněna, a zdroje, na které lze odeslat z komponenty uživatelský vstup. K nadefinování těchto zdrojů je potřeba vytvořit XML soubor, který je popsán v ukázce kódu \ref{code:xmlSource}. Tento soubor se zpracuje pomocí příslušného XML parseru a výsledkem je třída AFSwinxConnectionPack, která má v sobě uloženy jednotlivé části daného připojení, kokrétně kde je uložena definice komponenty, data a kam lze odeslat uživatelský vstup \cite{tomasek-thesis}. V průběhu tvorby komponenty se tyto části připojení využijí k zisku informací, které jsou na definovaných zdrojích, na něž připojení odkazuje, uloženy. Zisk informací je proveden pomocí HTTP požadavku, který vytváří a odesílá třída RequestMaker. Způsob provedení tohoto požadavku je na obou platformách dosti rozdílný, avšak mají společné to, že musí být asynchronní. V mobilních aplikacích se requesty musí provádět asynchronně na jiném vlákně než na hlavním, neboť hlavní vlákno spravuje celé UI aplikace a to by v momentě, kdy se požadavek provádí, přestalo reagovat na akce uživatele, což je nežádoucí. HTTP požadavek v případě, že se získává definice komponenty nebo data, vrací textový řetězec, který obsahuje JSON objekt převedený na textovou reprezentaci, v případě odesílání na server se akce pouze provede.

V Androidu se standardně HTTP requesty řeší v rámci objektu typu AsyncTask \cite{asynctask}, který umožňuje provést akci v pozadí, a výsledky akce předat hlavnímu UI vláknu. Tím se vývojáři Android aplikací vyhnou přímé práci s vlákny. Třída RequestMaker od AsyncTask dědí a přetěžuje metodu pro práci v pozadí, kde se celý požadavek vytvoří, odešle a získá se odpověď. Tvorbu a práci s požadavkem má na starosti třída HttpURLConnection, před Android 6.0 by bylo možné ještě použít Apache HttpClient, který byl s příchodem nové verze Androidu odstraněn. Lze ho po určitém nastavení nadále používat, ale ukazuje se, že HttpURLConnection je efektivnější z hlediska využití sítě a spotřeby energie \cite{httpclientremoval}. 

Specifikace požadavku, tj. http metoda, url adresa, data, content-type a případně bezpečnostní omezení, se předávají třídě v konstruktoru. Pokud je definované bezpečnostní omezení, je použita autorizace typu basic. Během procesu může dojít k výjimce a jelikož metoda, která pracuje na pozadí nemůže zpropagovat výjimku o úroveň výš, je nutné výjimky řešit v rámci zmíněné metody. Je ale žádoucí ponechat řešení těchto výjimek na vývojáři, proto byla zavedena alternativa, ve které metoda běžící na pozadí vrací Object, který buď může být textový řetězec s daty ze serveru nebo vzniklá výjimka. O úroveň výše tento fakt kontroluje a v případě, že se jedná o výjimku, je vyhozena a zpropagována až k vývojáři. 
Po vytvoření objektu RequestMaker je třeba ho spustit. K tomu slouží metoda executeOnExecutor, která umožňuje spustit v jednu dobu více objektů typu AsyncTask.

WindowPhone verze používá k tvorbě a manipulaci s požadavkem třídu HttpClient \cite{wp-httpclient} v rámci asynchronní metody, která request provádí. V tomto případě problém s propagací případné výjimky až k vývojáři nenastává. Nastavení požadavku se provádí přes konstruktor obdobně jako v Android verzi. 

\subsection{Generování komponent}
Na základě definice komponenty, která se ze serveru získá výše uvedeným způsobem, lze generovat komponenty. Typ komponenty, která se vygeneruje, závisí na zvoleném builderu. Každý builder je reprezentován třídou, která dědí od abstraktní třídy AFComponentBuilder. Tato třída definuje společné vlastnosti všech builderů a vynucuje implementaci metody pro vytvoření komponenty jako takové a vytvoření její grafické reprezentace. Proces vytvoření komponenty v Android frameworku je popsán na sekvenčním diagramu v příloze \ref{img:sdFormBuilding}. Proces tvorby ve WP verzi je obdobný s tím rozdílem, že není potřeba znát k vytváření GUI prvků kontext, proto nemusí být do argumetů funkcí předávána aktivita, ve které chceme komponentu vytvořit. Z obrázku \ref{img:sdFormBuilding} je patrné, že po inicializaci builderu a případném nastavení skinu, začíná proces tvorby formuláře, na jehož konci uživatel získá již hotový formulář. Nejprve dojde k získání metamodelu komponenty ze serveru. Poté začne tvorba jednotlivých polí formuláře, k tomu je zapotřebí odpověď s metamodelem rozparsovat, výsledkem čehož je třída AFClassInfo obsahující informace o celé komponentě. Tato struktura byla v analytické části popsána na obrázku \ref{img:metadataModel}, skutečná struktura v tomto případě pro Windows Phone framework je na obrázku \ref{img:metadataClass}. AFClassInfo obsahuje 0 až N objektů třídy FieldInfo, tedy objektu držící informace o tvořeném poli. Přes tyto objekty framework iteruje a postupně z nich vytváří objekty třídy AFField, které už obsahují i grafickou reprezentaci pole. 

Při iterování skrze popisy polí je důležité kontrolovat atribut, který určuje, zda pole zastupuje v modelu na serveru primitivní nebo složený datový typ. V případě, že se jedná o složený datový typ, vyhledá se k němu v AFClassInfo příslušná definice vnitřní třídy, na kterou se funkce, která tvoří jednotlivá pole, rekurzivně zavolá a na místo pole označeného jako vnitřní třída budou vytvořena pole, která jsou nadefinovaná v příslušné vnitřní třídě. 

Součástí tvorby pole není jen nadefinování vlastností a tvorba samotné grafické reprezentace. Při tvorbě formulářových polí je potřeba vytvořit tři elementy, ze kterých se pole skládá - label, místo pro zobrazení validačních chyb a aktivní prvek. To zajišťuje třída FieldBuilder. Vytvořit label a místo pro validační chyby je jednoduché, aktivní prvek je mírně komplikovanější, neboť se nejedná jen o textový popisek. Součástí třídy FieldInfo je i atribut widgetType, který určuje, jakým typem aktivního prvku má být pole reprezentováno. Na základě tohoto atributu se získává pomocí třídy WidgetBuilderFactory příslušný builder, který umí přímo příslušný aktivní prvek vytvořit. Tento builder poskytuje také funkcionalitu pro získání a nastavení dat vytvořenému aktivnímu prvku. Pro jednoduší přístup k této funkcionalitě má pole formuláře na builder, který mu vytvořil část s aktivním prvkem, referenci. Po vytvoření všech tří částí se zkompletují do celkové grafické reprezentace, která se poli taktéž nastaví. Takovéto pole se vrátí builderu komponenty, který si jeho existenci zaznamená. Po zaznamenání všech polí, které mají ve formuláři být, se jednotlivé grafické reprezentace polí seskupují do celkového vzhledu komponenty, který se komponentě nastaví a vývojář ho může z komponenty získat.

\subsubsection{Naplnění komponety daty}
Pokud má být komponenta naplněna daty, je nutné získat jednotlivé buildery, které vytvořily jenotlivá pole. Data se vkládají přímo z odpovědi serveru, která je obsahuje. V datech se objevuje vždy identifikátor daného pole a hodnota, která má být vložena. Dle tohoto identifikátoru je framework shopen vyhledat příslušné pole a z něj získat builder, který postavil jeho aktivní část. Jak už bylo řečeno, builder disponuje funkcí schopnou do pole, které vytvořil, vložit data, což také provede. Takto se to provede u všech polí, která se v přijatých datech vyskytují. Tím je proces tvorby formuláře popsaného na obrázku \ref{img:sdFormBuilding} ukončen a vývojáři je předána vytvořená komponenta. 

\subsubsection{Widget buildery}
Bylo zmíněno, že části formulářových polí, do kterých lze zadat uživatelský vstup, se vytváří pomocí příslušných builderů, jež lze získat pomocí třídy WidgetBuilderFactory. Tato třída využívá návrhový vzor Factory, který je vhodné použít, když je potřeba vytvořit objekt a zároveň zastínit logiku vytváření objektu klientovi \cite{factorypattern}. Tento návrhový vzor byl ve frameworku implementován proto, aby se v případě, že by bylo potřeba přidat nový builder, nezasahovalo do logiky vytváření polí, ale pouze do třídy WidgetBuilderFactory. Jednotlivé widget buildery lze nalézt na obrázku \ref{img:factoryWidgetBuilders} a jejich seznam je v tabulce \ref{table:widgetBuilders}. Každý builder dědí od základního builderu definovaného abstraktní třídou BasicBuilder implementující rozhraní AbstractWidgetBuilder. AbstractWidgetBuilder definuje metody pro vytvoření grafické reprezentace widgetu, nastavení dat do widgetu a zisk dat z widgetu. Tyto metody musí být každým novým builderem implementovány. Ukázka kódu \ref{code:createTextField} zobrazuje tvorbu grafické reprezentace textového pole v Android frameworku.

\begin{lstlisting}[caption=Ukázka tvorby grafické reprezentace textového pole,
label={code:createTextField}, basicstyle=\footnotesize]
@Override
public View buildFieldView(Activity activity) {
    EditText text = new EditText(activity);
    text.setTextColor(getSkin().getFieldColor());
    text.setTypeface(getSkin().getFieldFont());
    addInputType(text, getField().getFieldInfo().getWidgetType());
    if(getField().getFieldInfo().getReadOnly()){
	text.setInputType(InputType.TYPE_NULL);
        text.setTextColor(Color.LTGRAY);
    }
    return text;
}
\end{lstlisting} 

Z ukázky \ref{code:createTextField} je patrné, že na vstupní pole je aplikován skin, který upravuje jeho vzhled. Konkrétně v případě textového pole se nastavuje barva jeho textu a typ písma.

\begin{table}[h!]
\begin{center}
\caption{Podporované widget buildery}
\label{table:widgetBuilders}
\begin{tabular}{|p{4cm}|p{3cm}|p{7cm}|}
\hline
\textbf{Builder} & \textbf{Typ widgetu} & \textbf{Popis} \\
\hline
DateWidgetBuilder & 
Calendar & Používá se při reprezentaci atributů typu datum. Umožní uživateli zobrazit datepicker, pomocí kterého lze vybrat datum. \\
\hline
DropDownWidgetBuilder &
dropDownMenu & Menu, ze kterého lze vybrat jednu z několika možností. \\
\hline
CheckBoxWidgetBuilder & checkBox &
Zaškrtávací políčko reprezentující hodnotu true nebo false podle toho, zda je nebo není zaškrtnuto. \\
\hline
TextWidgetBuilder & textField &
Builder pro textové pole. Lze mu nastavit typ vstupu, podle kterého se změní klávesnice pro zadávání znaků například na číselnou, tedy používá se i pro widget typy NUMBERFIELD a NUMBERDOUBLEFIELD. V Android frameworku lze použít i pro pole s heslem. \\
\hline
PasswordWidgetBuilder & password &
Pole pro hesla. Text, který uživatel dovnitř napíše, je převeden na zástupné znaky. Tento builder se vyskytuje pouze ve WP verzi frameworku, Android verze k tomuto účelu používá TextWidgetBuilder. \\
\hline
OptionBuilder & option &
Vytvoří skupinu radio buttonů, ze které lze vybrat jednu hodnotu. Pokud nejsou definovány možnosti, vytvoří se dva radio buttony s hodnotami ano a ne. \\
\hline
\end{tabular}
\end{center}
\end{table}

\subsubsection{Skiny}
Skiny neupravují pouze vzhled aktivního prvku, ale také vzhled labelu, validačních chyb nebo i celé komponenty. Lze nastavit grafické vlastnosti jako je barva textu, pozadí, velikost textu a jeho font, ale také rozměry celé komponenty. Ve frameworcích existuje rozhraní Skin, které definuje všechny nastavitelné vlastnosti. Už bylo zmíněno, že skiny se musí nastavit před tvořením komponenty jejímu builderu. Toto nastavení je ale volitelné, a tak existuje třída DefaultSkin, která toto rozhraní implementuje a definuje základní vzhled, který se použije nespecifikuje-li vývojář jinak. Vývojář pravděpodobně nebude chtít předělat úplně celý vzhled, ale pouze části skinu. K tomuto mu stačí dědit od třídy DefaultSkin a pouze přetížit metody udávající vlastnosti, které chce změnit. 

\subsubsection{Layouty}
Při vytváření grafické reprezentace komponenty, server definuje, jak se její jednotlivé části v rámci komponenty uspořádají. V případě formuláře je nutné uspořádat pole tak, aby byl formulář přehledný a dobře použitelný. V případě listu jde hlavně o přehlednost, neboť jednotlivé položky listu by byly při větším obsahu moc vysoké a nevzhledné. Server definuje osu, podle které jsou komponenty vykreslovány, počet sloupců a pozici labelu \cite{tomasek-thesis}. Layout lze určit buď celé komponentě, ten je na obrázku \ref{img:metadataClass} reprezentován třídou TopLevelLayout nebo pouze části komponenty, tj. poli, který je na obrázku reprezentován třídou Layout. V případě layoutu komponenty je pouze definována osa a počet sloupců, layout jednotlivých polí přidává pozici labelu. Hodnoty, kterých jednotlivé vlastnosti nabývají, jsou popsány pomocí následujících výčtových typů, které lze nálezt i na obrázku \ref{img:metadataClass}.
\begin{itemize}
\item LayoutDefinitions, který určuje počet sloupců a nabývá hodnot ONECOLUMNLAYOUT a TWOCOLUMNSLAYOUT, tedy jednosloupcové a dvousloupcové rozložení \cite{tomasek-thesis}.
\item LayoutOrientation, který určuje osu, ve směru které se části komponenty vykreslují. Nabývá hodnot AXISX a AXISY. První určuje směr vykreslování podle osy X, druhá podle osy Y \cite{tomasek-thesis}.
\item LabelPosition, který určuje pozici labelu vzhledem k aktivnímu prvku. Nabývá hodnot BEFORE, AFTER a NONE. První znamená, že bude label před aktivním prvkem, druhá za aktivním prvkem a třetí znamená, že label nebude vůbec zobrazen \cite{tomasek-thesis}.
\end{itemize}

Layouty jsou v jednotlivých klientských frameworcích interpretovány různě. Zatímco v případě Android frameworku je layout realizován pomocí seskupení LinearLayoutů \cite{android-lin-layout}, WP verze používá k tvorbě layoutu Grid \cite{wp-grid}. Důvodem není nic zvláštního, k oběma třídám existují na druhé platformě ekvivalenty, pomocí kterých lze dosáhnout stejných výsledků, jde spíše o to, ukázat více možných způsobů implementace. 

Důležitým aspektem je také pořadí jednotlivých částí komponenty. To definuje server a v metadatech se informace o polích objeví už v daném pořadí, které je třeba zachovat. Proto bylo nutné použít k uskladnění kolekci, která zachovává pořadí svých prvků, takovou je například List \cite{tomasek-thesis}.

\subsubsection{Lokalizace}
Jelikož se mohou v metadatech objevit i texty pro překlad, bylo třeba naimplementovat třídu Localization, která se stará o překlady těchto textů. Třída disponuje samotnou funkcí pro překlad a funkcí pro změnu jazyku. V Androidu se typicky lokalizační texty umisťují do souboru strings.xml ve složce values. Pro Windows Phone je to obdobné. Zde se texty vkládají do souborů Resources.resw, které jsou umístěny ve složce s názvem jazyku ve složce Strings. S tímto umístěním textů oba frameworky počítají. U Android verze je možné nadefinovat jiný balíček, ve kterém by se texty měly hledat, ale stále se předpokládá umístění ve values/strings.xml. Pokud není překlad nalezen, je vrácena hodnota klíče, který se hledal. Třída Localization se dá využít i pro překlad jiných textů i mimo framework, a to i v rámci XML šablon, ve kterých je vytvořeno statické UI. Při startu aplikace se použije jazyk zařízení. Pokud by pro daný jazyk neexistovaly předklady, je vhodné mít defaultní soubory s texty, které se v tomto případě použijí.

\subsection{Práce s komponentami}
Vytvořená komponenta disponuje různými funkcemi, které umožňují s ní dále pracovat. Pro formulář jsou to metody pro odeslání, validaci, resetování nebo vyčištění formuláře. List je komponenta, která by měla pouze zobrazovat data, proto má omezenou množinu funkcionalit. Obsahuje však důležitou funkci, pomocí které lze získat obsah jedné položky listu, který lze využít například pro naplnění formuláře. 

\subsubsection{Odeslání, reset a vyčištění formuláře}
Nejdůležitější funkcí je však odeslání formuláře, které je zobrazeno na sekvenčním diagramu \ref{img:sdFormSend}. K tomuto účelu má formulář metodu sendData. V rámci této metody se nejdříve zjistí, zda vývojář pro komponentu nadefinoval v XML souboru s připojeními část označenou uzlem <send>, který definuje, kam se budou data odesílat a jehož struktura je zobrazena v ukázce kódu \ref{code:xmlSource}. Komponenty mají atribut connectionPack typu AFSwinxConnectionPack, který obsahuje všechna definována připojení. V případě, že vývojář specifikoval uzel <send>, se v tomto atributu formuláře objeví i připojení na zdroj, který přijímá uživatelský vstup. 

Pokud připojení nadefinované není, vyhodí se výjimka, která je propagována k vývojáři, který s ní dle svého uvážení naloží. Pokud připojení existuje, vygenerují se z formuláře data ve formátu, které server dokáže zpracovat. Klient nezná přesně objekty, které se odesláním na serveru vytvoří či upraví, zná ale jejich strukturu, která stačí k tomu, aby byla vytvořena data, která server dokáže přijmout. V rámci generování dat proběhne nejdříve jejich validace, která je blíže popsána v sekci níže. Pokud data nejsou validní, vrací metoda sendData hodnotu false, která určuje, že se odesílání nezdařilo. Samotný proces generování dat byl převzat z frameworku AFSwinx. Ten využívá pro znovupostavení dat objekt typu BaseRestBuilder disponující metodou reserialize \cite{tomasek-thesis}. 

BaseRestBuilder je implementován třídou JSONBuilder, která dokáže z formuláře vytvořit JSON soubor a použije se v případě, že server požaduje JSON formát. Další formáty nejsou prozatím podporovány. Metoda reserialize, kterou builder dat používá, již očekává vyparsovaná data z komponenty uložená v objektu typu AFDataHolder. Vytvoření těchto dat provádí komponenta sama. Uživatelský vstup se získává z aktivní prvků, ke kterým má skrz kolekci polí typu AFField přístup. Data se získávají z aktivního prvku pomocí widget builderu, který aktivní prvek vytvořil a na který má AFField referenci. Každý AFField má svůj unikátní identifikátor, který lze využít pro zjištění pozice proměnné v objektu na serveru, který má odeslání formuláře ovlivnit. Pokud indentifikátor obsahuje tečku, znamená to, že dané pole je částí reprezentace neprimitivního datového typu, tedy vnitřní třídy a k té je třeba hodnotu proměnné přidat. Pokud v identifikátoru tečka není, znamená to, že jsme na správném místě a do mapy AFDataHolderu se přidá nová proměnná s hodnotou \cite{tomasek-thesis}. Android verze využívá již naimplementovaný JSONBuilder, který využívá k sestavení dat framework GSON \cite{gson}.Ve Windows Phone verzi musel být builder dat přepsán a využívá Windows knihovnu pro práci s JSON soubory. Po vytvoření dat stačí data odeslat na definovaný zdroj na serveru pomocí třídy RequestMaker, která byla popsána výše. Jak už bylo zmíněno, v rámci procesu odeslání může dojít k výjimkám, které jsou předány vývojáři ke zpracování. 

Pro odeslání formuláře tedy stačí definovat zdroj v XML souboru a poté zavolat metodu sendData, například jako reakci na stiknutí tlačítka. Dále je nutné, ošetřit výjimky, které mohou při odesílání nastat, aby nedošo k pádu aplikace. V případě, že nedojde k žádné výjimce, data jsou validní a jejich vygenerování z formuláře bylo úspěšné, může vývojář navázat na odeslání další akci, například oznámení o úspěšném odeslání formuláře ve formě dialogu. Vývojáři je tedy celý proces odeslání schován a nemusí se jím vůbec zabývat, díky čemuž je použití frameworku snadné. Nevýhodou však je, že je vývojář omezen jen na použití nabízených funkcí frameworku a nemůže do procesu odeslání nijak zasahovat \cite{tomasek-thesis}. 

Dalšími funkcemi, které formulář nabízí, je resetování a vyčištění formuláře. Reset formuláře provede obnovení aktuálních dat ve formuláři. Pokud je tedy formulář naplněn daty a uživatel data změní, lze se k původním nezměneným datům před odesláním formuláře vrátit. Vyčištění formuláře nastaví aktivní prvky formuláře na prázdné.

\subsubsection{Validace a Validátory}
Formulář se před odesláním dat na server validuje. Pro každé pole ve formuláři může být nadefinováno 0 až N validačních pravidel, které vznikají na základě datových nebo bussiness omezení. Hlavním účelem validací je zajistit, aby byla tato omezení byla splněna a nedošlo při odeslání k chybě. Chyba by například nastala, pokud server očekává pouze celočíselnou hodnotu a přijde mu místo ní textový řetězec. Na Android platformě existuje vůči části takových chyb určitá prevence. Příkladem je pole, které vzniklo z atributu typu integer, jež má být reprezentováno typem widgetu Numberfield. V takovém případě framework použije widget Textfield a určí, že lze zadat pouze čísla. Tím se změní nabízená klávesnice, na které uživatel může zadat opravdu jen čísla od 0 do 9 a navíc do pole nelze zkropírovat nic jiného než čísla. WP také změní typ klávesnice, už však dovoluje do pole zkopírovat jiné znaky a nemá jako Android různé klávesnice pro celé a desetinné číslo, proto je třeba vstup z těchto polí dodatečně validovat.

Ke každému pravidlu existuje validátor, který umí pravidlo prověřit. Tabulka \ref{table:validations} zobrazuje seznam podporovaných validátorů, ke kterým pravidlům patří a jejich stručný popis. Proces validování formuláře byl již zobrazen na obrázku \ref{img:sdFormSend}. Nejprve se projdou všechna pole formuláře, což jsou objekty typu AFField. AFField obsahuje pravidla, což jsou objekty typu AFValidationRule, které je pro každé pole nutno proiterovat, vyhledat k nim příslušný validátor a provést validaci. Zisk příslušného validátoru probíhá pomocí tovární třídy ValidatorFactory, podobně jako zisk konktrétního widget builderu, který byl popsán dříve. Validátory implementují rozhraní AFValidator, které disponuje metodou validate, která vrací boolean určující, zda validace proběhla úspěšně či ne. Této metodě je předáván StringBuilder, obsahující chybové hlášky pro dané pole, tedy každé pole má svůj StringBuilder. Pokud alespoň jedna z validací selže, vytvoří se label, do kterého se nastaví hlášky ze StringBuilderu, a tento label se uživateli zobrazí. Vývojář má možnost upravit chybové hlášky, které se zobrazují uživateli. V tabulce \ref{table:validations} lze v popisu nalézt klíče jednotlivých chybových hlášek, které je potřeba vývojářem přeložit.

\begin{table}[h!]
\begin{center}
\caption{Podporované validace}
\label{table:validations}
\begin{tabular}{|p{3.5cm}|p{4.5cm}|p{7cm}|}
\hline
\textbf{Validátor} & \textbf{Pravidlo} & \textbf{Popis} \\
\hline
RequiredValidator & 
REQUIRED & Použije se, pokud má být pole povinné. Pokud je pole prázdné, vypisuje hlášku, která vzniká přeložením klíče validation.required. \\
\hline
MaxValueValidator &
MAX & Používá se na číselná pole, která mají svou hodnotu omezenou shora číslem. Pokud hodnota překročí definované číslo, vypisuje hlášku, která vzniká přeložením klíče validation.maxval. \\
\hline
MinValueValidator & MIN &
Používá se na číselná pole, která mají svou hodnotu omezenou zdola číslem. Pokud je hodnota menší než definované číslo, vypisuje hlášku, která vzniká přeložením klíče validation.minval. \\
\hline
MaxCharsValidator & MAXLENGTH &
Použije se, pokud počet znaků v poli překročí danou hodnotu. Pokud dojde k překročení, vypisuje hlášku, která vzniká přeložením klíče validation.maxchars.\\
\hline
NumberValidator & Pole vzniklo z atributu typu integer nebo double &
Kontroluje, zda jsou vloženy celočíselné hodnoty pro integer nebo desetinná čísla pro double. V případě, že má být v poli celé číslo a je tam jiný vstup, vypisuje se hláška, která vzniká přeložením klíče validation.integer. V případě double je to klíč validation.double. 
Tento validátor bylo nutné použít pouze ve WP frameworku, neboť Android disponuje prevencí, která byla popsána výše.\\
\hline
LessThanValidator & LESSTHAN &
Porovnává dvě pole obsahující datum. Pokud v poli, které obsahuje toto pravidlo není menší hodnota než v poli, jehož identifikátor je definován hodnotou tohoto pravidla, vypisuje se hláška, která vzniká přeložením klíče validation.lessthan.  \\
\hline
\end{tabular}
\end{center}
\end{table}

\section{Úprava AFSwinx a AFRest}
V rámci úpravy existujících frameworků AFSwinx a AFRest \cite{tomasek-thesis} bylo požadováno umožnit porovnání dvou polí obsahující datum ve smyslu menší rovno než, které se použije v případě, že mají v aplikaci existovat dvě pole s daty, z nichž jedno může obsahovat pouze dřívější nebo stejné datum než datum v druhém poli. Proto bylo vytvořeno validační pravidlo LESSTHAN, které lze nalézt v tabulce \ref{table:validations}. K vytvoření tohoto pravidla bylo nutné upravit nejen zmíněné frameworky, ale také serverovou část aplikace. Pravidlo bylo realizováno vytvořením nové anotace @UILessThan, kterou lze přidat k atributům entity typu datum na serveru. V anotaci je nutno definovat identifikátor druhého pole, se kterým se má pole označené anotací porovnávat. K vytvoření anotace bylo dle dokumentace AspectFaces \cite{aspect-faces} potřeba vytvořit v AFRest nový anotační deskriptor a zaregistrovat ho v konfiguračním souboru aspectfaces-config.xml na serveru. Tím se zajistí, že inspekce anotaci promítne do metadat ve formě XML. Aby se však anotace projevila i v metadatech, která přichází ze serveru, navíc v části s validačními pravidly, bylo nutné upravit i šablonu date.xml, která se používá při generování metadat o atributu typu datum. Upravená šablona je v ukázce kódu \ref{code:dateXml}.

Další úprava se týkala výčtového typu frameworku AFRest, který obsahuje podporované validace. K validaci bylo nutné vytvořit validátory ve všech existujících klientských frameworcích, tedy nejen v Android a WP verzi, ale také v AFSwinx, jež interpretuje UI pro Swing aplikace \cite{tomasek-thesis}. Nakonec byla anotace umístěna k atributům v serverových entitách, kterých se to týkalo. Příklad použití je v úkázce kódu \ref{code:uiLessThanUsage}.

\section{Porovnání přístupů}
V Android a WindowsPhone aplikacích lze GUI vytvářet buď za pomoci kódu v příslušném programovacím jazyce nebo staticky za pomoci XML šablon. V obou případech platí, že musí vývojář každý prvek GUI nadefinovat, naplnit daty a pokud má být na prvek navázána akce, musí naimplementovat i ji. V případě formuláře tedy musí obvykle vytvořit dané aktivní prvky s popisky, element pro zobrazení validačních chyb a tlačítko pro odeslání formuláře. K tomu musí naprogramovat logiku validací, získání dat z formuláře a pokud má být formulář předvyplněn daty, tak i způsob jeho naplnění. Jesliže jsou navíc data získávána ze serveru, je nutné naimplementovat připojení na server, jak se data získají a reprezentují a způsob jejich odeslání na server. To vše znamená pro vývojáře spoustu kódu, který musí napsat, zvlášť když se komponenta objevuje na více místech v aplikaci. 

Pokud jsou komponenty vytvářeny frameworkem, většina částí, které musí klasicky programátor řešit, za něj provádí framework. Při tvorbě komponent stačí vývojáři nadefinovat zdroje, ze kterých se má formulář vytvořit, naplnit daty nebo kam se má odeslat a pomocí frameworku komponentu vytvořit a vložit do GUI. Za pomoci frameworku se vygenerují aktivní prvky i s jejich popisky podle toho, co udává server. Server také určí validace, k nimž framework vytvoří validátory, a řeší i zobrazení případných validačních chyb. V případě formuláře definuje, jak se z něj získají data a jak se mají odeslat na server. Připojení na server také není potřeba implementovat. 

To vše vede k výrazné redukci kódu, což je nepochybně výhodou. Další výhodou je, že je GUI definováno serverem a reaguje tedy automaticky na změny v modelu na serveru. Nevýhodou je, že je vývojář omezen na funkce, které framework nabízí, a k využití frameworku je požadován server, který poskytuje příslušné definice komponent. Tedy při nasazení na existující aplikaci nestačí jen použít framework na straně klienta, ale je nutné výrazně upravit serverovou část. V případě klasického přístupu není nutné při tvorbě klienta server měnit a vývojář klienta pouze serveru přizpůsobuje. Také je nespornou výhodou, že má vývojář nad komponentami plnou kontrolu a daleko větší možnosti, jak s ní naložit, například co se úpravy vzhledu týče.
  
\chapter{Testování}
Testování je nedílnou součástí vývoje každé aplikace. Testuje se jednak z hlediska funkčnosti a splnění požadavků, ale také z hlediska použitelnosti uživateli. Testy tedy dělíme na dvě kategorie - testy bez uživatele, neboli automatické a testy s uživatelem, tedy manuální.  Automatické testování může mít různý charakter, lze testovat funkčnost jednotlivých částí systemu, například pomocí unit testů, ale také například kvalitu kódu pomoci jeho statické analýzy. Testy s uživatelem hlavně řeší použitelnost programu, tedy zda uživatel dokáže se systémem bez problému manipulovat. Tyto testy se často provádí prostřednictvím osobního sezení, kdy uživatel dostane sadu úkolů, které má splnit a testující zapisuje, co dělalo uživateli problémy. Aby byl test smysluplný, určí se nejdříve cílová skupina, na kterou se má software zaměřovat. Určením cílové skupiny se eliminuji případy, kdy uživatel nemá dostatečné předpoklady pro užití softwaru. Pro zjištění, zda testovaný člověk do této skupiny spadá, slouží pretest dotazníky v různých formách. Bývá zvykem probrat s uživatelem po testu, například formou dotazníku, jak se mu se systémem pracovalo, jaké měl problémy, výhody a nevýhody systému, aby testující získal co nejvíce informací.  

Frameworky v této práci byly vytvořeny v první řadě jako proof of concept, což znamená, že se nejedná o verzi, ktera byla určena k vydání. Z tohoto důvodu se na testování nekladl takový důraz, jako by se kladl v případě verze připravené pro vydání. Bylo provedeno pak ukázkových unit testů a test s uživateli na použití frameworku z hlediska vývojáře. Také byl proveden jednoduchý kvantitativní test, který zkoumá rychlost frameworku v závislosti na složitosti vytvářené komponenty.

\section{Unit testy} 
Unit testy dostaly název podle toho, že testují co nejmenší části, jednotky, zdrojového kódu aplikace. Unit testy by měly testovat jednotlivé metody systému a komunikaci mezi nimi, neměly by testovat celý systém. Ideálně by každý test měl být nezávislý na ostatních testech. Výhodou unit testů je, že odhalí chyby relativně brzo v procesu vývoje aplikace. V některých případech jsou unit testy vytvářenxy předtím, než vývojář začně vůbec funkcionalitu implementovat. Při vytváření testů je nutné zvážit, zda se test vůbec oplatí dělat. Například se nebude testovat jednoduchá metoda na sčítání nebo třeba gettery a settery.

Pro testování Android frameworku byl využit testovací framework JUnit \cite{junit} a pro WP verzi byly použity testovací nástoje Visual Studia \cite{vs-unit}. Oba nástroje disponují množstvím různých porovnávání, lze určit chování testu a dalšími způsoby test přizpůsobit. V JUnit k tomuto slouží anotace, kterými lze například určit, co se vykoná před každým spuštěním testu nebo po jeho dokončení. Lze spustit jeden nebo více unit testů najednou. Vývojová prostředí většinou disponují nějakým nástrojem, který přehledně ukáže výsledky testů a případné chyby. 

V Android frameworku byly provedeny tyto testy:
\begin{itemize}
\item Test JSON parseru, který parsuje metadata přicházející ze serveru. Testuje se, zda jsou správně rozparsovány a uloženy všechny vlastnosti. Test je proveden na dvou řetězcích popisující metadata. Jeden z nich je v počádku a druhý poškozen.
\item Test utility, která parsuje z textové reprezentace datum. Metoda umí parsovat ze dvou formátů. Testují se převedení pomocí obou formátů a je také proveden test na neexistující formát, při kterém by metoda měla vrátit null.
\item Test správného vytvoření adresy z definovaného připojení. Testuje se případ s portem a bez portu.
\end{itemize} 
Ve Windows Phone verzi byly provedeny následující testy:
\begin{itemize}
\item Stejné testy jako v případě Androidu.
\item Test nahrazení proměnných v XML souboru se zdroji hodnotami z Dictionary. Tyto hodnoty mohou ovlivňovat, jak je zdroj nadefinován a tedy i proces připojení na něj. Tento test hlavně testuje správné přepsání z Javy do C\#, protože Java metoda již byla otestována \cite{tomasek-thesis}.
\item Test na získání výčtového typu podle jeho hodnoty. V C\# nejsou totiž stejné výštové typy jako v Javě a nelze jim jednoduše nadefinovat nějakou hodnotu. Výčtové typy ve WP verzi se snaží přiblížit chování Java enumů a s tím souvisí i metoda valueOf, která by na základě hodnoty měla vrátit příslušný výčtový typ. Tento test ověřuje její správnost.
\end{itemize} 

\section{Test s uživatelem}
Při návrhu frameworků bylo potřeba zaměřit se hlavně na potřeby vývojářů mobilních aplikací. Každý takový vývojář je už zvyklý na určité možnosti a vlastnosti, které při vývoji používá. Například by měl framework umožnit nastavovat vzhled komponent. Při návrhu testu jsem čerpal z předmětu Testování uživatelských rozhraní, kde jsem podobný test navrhoval v rámci semestrální úlohy. Bylo třeba určit cílovou skupinu a navrhnout úkoly pro test. Také bylo žádoucí získat od partipantů další informace, například jaký mají z frameworku dojem.

\subsection{Cílová skupina}
Cílovou skupinou pro tento test jsou výjojáři Android nebo Windows Phone aplikací. Účastníci testu by měly vědět jak vytvořit nějaký layout, do kterého by mohly frameworkem vytvářené komponenty vkládat, umět si vytvořit tlačítko, na které naváží například odeslání formuláře. Zkušenosti vývojáře se standartním vývojem mobilních aplikací jsou důležité i proto, aby byl schopen porovnat přístup frameworku s klasickým způsobem. Potencionální účastníci testu byli vybírání z mého ročníku a byli kontaktováni osobně či skrz sociální sítě. Bylo osloveno celkem deset lidí, o kterých jsem předpokládal potřebné zkušenosti, z nichž tři lidé požadavky nesplňovali a dva lidé test odmítli. Ze zbývajících pěti lidí uměli tři Android a dva Windows Phone. Pro vyváženost testu byl jeden participant, který uměl Android odmítnut.    

\subsection{Příprava testu}
Pro test byla vytvořena sada úkolů a uživatelská příručka. Aby nebyl test příliš časově náročný, byla příručka rozeslána participantům předem s tím, že mohou mít ke všemu dotazy. Participanti měli vytvořit novou klientskou aplikaci nebo použít existující, do které si sami lokálně naimportují příslušný framework. K tomu bylo nutné jim framework poskytnout. Participanti pro test nepotřebovali v podstatě nic. Vlastní telefon s Androidem nebo WindowsPhonem a naisntalované IDE pro vývoj bylo výhodou. Vyvíjet však mohli i na mém notebooku, kde bylo pro participanty připraveno VisualStudio 2015 a AndroidStudio 1.5.1. K odzkoušení aplikací byly připraveny smartphony s Androidem 4.3 nebo Windows Phone 8.1. V případě Androidu bylo možno využít také emulátor obsahující Android 6.0. K testu byla potřeba serverová část
 
\subsection{Testované úkoly}
Testované úkoly byly navrženy tak, aby pokryly co největší množství frameworkem poskytované funkcionality a pokud možno na sebe navazovaly. V úvahu byla vzata i náročnost úkolů. Po dokončení většiny úkolů se předpokládalo, že si participant aplikaci nahraje do zařízení a odzkouší ji. Pokud tak uživatel neučinil, bylo mu to připomenuto. K programování mohl testovaný používat přiloženou uživatelskou příručku. Časová náročnost testu bylo odhadnuta na zhruba jednu hodinu a patnáct minut. Uživateli byly testované úkoly předány ve formě PDF souboru v následujícím pořadí.

\begin{enumerate}
\item Před testem prostudujte uživatelskou příručku, pokud jste již tak neučinili.
\item Založte si novou aplikaci nebo si otevřete existující a importujte do ní framework.
\item Vytvořte si XML soubor pro definici připojení. V něm vytvořte dvě připojení pro formulář a list, které si libovolně pojmenujte. 
\begin{enumerate}
\item Formulář
\begin{enumerate}
\item Definice komponenty je na adrese \url{http://toms-cz.com/AFServer/rest/country/definition}. Zdroj je veřejný a poskytuje data ve formátu JSON.
\item Formulář lze odeslat na adresu \url{http://toms-cz.com/AFServer/rest/country}. Zdroj je zabezpečený, může k němu přistupovat jen přihlášený uživatel. Server očekává data ve formátu JSON.
\end{enumerate}
\item List
\begin{enumerate}
\item Definice komponenty je na adrese \url{http://toms-cz.com/AFServer/rest/country/definition}. Zdroj je veřejný a poskytuje data ve formátu JSON.
\item Data, kterými lze list naplnit jsou na adrese \url{http://toms-cz.com/AFServer/rest/country/list}.  Zdroj je veřejný a poskytuje data ve formátu JSON.
\end{enumerate}
\end{enumerate}
\item Vytvořte prázdný formulář přidávající/upravující zemi a libovolně ho vložte do aplikace. Komponentu vhodně pojmenujte.
\item Vytvořte list naplněný existujícími zeměmi. Komponentu vhodně pojmenujte.
\item Zajistěte jazykovou srozumitelnost komponent (tj. v češtině nebo angličtině).
\item Zprovozněte odesílání formuláře po stisknutí nějakého vámi vytvořeného tlačítko.
\begin{enumerate}
\item Posluchač události vytvořte nejlépe ve zvláštní metodě.
\item Po odeslání informujte uživatele o úspěchu/neúspěchu akce.
\item Volitelně zprovozněte reset a clear formuláře.
\end{enumerate}
\item Upravte libovolně vzhled formuláře (např. změňte barvu labelů)
\item Upravte libovolně vzhled listu (například upravte border listu)
\item Propojte list s formulářem tak, že po kliknutí na položku listu se data nahrají do formuláře
\end{enumerate}
\subsection{Výsledky testu}
 TODO dotestovat lidi a pak udelat nejaky zaver

\section{Kvantitativni test} 
Tento test ukazuje závislost doby vytvoření komponenty na počtu částí, ze kterých se skládá a na rychlosti připojení k internetu. Test byl proveden na obou platformách. Bohužel nemůžeme platformy jednoduše mezi sebou porovnat, neboť zařízení, na kterých test proběhl, nemají stejné parametry, které taky rychlost vytváření ovlivňují. Nicméně bylo provedeno měření pro dva formuláře, jeden z nich se skládal ze dvou polí a nebyl naplněn daty, druhý měl polí čtrnáct a daty naplněn byl. Potom byl test proveden pro listy, kde bylo sledováno zda rychlost vytvoření komponenty ovlivňuje i množství dat, kterými se komponenta naplní. Jedna položka listu obsahovala u obou listů stejný počet informací, v obou listech byl ale rozdílný počet záznamů. Pro každou komponentu bylo provedeno 10 měření, ze kterých se pak udělal průměr. 

Test na Android platformě ukázal, že na počtu polí ve formuláři i na rychlosti připojení záleží. Zatímco menší formulář byl na Wi-Fi v průměru vytvořen za 684 milisekund, větší formulář v průměru za 1084 milisekund a potom se ještě 645 ms nalňovat daty. Na mobilních datech pak vytvoření menšího formuláře trvalo téměř dvakrát déle, tedy 1110 milisekund, vytvoření většího trvalo přibližně 2035 milisekund bez naplnění daty a 2990 milisekund s vloženými daty. Test tvorby listu zase ukázal, že na množství položek v listu také záleží. List s jednou položkou se vytvořil a naplnil daty na Wi-Fi v průměru za 1450 milisekund a listu s osmi záznamy to trvalo v průměru pouze od 20 milisekund déle. Při slabším připojení byl rozdíl už markantnější, činil v průměru asi 350 milisekund.

Windows phone platforma měla hodnoty samozřejmě jiné, podstatné však je, že se chovala obdobně jako Android. Ukázalo se tedy, že záleží jednak na rychlosti připojení k internetu, počtu částí, ze kterých je komponenta vytvořena a i na množství dat, kterými je komponenta naplněna.

\section{Ukázkové projekty}
Pro otestování frameworků v praxi byly vytvořeny dva ukázkové projekty, pro každou platformu jeden. Vše, co bylo potřeba pro vytvoření klientské aplikace, byl příslušný framework a již naimplementovaná serverová strana, která byla využita i pro demonstraci AFSwinx a AFRest \cite{tomasek-thesis}. Aplikace na serveru slouží jako systém k zadávání absencí, která disponuje zdroji, ze kterých lze získat definici UI a data. Také poskytuje zdroje pro odesílání formulářů, které vytváří nebo upravují záznamy v databázi. Některé zdroje jsou zabezpečené a může k nim přistupovat jen uživatel s určitou rolí. Z tohoto důvodu musí v klientských aplikacích existovat security context, ve kterém je přihlášený uživatel udržován a v případě potřeby zasílán na server. 

Uživatelské rozhraní klientských aplikací umožňuje uživateli vytvořit absenci, která pak čeká na schválení. Uživatel s příslušnými právy ji pak může přijmout nebo zamítnout, svou absenci může uživatel zrušit. Typy absencí lze vytvářet a upravovat. K dispozici je také vytváření a úprava zemí. Pro každou zemi jsou typy absencí různé a tedy každý uživatel v závislosti na tom, z jaké pochází země, vidí různé druhy absencí, které může vytvořit. Také si lze zobrazit a upravit svůj uživatelský profil.

Na obrázcích, které zobrazují klientské aplikace je vždy vlevo ukázková aplikace pro Android a vpravo pro Windows Phone.

\subsection{Přihlášení}
Pro vstup do ukázkových aplikací je nejprve třeba se přihlásit. Při spuštění aplikace je zobrazen přihlašovací formulář, který lze vidět na obrázku \ref{img:login}. Po přihlášení je uživatel uložen do aplikačního kontextu, ze kterého ho lze v případě potřeby zase získat. Pro přihlášení lze použít dva uživatele, kteří jsou registrování v databázi na serveru. Jeden z uživatelů má roli admina a může provádět věškeré operace. Druhý je pouhým uživatelem, což má za následek to, že se mu zobrazuje jen určitá podmnožina dat, některé aktivní prvky jsou pro něj vypnuty nebo nemá právo na odeslání formuláře.

\begin{figure}[h!]
\centering
\includegraphics[width=0.7\linewidth]{figures/screenshots/Login}
\caption{Ukázkové projekty - přihlášení}  
\label{img:login}
\end{figure}

\subsection{Správa zemí}
Tato sekce obsahuje list, který zobrazuje již existující země a formulář, kterým země lze vytvořit nebo upravit, což reprezentuje obrázek \ref{img:country}. Pro upravení existující země stačí na zemi kliknout a zobrazí se její náhled ve formuláři níže. Pokud je takto formulář naplněn, tak jeho změna zemi ve formuláři edituje, pokud je formulář prázdný je provedením změn vytvořena země nová. Upravovat a přidávat země smí jen admin, obyčejný uživatel má prvky formuláře neaktivní. V případě, že by došlo k chybě a obyčejný uživatel by přesto poslal data na server, kontroluje se uživatelská role i na serveru. Po přidání či úpravě země se uživateli zobrazí výsledek akce v Androidu formou Toast zprávy, ve WP formou dialogu. Obrazovka disponuje ještě tlačítky na resetovaní a vyčištění formuláře.
\subsection{Správa absencí}
Dalším případem, kdy zavisí na uživatelské roli je obrazovka reprezentující správu absencí. V této sekci je pro zobrazení absencí vytvořen list a pod ním je opět formulář, do kterého lze nahrát data z listu a případně je upravit. Každá uživatelská role vidí v listu různá data, obyčejný uživatel vidí pouze své vytvořené absence a má možnost je zrušit či nechat ve stavu, kdy žádá o schválení, což je zobrazeno na obrázku \ref{img:AbsenceManagement}. Adminstrátor vidí všechny absence včetně svých a může je schvalovat, rušit nebo zamítnout, což lze vidět i na obrázku \ref{img:AbsenceManagementAdmin} . V případě, že admin žádost schválí nebo zamítne, mizí běžnému uživateli absence z listu a již s ní nelze operovat, administrátor vidí absence ve všechy čtyřech stavech. 
\subsection{Správa absenčních typů}
Každá země má své typy absencí, které lze použít. Uživatel má na výběr z typů absencí, které přísluší zemi, která je uživateli nastavena. Data v listu, který je na obrazovce se správou typů, jsou závislé na volbě země, která je reprezentována formulářem o jednom prvku, ze kterého se předá pro tvorbu listu identifikátor vybrané země. Tento formulář je v Android aplikaci ve vrchní části obrazovky. Ve Windows Phone verzi byl zvolen jiný přístup, při pokusu přejít na tuto obrazovku je nejprve zobrazen dialog s formulářem, pro výběr země. Po vybrání země a potvrzení se zobrazí již příslušný list. Formulář, který je v Android aplikaci pod listem je ve WP verzi na vedlejší stránce. Při kliku na položku v listu aplikace přejde sama na vedlejší stránku s naplněných formulářem, při úspěšném přidání či úpravě přejde aplikace z formuláře zpět na list. Tento přístup byl zvolen pro demostraci toho, že vytvořené komponenty lze opravdu vložit do jakéhokoliv jiného prvku uživatelského rozhraní. Příslušná část aplikace je zobrazena na obrázku \ref{img:AbsenceType}.
\subsection{Uživatelský profil}
Každý uživatel má svůj profil, který může v aplikaci upravovat. Tato obrazovka slouží hlavně k demostraci většiny typů aktivních prvků, které umí framework interpretovat. Pole ve formuláři upravující profil také podléhají validacím. Na obrázku \ref{img:profileValidations} lze vidět dvě z nich, validaci pravidel REQUIRED a MAX z tabulky \ref{table:validations}. Také lze vidět, že je Android aplikace převážně v češtině a WP verze v angličtině, za což je zodpovědná lokalizační část frameworku. Některé popisky však ze serveru nepřichází ve formě klíčů k přeložení a nejsou proto lokalizovány. 

\chapter{Závěr}
\section{Budoucí vývoj}
Jelikož uživatelský test měl přívětivé výsledky a všichni testovaní projevili o frameworky zájem, je v plánu oba frameworky nadále vyvíjet. Dalším krokem bude rozšířit řešení i na iOS platformu. Windows Phone verze frameworku běží na platformě C\#, na které běží i Windows desktopové aplikace a proto bude možné pouze s drobnými úpravami rozšířit framework i tímto směrem. Frameworky nyní obsahují množinu validačních pravidel, kterou bychom rádi také rozšířili o další pravidla. Také plánujeme přidat další typy vstupních polí jako například widget pro nahrání fotografie nebo posuvníky. V Android frameworku existuje navíc návrh další komponenty - tabulky, která nyní není plně funkční, neboť bylo rozhodnuto nahradit ji listem. Tabulky bychom ale rádi dotvořili, neboť může být vhodnou komponentou pro zobrazování dat například na tabletech. Také bychom rádi přidali více možností pro úpravu vzhledu komponent, neboť je vzhled velmi důležitou součástí UI. Mobilní zařízení o sobě také ví spoustu užitečných informací jako je například aktuální orientace displeje nebo stav baterie, kterých bychom rádi také při generování a interpretaci UI využili.

Budoucí vývoj bude průběžně testován s uživateli, protože jedním z hlavních cílů frameworku je i jeho použitelnost a hodnocení uživatelů nám výrazně pomůže upravit způsob fungování a využití frameworku.

Ve finále by měly všechny frameworky pro mobilní platformy být plně nasaditelné a připravené k vydání v rámci AspectFaces \cite{aspect-faces}. Mimo mobilních platforem se pod AspectFaces těchto klientských frameworků plánuje více, již existuje interpret pro Swing aplikace na Java SE platformě a je také ve vývoji verze využívající framework AngularJS. Všechny klientské frameworky využívají stejných definic, které přijímají ze serveru, kde lze nadefinovat UI pouze jednou a na jednom místě a klienstké frameworky budou případné změny automaticky reflektovat. To umožní výrazně snížit náklady na vývoj a údržbu klientských aplikací, neboť bude potřeba méně kódu a v případě změny méně úprav.

\section{Zhodnocení práce}
Cílem této práce bylo navrhnout a naimplementovat klienstké frameworky pro dvě mobilní platformy, které získávají a interpretují definice UI ze serveru. Při vytváření frameworků bylo nutné se adaptovat na strukturu těchto definic, které vytváří framework AFRest. Na základě tohoto popisu bylo potřeba vytvořit konkrétní komponenty, případně je naplnit daty a umožnit uživateli s nimi dále pracovat. Při vytváření komponent frameworky zohledňují rozložení komponent, přidružená validační pravidla, bezpečnost, různé typy aktivních prvků, kterými mají být části komponenty reprezentovány a umožňují uživateli nastavit vzhled komponenty. Důležitým požadavkem také byla lokalizace textů, které ze serveru přicázely ve formě klíčů určených k přeložení a v zájmu jednotného použití, umožnit uživateli tuto část framewoku použít i pro překlad svých textů. 

Všechny tyto požadavky byly splněny. Frameworky umí na základě definic sestavit formulář či list, kterým umí správně nastavit rozložení, reprezentovat jejich části správným grafickým prvkem a umí je korektně naplnit daty. Frameworky podporují rozložení částí komponenty do jednoho či dvou sloupců a prvky mohou být vykreslovány ve vertikálním i horizontálním směru. Co se widgetů týče, jsou frameworky schopné vytvořit nejčastěji používané aktivní prvky jako například pole pro text, heslo, číslo nebo třeba datepicker a při vytváření jsou zohledňovány i vlastnosti jako viditelnost nebo zda má být prvek jen pro čtení. Při odeslání formuláře na server je framework schopen ze znalosti popisu modelu, ze kterého je komponenta vytvořena, sestavit data k odeslání ve správné formě, kterou server bez problému přijme. Formulář dále disponuje dalšími funkcemi jako je resetování dat, vyčištění formulářea a jeho validace. Každé pole ve formuláři má přidružená validační pravidla, která jsou kontrolována za pomocí příslušných validátorů. List, který je určen k zobrazování většího množství dat, disponuje možností získat data o jednom jeho prvku a vložit je do formuláře k případné úpravě. Z hlediska bezpečnosti podporují frameworky basic autentizaci. Lokalizační část umožňuje překlad textů přicházejících ze serveru i textů vyskytujících se mimo proces tvoření komponent, tedy například texty tlačítek, které uživatel vytvořil. Lokalizační část dále umožňuje změnit jazyk za běhu aplikace. Frameworky také disponují v rámci skinů značným množstvím funkcí pro úpravu vzhledu komponent.

Z hlediska použitelnosti tvorba a práce s komponentami vyžadují menší množství kódu a jsou relativně jednoduché na naučení, což potvrdil i uživatelský test. Frameworky díky neustálému stahování aktuálních definic ze serveru umí pružně reagovat na změnu v datém modelu na serveru, což je nesporně výhodou, neboť není nutné vydávat při změně modelu novou verzi aplikace. Nevýhodou je, že musí být klientské aplikace připojeny k internetu, což ale v dnešní době není tak velkým problémem. 

Oba frameworky byly vytvořeny a otestovány unit testy, vytvořením ukázkových aplikací a také uživateli, kteří měli na oba frameworky z hlediska použitelnosti velmi pozitivní názor.

%*****************************************************************************
% Seznam literatury je v samostatnem souboru reference.bib. Ten
% upravte dle vlastnich potreb, potom zpracujte (a do textu
% zapracujte) pomoci prikazu bibtex a nasledne pdflatex (nebo
% latex). Druhy z nich alespon 2x, aby se poresily odkazy.

% originally following specification for bibliography formating was used
%\bibliographystyle{abbrv}

% Here is an improvment by Petr Dlouhy (April 2010).
% It is mainly for supervisors who expect Czech fomrating rules for references
% Additional feature is live url addresses to sources from your pdf file
% It requires the file csplainnat.bst (included in this sample zipfile).
\bibliographystyle{csplainnat}

%bibliographystyle{plain}
%\bibliographystyle{psc}
{
%JZ: 11.12.2008 Kdo chce mit v techto ukazkovych odkazech take odkaz na CSTeX:
\def\CS{$\cal C\kern-0.1667em\lower.5ex\hbox{$\cal S$}\kern-0.075em $}
\bibliography{reference}
}
%%%%%%%%%%%%%%%%%%%%%%%%%% 
% vše co následuje bude uvedeno v přílohách
\appendix	

\printnomenclature
\label{apx:zkratky}
\chapter{Seznam použitých zkratek}

\begin{description}
\item[CUI] Character User Interface
\item[GUI] Graphical User Interface
\item[UI] User Interface
\item[REST] Representational State Transfer
\item[URI] Uniform Resource Identifier
\item[HTTP] Hypertext Transfer Protocol
\item[JSON] JavaScript Object Notation
\item[XML] Extensible Markup Language
\item[XAML] Extensible Application Markup Language 
\item[Java SE] Java Platform Standart Edition
\item[Java EE] Java Platform Enterprise Edition
\item[URL] Uniform Resource Locator
\item[PHP] Personal Home Page
\item[HTML] HyperText Markup Language
\item[SQL] Structured Query Language
\item[CSS] Cascading Style Sheets
\item[JPA] Java Persistence API
\item[LGPL] GNU Lesser General Public License 
\item[UML] Unified Modeling Language
\item[C\#] C Sharp
\item[WP] Windows Phone


\end{description}
\label{apx:install}
\chapter{Instalační a uživatelská příručka}
Všechny android aplikace jsou Gradle projekty, tedy android framework, který byl zkompilován do AAR souboru, lze přidat jako knihovnu nebo jako Gradle závislost. Windows phone verze byla zkompilována do DLL souboru a přidává se do projektů jako refrence. Podrobnější postup integrace frameworků do projektu a také způsoby jejich použití jsou popsány v uživatelské příručce na přiloženém CD.

\section{Integrace AFAndroid frameworku}
Nejjednoduším způsobem je intergrovat framework ve vývojovém prostředí Android Studio. To nabízí možnost přidat framework jako nový modul a všechny potřebné gradle závislosti se vytvoří a nakonfigurují automaticky. V případě, že bude framework vložen tímto způsobem vytvoří se v kořenovém adresáři projektu balíček obsahující importovaný AAR soubor, IML soubor popsiující právě vytvořený modul a soubor build.gradle, ve kterém jsou dva důležité řádky zobrazené v ukázce kódu \ref{code:moduleGradle}
\begin{lstlisting}[caption={Gradle soubor pro build v balíčku s novým modulem},
label={code:moduleGradle}, basicstyle=\footnotesize, frame=single]
configurations.create("<nazev konfigurace>") 
artifacts.add("<nazev konfigurace>", file('<jmeno souboru>.aar'))
\end{lstlisting}

Aplikace, do které je framework přidáván má také svůj soubor build.gradle \ref{code:appGradle}, do kterého přibudou následující závislosti. Také je nutné přidat závislost pro knihovnu GSON, která je frameworkem vnitřně používána. 
\begin{lstlisting}[caption={Gradle soubor pro build aplikace, do které je framework integrován},
label={code:appGradle}, basicstyle=\footnotesize, frame=single]
dependencies {
	compile fileTree(include: ['*.jar', '*.aar'], dir: 'libs')  
	//dalsi zavislosti     
	compile project(":<nazev importovaneho modulu>") 
	compile 'com.google.code.gson:gson:<verze napr. 1.7.2.>'
} 
\end{lstlisting}

Dále do souboru settings.gradle \ref{code:settingsGradle} přibude následující řádek
\begin{lstlisting}[caption={Gradle soubor s nastaveními aplikace, do které je framework integrován },
label={code:settingsGradle}, basicstyle=\footnotesize, frame=single ]
include ':<nazev modulu s aplikaci>', ':<nazev importovaneho modulu>' 
\end{lstlisting}

Jelikož aplikace pro vytvoření komponent komunikuje se serverem, je nutné přidělit aplikaci přístup k internetu, což se nastaví v manifestu aplikace, tedy ve souboru AndroidManifest.xml, přidáním následujícího řádku, který lze vidět v ukázce kódu \ref{code:internetManifest}.
\begin{lstlisting}[caption={Android manifest - udělení přístupu k internetu},
label={code:internetManifest}, basicstyle=\footnotesize, frame=single]
<uses-permission android:name="android.permission.INTERNET" /> 
\end{lstlisting}

Nyní stačí provést build projektu a framework je připraven k použití.

\section{Integrace AFWinPhone frameworku}
Pro vývoj Windows Phone aplikací se nejčastěji používá vývojové prostředí Visual Studio, pro které je popsán způsob vložení frameworku. Ve Visual Studiu má každá aplikace ve svém kořenovém adresáři položku References, která obsahuje knihovny používané aplikací. Postup vložení je následující:
\begin{enumerate}
\item Klikněte pravým tlačítkem na References a poté na Add Reference.
\item Objeví se dialog, kde klikněte na Browse… a vyhledejte příslušný DLL soubor
\item Zkontrolujte, zda je nalezený soubor v dialogu zaškrtnutý.
\item Povtrďte volbu stisknutím OK.
\item Proveďte build projektu. 
\end{enumerate}
Po těchto krocích by měla být knihovna připravena k použití.

\section{Ukázkové projekty}
Ukázkové projekty pro Android i pro Windows phone jsou přiloženy na CD. Jelikož je potřeba pro běh klientských aplikací server poskytující potřebné definice, je na CD přiložena i serverová část AFServer vytvořená jako ukázkový projekt pro demonstaci frameworků AFRest a AFSwinx. Způsob spuštění lze nalézt v uživatelské příručce těchto frameworků, která je na CD rovněž přiložena. \cite{tomasek-thesis}. Pro každý ukázkový projekt jsou na CD dva soubory, jeden z nich má definovaná připojení pro localhost a je nutné si pro jejich vyzkoušení spustit server lokálně dle zmíněného příručky pro AFRest a AFSwinx a klienstkou aplikaci nejlépe v emulátoru na stejném stroji. Druhý projekt komunikuje se serverem vystaveným na adrese \url{toms-cz.com/AFServer} a lze tak aplikace nahrát a vyzkoušet je přímo na mobilních zařízeních. 
\subsection{Nahrání ukázkové aplikace do Android zařízení}
Pro nahrání aplikace do zařízení postačí aplikaci ve formě APK souboru vložit přes USB kabel do zařízení a provést následující kroky.
\begin{enumerate}
\item Povolte v nastavení zařízení instalaci aplikací ze zdrojů jiných než Play Store.
\item Nalezněte aplikaci v uložišti zařízení a spusťte instalaci.
\item Povolte všechna oprávnění, která aplikace vyžaduje.
\item Po instalaci aplikace se připojte k internetu a aplikaci spusťte.
\end{enumerate}
\subsection{Nahrání ukázkové aplikace do Windows Phone zařízení}
Pro nahrání aplikace do zařízení je třeba nahrát APPX soubor do zařízení prostřednictvím aplikace Application Deployment. To se provede následujícími kroky.
\begin{enumerate}
\item Nalezněte v počítači aplikaci Application Deployment. Měla by se nainstalovat společně s Visual Studio. Aplikaci spusťte.
\item Vyberte zařízení připojené přes USB kabel.
\item Najděte cestu k příslušnému APPX souboru. 
\item Klikněte na Deploy. Aplikace se do zařízení nahraje a poté ji můžete spustit.
\end{enumerate}



\label{apx:uml}
\chapter{UML diagramy a obrázky}
V této sekci naleznete použité UML diagramy a velké obrázky, na které bylo v textu odkazováno.

\begin{figure}
\begin{center}
\includegraphics[angle=270]{figures/createFormActivityDiagram}
\caption{Diagram aktivit popisující proces tvorby formuláře}
\label{img:createFormActivityDiagram}
\end{center}
\end{figure}

\begin{figure}
\begin{center}
\includegraphics{figures/formWorkActivityDiagram}
\caption{Diagram aktivit popisující práci s formulářem}
\label{img:formWorkActivityDiagram}
\end{center}
\end{figure}

\begin{figure}
\begin{center}
\includegraphics{figures/useCaseModel}
\caption{Model případů užití frameworku}
\label{img:useCaseModel}
\end{center}
\end{figure}



\label{apx:code}
\chapter{Ukázky zdrojových kódů}
V této sekci naleznete ukázky zdrojového kódu, na které bylo v textu odkazováno.
\newpage
\begin{lstlisting}[caption=Upravená šablona date.xml,
label={code:dateXml}, basicstyle=\footnotesize]
<widget>
	<widgetType>calendar</widgetType>
	<fieldName>$field$</fieldName>
	<label>$label$</label>
	<readonly>false</readonly>
	<validations>
		<required>$required$</required>
		<lessthan>$lessthan$</lessthan>
	</validations>
	<fieldLayout>
		<layoutOrientation>$layoutOrientation$</layoutOrientation>
		<labelPossition>$labelPossition$</labelPossition>
		<layout>$layout$</layout>
	</fieldLayout>
</widget>
\end{lstlisting}

\begin{lstlisting}[caption=Ukázka použití anotace @UILessThan,
label={code:uiLessThanUsage}, basicstyle=\footnotesize]
@Entity
public class AbsenceInstance {
	...
	@Temporal(value = TemporalType.DATE)
	private Date startDate;
	@Temporal(value = TemporalType.DATE)
	private Date endDate;
	...
	@UILessThan(value="endDate")
	public Date getStartDate() {
        	return startDate;
	}
	...
}
\end{lstlisting}

\begin{lstlisting}[float=h, caption=Ukázka XML specifikace zdrojů,
label={code:xmlSource}, basicstyle=\footnotesize]
<?xml version="1.0" encoding="UTF-8"?>
<connectionRoot xmlns:xsi="http://www.w3.org/2001/XMLSchema-instance">
	<connection id="personProfile">
		<metaModel>
			<endPoint>toms-cz.com</endPoint>
			<endPointParameters>
				/AFServer/rest/users/profile
			</endPointParameters>
			<protocol>http</protocol>
			<port></port>
			<header-param>
				<param>content-type</param>
				<value>Application/Json</value>
			</header-param>
		</metaModel>
		<data>
			<endPoint>toms-cz.com</endPoint>
			<endPointParameters>
				/AFServer/rest/users/user/#{username}
			</endPointParameters>
			<protocol>http</protocol>
			<port></port>
			<security-params>
				<security-method>basic</security-method>
				<userName>#{username}</userName>
				<password>#{password}</password>
			</security-params>
		</data>
		<send>
			<endPoint>toms-cz.com</endPoint>
			<endPointParameters>
				/AFServer/rest/users/update
			</endPointParameters>
			<protocol>http</protocol>
			<method>post</method>
			<port></port>
			<security-params>
				<security-method>basic</security-method>
				<userName>#{username}</userName>
				<password>#{password}</password>
			</security-params>
		</send>
	</connection>
</connectionRoot>
\end{lstlisting}



\chapter{Obsah přiloženého CD}
\textbf{\large Tato příloha je povinná pro každou práci. Každá práce musí totiž obsahovat přiložené CD. Viz dále.}

Může vypadat například takto. Váš seznam samozřejmě bude odpovídat typu vaší práce. (viz \cite{infodp}):

\begin{figure}[h]
\begin{center}
\includegraphics[width=14cm]{figures/seznamcd}
\caption{Seznam přiloženého CD --- příklad}
\label{fig:seznamcd}
\end{center}
\end{figure}

Na GNU/Linuxu si strukturu přiloženého CD můžete snadno vyrobit příkazem:\\ 
\verb|$ tree . >tree.txt|\\
Ve vzniklém souboru pak stačí pouze doplnit komentáře.

Z \textbf{README.TXT} (případne index.html apod.)  musí být rovněž zřejmé, jak programy instalovat, spouštět a jaké požadavky mají tyto programy na hardware.

Adresář \textbf{text}  musí obsahovat soubor s vlastním textem práce v PDF nebo PS formátu, který bude později použit pro prezentaci diplomové práce na WWW.

\end{document}
