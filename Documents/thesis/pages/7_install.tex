\chapter{Instalační a uživatelská příručka}
Framework byl vytvořen jako Maven projekt. Do aplikace ho lze přidat buďto jako knihovnu nebo jako Maven závislost. Způsob, jakým lze integrovat projekt a jak ho používat je detailně popsán v uživatelské příručce na přiloženém CD. 
\section{Maven závislosti}
Nejprve je potřeba provést build frameworku. Zdrojové kódy jsou na přiloženém CD. Framework zatím není v žádném z veřejně dostupných repositářích. Poté lze na serveru stranu přidat následující závislosti:
\begin{lstlisting}[caption={Závislosti na serveru},
label={code:mavenDependency}, basicstyle=\footnotesize]
<dependency>
	<groupId>com.tomscz.af</groupId>
	<artifactId>AFRest</artifactId>
	<version>0.0.1-SNAPSHOT</version>
</dependency>
<dependency>
	<groupId>com.codingcrayons.aspectfaces</groupId>
	<artifactId>javaee-connector</artifactId>
	<version>1.5.0-SNAPSHOT</version>
</dependency>
<dependency>
	<groupId>com.codingcrayons.aspectfaces</groupId>
	<artifactId>annotation-descriptors</artifactId>
	<version>1.5.0-SNAPSHOT</version>
</dependency>
\end{lstlisting}
Repozitář pro aspectFaces je zde:
\begin{lstlisting}[caption={AspectFaces repozitář },
label={code:mavenAspectFacesRepo}]
<repository>
	<id>codingcrayons-repository</id>
	<name>CodingCrayons Maven Repository</name>
	<url>http://maven.codingcrayons.com/content/groups/public/</url>
</repository>
\end{lstlisting}
Do složky WEB-INF je potřeba rozbalit soubor templates.zip, v kterém je předpřipravená konfigurace a do web.xml je potřeba přidat listener, který provede nastavení AspectFaces během startu. 
\begin{lstlisting}[caption={AspectFaces listener},
label={code:mavenAspectFacesBootStrap}, basicstyle=\footnotesize]
<listener>
	<!-- Include Aspect Faces listener to perform proper framework initialization 
	during application start -->
	<listener-class>com.codingcrayons.aspectfaces.plugins.j2ee.AspectFacesListener</listener-class>
</listener>
\end{lstlisting}
Na klientskou stranu je potřeba přidat následující závislost:
\begin{lstlisting}[caption={Závislost na klientské straně},
label={code:mavenAFSwinx}, basicstyle=\footnotesize]
<dependency>
	<groupId>com.tomscz.af</groupId>
	<artifactId>AFSwinx</artifactId>
	<version>0.0.1-SNAPSHOT</version>
</dependency>
\end{lstlisting}
\section{Ukázkový projekt}
Ukázkový projekt se svojí klientskou a serverovou částí je již vytvořen a přiložen na CD. Serverovou část lze bez dodatečné konfigurace spustit na serveru GlassFish V3. Ukázkový projekt je možné spustit i na GlassFish V4 nicméně, při deployi aplikace je v konzoli zobrazena chyba, avšak aplikace je plně funkční. Na toto chování byl založen bug, který není prozatím vyřešen. Je potřeba provést následující:
\begin{enumerate}
\item Rozbalte aplikační server GlassFish, který je přiložen na CD nebo si stáhněte verzi 3 z http://www.oracle.com/technetwork/middleware/glassfish/downloads/java-archive-downloads-glassfish-419424.html Verzi 4 lze stáhnout 
\\z http://dlc.sun.com.edgesuite.net/glassfish/4.1/release/glassfish-4.1.zip
\item Rozbalte soubor, ve složce bin spoustě utilitu asadmin napsáním asadmin
\item Vložte následující příkaz: start-domain domain1
\item Vložte následující příkaz deploy PATHTOFILE/AFServer.war
\item Otevřete webový prohlížeč na adrese http://localhost:8080/AFServer - zobrazí se text: I am alive. Serverová strana nedisponuje grafickým uživatelským rozhraním. Funkčnost můžete otestovat pomocí rest klienta například na adrese 
\\http://localhost:8080/AFServer/rest/country/list - content-type: application/json metoda GET.
\end{enumerate}
Nyní je potřeba spustit klientskou část aplikace. Ve složce s Showcase.jar, který je přiložen na CD spusťte java -jar Showcase.jar . Aplikace bude spuštěna.


