\chapter{Popis problému a specifikace cíle}
\section{Popis problematiky}
Softwarové systémy jsou určeny k tomu, aby méně či více úspěšně poskytovali uživateli nástroj, který mu pomůže s řešením problémů. Systém tedy musí komunikovat s uživatelem. K tomuto účelu se využívá uživatelské rozhraní. Vývoj uživatelského rozhraní zabere přibližně 60% času, který je určen na vývoj konkrétního systému. Tento údaj se samozřejmě může lišit v závislosti na účelu a velikosti systému. Při tvorbě uživatelského rozhraní se obvykle zaměřujeme na použitelnost. Zde se zaměřujeme již na cílovou skupinu. Provádíme testy použitelnosti a na základě těchto a dalších testů jsme schopni určit, zdali je návrh použitelný či nikoliv. Důvodem tohoto testování je fakt, že obvykle systém stavíme pro uživatele ne obráceně. Z výše uvedených skutečností vyplívá, že je potřeba uživatelské rozhraní důkladně testovat, aby bylo pro cílovou skupinu správně použitelné. Bohužel, když se mluví o uživatelském rozhraní, tak se často zapomíná na to, že toto rozhraní se musí nejen vytvořit, ale také udržovat. Softwarový systém tráví většinu svého života v udržovacím režimu, kterému se říká support. V této fázi přichází na systém mnoho požadavků, které musí být proveditelné a to za přijatelné náklady. Nedílnou součástí jsou změny, které se týkají databázového modelu a obvykle tyto změny musí reflektovat UI. Podívejme se proto na systém z pohledu vývojáře. Systém pro něj musí být snadno udržovatelný, změny lehce proveditelné a bez větších dopadů na systém. V tomto případě by bylo vhodné reflektovat tyto změny v UI. 
\subsection{Typy uživatelských rozhraní}
Jak již bylo zmíněno, tak uživatelské rozhraní se testuje na základě typu aplikace a jejím použití. Je také důležité vzít v potaz zařízení, na kterém je aplikace provozována. Může se jednat o desktopovou, mobilní či serverou aplikaci. V každém z výše uvedených případů bude návrh uživatelského rohraní podíměn jinými faktory, které jsou specifické pro dané zařízení. Těmito faktory jsou způsoby, jakým se aplikace ovládá, prostředí, v kterém se uživatel právě nachází a účel, ke kterému je aplikace určena. Například aplikace na mobilních zařízení nemusí podporovat klávesové zkratky, ale mohla by podporovat gesta. Obdobně aplikace použitá na desktopu může počítat s použití myše, touchpadu, klávesnice - jak standardní tak dotykové či jiného externího zařízení. Je tedy zřejmé, že uživatelské rozhraní je kromě jeho účelu podmíněno i zařízením, na kterém je používáno.
\subsection{Získávání dat}
 